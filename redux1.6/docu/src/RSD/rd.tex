\section{The RD-Subsystem}
\label{se:rd}

The ReDuX random term generation program consists of several files:
\begin{itemize}
\item {\tt rd.h} which contains declarations of global
   variables and some macro definitions.
\item {\tt trand.ald} holds the algorithms for initializing
   the signature descriptor variables, the global data arrays, for
   mapping natural numbers to terms and to compute random terms.
\item {\tt trd.front}, {\tt trd.end} and {\tt trdsub.ald} contain the main
   algorithm for the ReDuX random term generation demo program and some
   algorithms for generating sequences of random terms (which can be
   written to a file) and for making statistics.
\end{itemize}

To use the the random term generator, several things need to be done:
\begin{enumerate}
\item {\tt rd.h} must be included within the main program to
   define the global variables in the main algorithm. If any of the
   global variables or macro definitions are used outside the main
   algorithm, {\tt rd.h} must additionally be included outside
   the main algorithm before the first occurrence of a global variable
   or macro. Appropriate values for {\tt STACK}-- and {\tt SPACE}--size
   have to be chosen.
\item The ReDuX-system (including the {\tt it}-module) must be
   initialized by a call to {\tt BGNRWD} with an argument greater than
   or equal to 5.
\item The signature descriptor variables must be initialized. This is
   done by a call to the function {\tt SDINIT} which has one argument,
   the ReDuX data type as returned by {\tt DTPRS}.
\item The global variable {\tt DIPOS} (data initialized for number of
   positions) has to be set to $0$.  (It contains the number of
   positions up to which the global data arrays are initialized.) The
   global variable {\tt DIDPT} (data initialized for depth) has to be
   set to $-1$. (It contains the depth up to which the global data arrays
   are initialized.)
\item The global variable {\tt TGTRC} (term generation trace) must be
   initialized to 0 (no term generation trace) or 1 (the term
   generation trace is turned on).
\end{enumerate}

After those mandatory initializations, the further initializations
depend on the algorithms to be used:
\begin{itemize}
\item If the high-level random term generation routines {\tt TRANDE}
   ({\tt TRANDD}) are used, nothing else must be initialized. They
   are called with one argument, the maximal extent (depth) the
   random term should have.
\item If the algorithm {\tt NT2TE}  is called explicitly, the global
   arrays must be initialized by a call to {\tt GDINIM} (global data
   initialization, show statistical messages).  Either the largest
   number of positions {\tt p} of the terms to be computed or the
   largest number {\tt n} which should be mapped to a term is known.
   In the former case, the globals arrays are initialized with {\tt
   GDINIM(p,0,0)}. In the latter case the global arrays are initialized
   with {\tt GDINIM(IADD(n,1),0,1)}.
\item If the algorithm {\tt NT2TD}  is called explicitly, the global
   arrays must be initialized by a call to {\tt GDINIM} (global data
   initialization, show statistical messages).  Either the largest
   depth {\tt d} of the terms to be computed or the largest number {\tt
   n} which should be mapped to a term is known.  In the former case,
   the globals arrays are initialized with {\tt GDINIM(d,1,0)}. In the
   latter case the global arrays are initialized with {\tt
   GDINIM(IADD(n,1),1,1)}.
\end{itemize}
The advantage of a direct call to {\tt NT2TE} ({\tt NT2TD}) is a slight
(usually negligible) increase in performance. For an example of a direct
call, look at the algorithms {\tt TRGENE} ({\tt TRGEND}) in file {\tt
trdsub.ald}. It can be copied to other programs if step 6 is
replaced by suitable statements. The random term generation program
allows to compare both possibilities by using the menu items {\tt r}
and {\tt R}. The difference between the two algorithms is rather
small.

The variables in a term returned by {\tt NT2TE} and {\tt NT2TD} (and
subsequently {\tt TRANDE} and {\tt TRANDD}) which have the same {\tt
TKIND}--field are not independent, i.\,e.\ they share the same {\tt
TCONT}--field. To get terms with independent variables, the generated
terms can either be copied using algorithms from the {\tt
tpcy}--module, written to a file and read-in by another program or step
2 in algorithm {\tt NT2T2E} ({\tt NT2T2D}) can be modified to return
variables with distinct {\tt TCONT}--fields.

