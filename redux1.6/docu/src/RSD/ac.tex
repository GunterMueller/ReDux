\section{The AC-Subsystem} \label{se:ac}

The {\it ac}-subsystem provides all algorithms needed for a completion
procedure modulo an equational theory \`{a} la Peterson and Stickel
\index{completion!Peterson-Stickel}
\cite{PetersonStickel:81}.
Most of the algorithms are implemented such that they can be extended to
other (operator dependent) theories.
So far the only equational theories  implemented are those described by 
commutative (C) operators and by
\index{commutative}
associative and commutative (AC) operators.
\index{AC}
\index{associative and commutative}
C-operators have entry DCO in TUSTAT and
\index{operators!C{-}}
AC-operators have the entry DAC in TUSTAT.
\index{operators!AC{-}}

Therefore the algorithms for equality tests, unifications, matches are
extended to operate modulo commutativity and
modulo associativity and commutativity.
In addition the AC-reduction relation described in
\index{reduction!AC{-}}
\cite{PetersonStickel:81} is implemented.
This reduction relation requires that the term rewriting system is
AC-compatible and therefore AC-extension rules must be added to some 
\index{rule!AC-extension}
rewrite systems.
In the completion procedure, the collapse inference rule applies to extension
rules only when it applies to the original rule.
The extension rules play also a r\^{o}le when computing AC-critical pairs.

All algorithms in the {\it ac}-subsystem are designed to handle mixed terms.
In general external terms are handled like constants.
External term equality is determined using the function whose key is stored
in TUSTAT.

\subsection{The {\tt actto}-Module}

The module {\tt actto} provides procedures to determine the (structural)
equality modulo AC of two terms,
\index{term!AC-equality}
i.\,e., the position and the `tree structure' of term parts with the same
AC-operator do not matter.
Therefore it is convenient to represent terms with AC-operators in a
{\em flattened} form.
\index{term!flattened}
Namely AC-operators are considered to be of variable rank (arity) and no
argument of an AC-operator  $f$ may have the top operator $f$.

In addition, there must be a total ordering $\leq_{\mbox{\scriptsize tto}}$
\index{ordering!total}
such that \( s \leq_{\mbox{\scriptsize tto}} t \) and
\( t \leq_{\mbox{\scriptsize tto}} s \) iff \( s =_{AC} t \),
all variables are smaller than non-variable terms and
\( s <_{\mbox{\scriptsize tto}} t \) if the top operator of $s$ is less than
the
top operator of $t$ w.\,r.\,t.\ a total ordering on constants and operators.

The implementation of the total term ordering $\leq_{\mbox{\scriptsize tto}}$
depends on the storage model.
In particular it is assumed that the garbage collection 
\index{garbage collection}
\index{SPACE}
does not change the relative order of list cells on the heap (SPACE).
The relative order of external objects under the same coercion operator 
is determined using the function key stored as a property of the external type
\index{type!external}
under indicator DXLA.

TFCONT computes the flattened contents of a term with associative
\index{term!flattened}
(and commutative) top operator.
The function TTO returns 0, -1 or 1 if the first argument is (w.\,r.\,t.\
$\leq_{\mbox{\scriptsize tto}}$) equal to, less than or greater than the
second argument.

The module {\tt actto} contains further algorithms depending on the total
term ordering $\leq_{\mbox{\scriptsize tto}}$.
There is a sorting algorithm for term lists (TLBSD), a multiset ordering
for term lists (TLTMSO), a term list membership test (TLMTTO) w.\,r.\,t.\ 
$=_{\mbox{\scriptsize tto}}$ (i.\,e.\ AC-equality),
and a difference function for two ordered term lists (TLODIF).
The procedure TLOEML computes the multiplicities of all 
$=_{\mbox{\scriptsize tto}}$-equal elements in a list.

\subsection{The {\tt acde}-Module}

According to Stickel \cite{Stickel:81} the problem of unifying two terms
\index{unification!AC{-}}
modulo associativity and commutativity can in a first step be reduced to
solving a linear Diophantine equation over the set of positive
\index{equation!linear Diophantine}
integers.
The module {\tt acde} provides algorithms for solving such Diophantine
equations, imposing certain constraints on their solutions and
algorithms to translate the numerical solutions to substitutions.

A linear Diophantine equation
\[ a_1 x_1 + \ldots + a_m x_m = b_1 y_1 + \ldots + b_n y_n \]
is represented by two lists X and Y of period 2:
\[ \begin{array}{lcl}
   X = (a_1, x_1, \ldots, a_m,x_m), \\
   Y = (b_1, y_1, \ldots, b_n,y_n)
   \end{array}
\]
where the coefficients $a_i$, $b_i$ are $\beta$-integers and the 
indeterminates $x_i$, $y_i$ are represented as different pointers.
In the context of AC-unification the indeterminates point to subterms of the
original unification problem.

A solution of a Diophantine equation is a list
\[ S = ( s_1, z_1, \ldots, s_k,z_k) \]
where $z_1, \ldots, z_k$ are the indeterminates occurring in the equation and 
\( z_i=s_i \) for \( 1 \leq i \leq k \) is an assignment which solves the 
Diophantine equation.
Sets of solutions are represented by lists of solutions.

The algorithm DEBSS computes the set of basic solutions of a linear
Diophantine equation over the natural numbers.
It uses the completion algorithm described in \cite{Fortenbacher:83}.

DESSRI deletes solutions from a (basic) solution set which assign a number not
equal to one to an indeterminate represented by a non-variable term.

DESSNS behaves like the coroutines in the {\it co}-subsystem.
It computes one by one the index of all solutions composed
from the basic solution set which meet certain conditions:
all assignment values of a solution must be positive and all those
assignment which are associated with an indeterminate pointing to
a non-variable term must be one.
The solution index is an integer which encodes the sum of which basic
solutions represents a valid (candidate) substitution for the unification 
problem.
DESSTV computes the terms of this substitution.

\subsection{The {\tt acu}-Module}

The module {\tt acu} contains unification algorithms for terms which may 
\index{unification!AC{-}}
contain associative commutative operators.
In contrast to {\tt tcu} the most general unifier modulo AC ($\mbox{mgu}_{AC}$)
--- which generally consists of more than one substitution --- is computed.
\index{substitution}
Therefore the unification procedures have no side effects.
TLMGUE computes the $\mbox{mgu}_{AC}$ for two lists of terms and TMGUAC
computes the $\mbox{mgu}_{AC}$ for two terms which have the same AC-operator 
at their top positions.

Applying two different substitutions from the $\mbox{mgu}_{AC}$ to a unified
term may result in a single common instance.
Therefore TMGCIE first computes the $\mbox{mgu}_{AC}$ and then computes a
minimal set of most general common instances.
Warning: minimizing the $\mbox{mgci}_{AC}$ is a very expensive operation.

\subsection{The {\tt acu1}-Module}

The module {\tt acu1} exports the procedure TCSIE that given a term
$t$ and a set of substitutions $\Sigma$ computes a minimal subset
\( T \subseteq \{ t\sigma \mid \; \sigma \in \Sigma \} \) such that
each \( t' \in \{ t\sigma \mid \; \sigma \in \Sigma \} \) is an instance of
a \( t^* \in T \).
The module introduces a new data structure {\em augmented term}
that is a quintuple $(w,t,V,b,\sigma)$ where $t$ is a term $w$ is its
weight, $V$ is its list of variables, $b$ is TRUE iff $t$ is linear and
$\sigma$ is a substitution.

\subsection{The {\tt acm}-Module}

The module {\tt acm} contains matching algorithms for matching modulo AC.
\index{match!AC{-}}
As with AC-unifications the solution of an AC-matching problem is generally 
a complete set of matching substitutions (matchers).
\index{substitution} 
However in the context of term rewriting, one is only interested in a single
matching substitution.
 
TLMSE computes a matcher for two term lists, TMSAC returns a matcher for two
terms with common top AC-operator and TSMTHE tests whether one term is an
AC-instance of a subterm of another term.
TSMTHE is thought as a reducibility test (compare with TSMTH from {\tt tcu}).

The matching algorithms in the {\tt acm}-module are based on those 
described in \cite{Hullot:79} and \cite{KuechlinLeitsch:88}.
However no backtracking is implemented, so the algorithms
TLMSE and TMSAC are based on algorithms for computing complete matchers
(TLMSSE, TMSSAC, VLMSAC with the last argument equal to FALSE).
None of the AC-matching algorithms has side effects.

\subsection{The {\tt acr}-Module}

The module {\tt acr} contains algorithms to implement reduction relations
\index{reduction!AC{-}}
modulo AC based on matching subterms (\cite{PetersonStickel:81}).

APUTXR adds an extension rule as property under indicator DXR to a rule with
\index{rule!extension}
an AC-top operator at its left-hand side,
if the conditions described in \cite[pp 255]{PetersonStickel:81} are
fulfilled.

TIRDSE normalizes a term modulo AC w.\,r.\,t.\ to a list of rewrite rules.
\index{normalization}
Like TIREDS, TIRDSE uses an innermost reduction strategy but the marking
mechanism had to be modified to take into account that the notion of a position
does not apply for the parts of a term whose positions are labeled by the
same AC-operator. 
Therefore if a  flat term with an AC-top operator were to be marked the 
mark is is shifted below the AC-operator.

There is also a function TNDFLT that destructively flattens a non-atomic
\index{term!flat}
term.

\index{substitution}
SBERDE reduces the terms of a substitution.

\subsection{The {\tt accp}-Module}

The module {\tt accp} handles (generalized) AC-critical pairs and 
queues of AC-critical pairs.
\index{critical pair!AC{-}}
The data structure used for AC-critical pairs is the one for
critical pairs as defined in {\tt tccp}.
For AC-critical pairs the extension rules must be considered and again
the notion of a position does not apply for the parts of a term made 
of AC-operators.

The procedure ACPAC computes all AC-critical pairs between a rule and
a  set of rules.
the procedure ACPACG computes all generalized  AC-critical pairs
\index{critical pair!generalized}
between a rule and a  set of rules.
For the theory of generalized critical pairs see \cite{Buendgen:94}.
For ACPACG to work properly all participating rules are assumed to
have the list of linearly constrained subterms of their left-hand sides
\index{subterm!linearly constrained}
stored as a property under indicator DST.
Both ACPAC and ACPACG have a Boolean forth parameter that determines
whether intrinsic pairs are to be computed or not.
In contrast to the case of the empty theory (ASCPS, ASCPSU), the AC-critical
pairs computed by ACPAC and ACPACG are normalized by a
term rewriting system depicted in their third argument.

CPLCLE partitions a list of AC-critical pairs into a list of pairs with 
reducible terms and into a list of irreducible pairs.

CPLOW8 updates the the weights of a list of critical pairs depending
on a term ordering $\succ$ denoted by a string.
If a pair is comparable w.\,r.\,t.\ $succ$ the weight of the larger
term becomes the weight of the pair otherwise it becomes the maximum of the
two terms' weights.

CPSTST tests whether one critical pair subsumes (i.\,e., matches)
another critical pair.
Using this test CPLIST inserts a critical pair into an 
ordered (w.\,r.\,t.\ their weights) list of critical pairs un less it is
subsumed by one of the pairs in the list.
On the other hand pair of the list that are subsumed by the new pair
may be deleted.

There is a trace facility built into the critical pair computation
procedures.
The critical pair trace is controlled by the global variable CPTRC
which must externally be declared safe whenever referenced.
There are two trace levels.
A strong trace (CPTRC $= 2$) which prints the details of each superposition 
and weak trace (CPTRC $= 1$) which only lists unresolved critical pairs.

\subsection{AC-Completion and Fertilization}

The Peterson-Stickel AC-completion procedure DTKBAC is implemented in the
\index{completion!AC{-}}
\index{completion!Peterson-Stickel}
{\tt accmp}-module.
It uses  the strategy that is sketched in
\index{strategy}
Figure~\ref{fi:accmpl}.
Figure~\ref{fi:accmpl}.
\begin{figure}[ht]
\begin{center}
\fbox{
\begin{minipage}{5.5in} %necessary if there is a footnote within the figure
\begin{quote}
\begin{center} ${\cal R} \leftarrow$ {\bf AC-COMPLETE}$({\cal E},\succ)$
\end{center}
[Completion procedure. \\
$\cal E$ is a set of equations and $\succ$ is a terminating term ordering.
Then ${\cal R}$ is a canonical term rewriting system with
\( =_{\cal R} \; = \; =_{\cal E} \).]
\begin{description}
 \item[{(1) [Initialize.]}] \(  {\cal R} := \emptyset \).
 \item[{(2) [Simplify.]}] {\bf while} the {\em simplify}-inference rule
   applies {\bf do} \\
  \( ({\cal E};{\cal R}) := Simplify(({\cal E};{\cal R})). \)
 \item[{(3) [Delete.]}] {\bf while} the {\em delete}-inference rule
   applies {\bf do} \\
  \( ({\cal E};{\cal R}) := Delete(({\cal E};{\cal R})). \)
 \item[{(4) [Stop?]}] {\bf if} \(  {\cal E} = \emptyset \) {\bf then return}
      ${\cal R}$ and {\bf stop}.
 \item[{(5) [Orient.]}] Let \( a \leftrightarrow b \in {\cal E} \) then
      \( {\cal E} := {\cal E} \setminus \{ a \leftrightarrow b \} \); \\
      {\bf if} \( a \succ b \) {\bf then} \(  \bgn l := a; r  := b \nd \) \\
      {\bf elsif} \( b \succ a \) {\bf then} \(  \bgn l := b; r  := a \nd \) \\
      {\bf else stop} with failure; \\
      \( {\cal R} := {\cal R} \cup \{ l \rightarrow r \} \).
 \item[{(6) [Extend.]}] 
    \( {\cal R} := {\cal R} \cup \{ l \rightarrow r \} \; \cup \;
    \) AC-extension of $l \rightarrow r$
 \item[{(7) [Collapse.]}] {\bf while} the {\em collapse}-inference rule
     applies {\bf do} \\
    \( ({\cal E};{\cal R}) := Collapse(({\cal E};{\cal R})). \)
 \item[{(8) [Subconnectedness criterion.]}] \null \\
    \( {\cal E} := {\cal E} \setminus \{ \mbox{Equations originating from a
    collapsed rule} \} \).
 \item[{(9) [Compose.]}] {\bf while} the {\em compose}-inference rule
   applies {\bf do} \\
   \( ({\cal E};{\cal R}) := Compose(({\cal E};{\cal R})). \)
 \item[{(10) [Deduce.]}] Compute all (generalized) $\cal R$-normalized
      AC-critical pairs ${\cal P}$ of     $l \rightarrow r$ and rules
    in $\cal R$. 
 \item[{(11) [Subsumption test.]}] Delete pairs from $\cal P$ that are
      subsumed by other pairs in $\cal P$.
 \item[{(12) [Update $\cal E$.]}]
        \( {\cal E} := {\cal E} \cup {\cal P} \)\footnote{Here we
                                             identify pairs and equations.};
     {\bf continue} with step 2. \hfill $\Box$
\end{description}
\end{quote}
\end{minipage}
}
\end{center}
\caption{Algorithm {\em AC-COMPLETE}} \label{fi:accmpl}
\end{figure}

The equation queues are organized as in DTKB:
\index{queue}
\begin{enumerate}
 \item a queue Q1 for initial equations and collapsed rules,
 \item a standard critical pair queue Q2 and
 \item a queue Q3 where critical pairs are held which
    were manually but back (option ``B'' in QCA).
\end{enumerate}
Q1 has priority over Q2 which has priority over Q3.

The following modifications to the procedure presented in
AC-COMPLETE  may be selected by the user.
\begin{itemize}
 \item The user may select a subset of ${\cal R}_0 \subseteq {\cal E}$
       of which he know that it is complete.
       Then $\cal R$ is initialized by ${\cal R}_o$ instead of the empty
       set.
 \item Since the reduction strategy depends on the order of the rules,
       the user may influence this order or use the automatic rule
       ordering.
 \item The user may determine whether he wants to orient rules manually
        or using the term ordering.
       In the second case he can chose whether he wants to orient 
       non-orientable rules manual (or take advantage of the
       display, trace and exit options at that point) or
       if non-orientable rules shall be added to the end of the
       queue. The last option risks non-termination.
 \item The user may determine whether the input equations shall be
       selected first in step~5 (insert $\cal E$ into Q1) or not
       (insert $\cal E$ into Q2).
 \item The user may determine whether only the selected equation or
        all equations in $\cal E$ shall be  kept normalized in step 2.
 \item The user may determine whether collapsed rules are to be inserted
       into Q1 (priority) or into Q2.
 \item The user may choose to apply the subconnectedness criterion in
       step~8 or not.
 \item The user may choose to compute standard critical pairs or
\index{critical pair!generalized}
       generalized critical pairs.
 \item The user may choose to delete subsumed critical pairs in step~11
       or not.
\end{itemize}
The {\tt accsc}-modules provides a data structure called
{\em completion strategy control record} containing the
switches to control the configuration of the completion procedure.
It also contains the queries to ask the user for the different options.

%So far the completion procedure works only in step mode (STEPM $= 1$).

In the dialogue of the orientation step (QCA) the user has the possibility to
call the fertilization procedure. 
The {\em fertilization} procedure CPFERT (see also \cite{Buendgen:91b}) is
\index{fertilization}
implemented in the {\tt acf}-module.
It allows for guided computation of critical pairs between a single rule and 
the critical pair selected from the equation queues.
Before the fertilization process starts the selected pair must be oriented and
possibly extended.
The user can than select one of the new pairs to continue the 
completion procedure starting with the orientation step.
The original pair is pushed back onto Q2.
As shown in \cite{Buendgen:91b} the fertilization of critical pairs is very 
helpful if the left-hand side of the oriented original critical pair is
reducible by the rule associated with the pair computed by the
fertilization procedure.
Since the completion is resumed at the orientation step, the fertilization 
process may be iterated several times.

The AC-completion calls (via QCTS0 from {\tt tcint}) a QCT procedure private to
the {\it ac}-subsystem.

\subsection{The {\tt acx}-Module}

The module {\tt acx} provides the function TXCMPT which computes a
mixed term:
Let $f$ be a function symbol that has an algorithm key stored under indicator
DXIA, Let $A$ be the \ALDES\ function denoted by that algorithm key, 
let \( t_1, \ldots, t_n \) be external terms importing the external objects 
\( o_1, \ldots, o_n \) and $o$ be the result of calling $A(o_1, \ldots,o_n)$.
Then $s[c(o)]_p$ is the evaluation of $s[f(t_1,\ldots,t_n)]_p$ at position
$p$ if $c$ is an appropriate coercion operator with the same result type
\index{operator!coecrion}
(ORTYPE) as $f$. 
The appropriate coercion operator is stored under indicator DXCO
either as property of $f$ or as property of the symbol denoting
the result sort of $f$.
Applying the evaluation procedure as often as possible to a term
\index{compute}
is called {\em computing} a term.

\subsection{The AC-Rewrite Laboratory}

The main algorithm AC is a rewrite laboratory for term rewriting systems
\index{main algorithm}
modulo AC-theories.
The program first reads a data type from input stream 4, 
then adds possible extension rules to its axioms and displays the data type.
According to a menu selection the user may then AC-unify, AC-match, 
\index{unification!AC{-}}
\index{match!AC{-}}
test two terms for equality modulo AC, order an axiom or order the axioms
of the data type according to the installed and selected term ordering.
He may also normalize a term, toggle traces, set or reset timers and counters,
display the current data type,
compute selected critical pairs, prove equality modulo the AC-term rewriting
\index{term!AC-equality}
system of the data type and complete the set of axioms using the AC-completion
algorithm.
