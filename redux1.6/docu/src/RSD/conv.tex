\section{Programming Style and - Conventions}

This  section contains some general remarks on the programming style and 
\index{programming style}
\index{programming convention}
conventions used in the \redux-system.
The recommendations and conventions presented in this section must be taken
seriously, especially by programmers who expect their code to be upward
compatible with future versions of the \redux-system.

\subsection{Modularization}

As already mentioned in the introduction, the \redux-system is subdivided
into several subsystems.
\index{subsystem}
Each subsystem is again subdivided into {\em modules}.
A \redux\ module can either be a collection of algorithms which have related
tasks (E.\,g.\ the module {\tt tpcy} contains copying algorithms.),
the specification of an abstract data type (E.\,g.\ {\tt tpsb} specifies the
abstract data type substitution.) or the implementation of a complicated
algorithm which is broken into subtasks (E.\,g.\ {\tt icip}).

Each module is physically a file. It begins with a documentation header.
\index{documentation header}
\begin{example} \null \\
 \begin{tabbing}
   module specification\=: \= \kill
  [Project: ReDuX Term Primitives \\
   module name        \>: \> TPDD -- -- Data type Definition \\
   module specification\>: \> basic data type definitions and manipulation \\
                        \> \> algorithms for data types allowing \\
                        \> \> function symbols with  AC-status. \\
                        \> \> upgrade of icd1.ald as of Oct. 13, 1989 \\
                        \> \> BLKW has been changed to match the new \\
                        \> \> specification of TAB and to circumvent \\
                        \> \> endless loop caused by TAB. \\
   programmer         \>: \> Wolfgang Kuechlin / Reinhard Buendgen \\
   language           \>: \> \ALDES \\
   last change        \>: \> 7.3.90 12.11.89 10.08.89 \\
                        \> \> \$Date: 91/09/09 09:10:19 \$ \\
                        \> \> by \$Author: buendgen \$ \\
   version            \>:\> 1 without labels and AC-status \\
                        \> \> TEPRS added \\
                        \> \> \$Revision: 1.3 \$ \\
   bugs               \>:\> none known \\
   modules needed     \>: \> SAC2: low/bs, low/sy, low/lp \\

]
\end{tabbing}
\hspace*{\fill} $\Box$
\end{example}
There may be comments denoting header files (see \RIX)
to be concatenated in front of the source file.
The header files contain macros declarations to manipulate and access data 
types and declarations of global (safe) variables.
\index{header file}
The macro definition techniques used in \redux\ are explained in
Section~\ref{ss:mac}.
Then some more  macro definitions specific to that file may be declared.
\begin{example}  \null \\
 \begin{tabbing}
const \= \kill
{[ \#include tp.h ]} \\
{[ \#include ic.h ]} \\
{[ \#include globs.h ]} \\
\end{tabbing}
\hspace*{\fill} $\Box$
\end{example}
Finally the algorithms which may be grouped into different {\em sections}
\index{section}
follow.
Modules containing abstract data types or complex procedures contain
\index{abstract data type}
sections called ``Algorithms for export'' and ``Algorithms for internal use''.
If these sections exist, only the procedures and functions in the section 
``Algorithms for export'' may be called from outside the module.

\subsection{Global Variables} \label{ss:gv}

The use of global variables in the ReDuX-system should be avoided.
\index{variable!global}
However there are a few occasions where global variables are used.
\begin{itemize}
 \item pointers to symbols in the global symbol table,
\index{symbol}
 \item global time accumulators,
\index{time accumulator}
 \item global counters and
\index{global counter}
 \item flags for toggling traces
 \item global hash arrays for the random term generator
\index{trace flag}
\end{itemize}
may be stored in global variables.
In any case, global variables should be used with utmost care and whenever
possible should their use be restricted to one module.
\index{declaration!external}
\index{declaration!safe}
Global variables must be declared externally\footnote{External declarations
precede the algorithms.} safe\footnote{I.\,e.\ they should not have any
impact on the garbage collection.}.
See the \RIX\ for a list of 
all global variables used in ReDuX.

\paragraph{Pointers to Symbols}
Symbols are mostly used for grammar symbols and indicators of global
\index{symbol}
properties.
Therefore most of the global symbols are declared and used in the term primitive
data type definition module {\tt tpbgn}.
Pointers to symbols are supposed to keep a constant value once they are 
initialized.

The variable names for global symbols should consist of three or four upper 
case letters, the first being the letter `D' and the last preferably being 
the letter `S'.
There are some exceptions to this rule, mainly for symbols needed by the old
parser in {\tt tpprs} and for symbols which are only referred to by a single 
module.
Global variables related with external objects
\index{object!external}
should start with `DX'.

\paragraph{Time Accumulators}
\index{time accumulator}
Time accumulators accumulate the time spent during certain operations.
Their names typically consist of  three letters, the last two being a `TI'.
\begin{example}
 A typical update of a time accumulator looks as follows:
 
 \begin{tabbing}
  global \= \kill
  \> \vdots \\
  global \> UTI. safe UTI. [Unification time] \\
       \> \vdots \\
       \> UTI:=UTI-CLOCK(); I:=TNRMCI(T1,T2); UTI:=UTI+CLOCK(); \\
       \> \vdots
 \end{tabbing}
 
 The time spent in TNRMCI is added to UTI. \hfill $\Box$
\end{example}

\paragraph{Counters}
\index{global counter}
Global counters are used to count how often certain operations are performed.
Their names typically consists of two or three letters, the last being a `C'.

\paragraph{Trace Flags}
Trace flags are integer variables which control whether certain features
of a program should be traced.
The value 0 means that the trace is switched off.
The names of the trace flags consist of five upper case letters or digits,
the last three of which must be `TRC'.

\paragraph{Exceptions}
The ReDuX system is not a homogeneous piece of software.
Several programmers have been involved in its development and up to now there
have not been any written guidelines on how to use global variables.
That is why there are a few exceptions from the rules described so far.
These exceptions are supposed to be the only exceptions in future extensions of 
the ReDuX system and they will be replaced step by step in the following
versions.
Here we list all exceptional global variables known to us:
\index{variable!global}
\begin{itemize}
 \item The symbols KIND, IDENT, XIDNT, INT, GRSYM, SYM are
  used in scanner and parser routines.
 \item The symbol FLAT is used in scanner, parser and output routines.
 \item There are trace flags TRC1 (theorem derivation trace), TRC2 (prove
  theorem trace) and TRC3 (formula reduction trace).
 \item the global variable AUTOM is used in completion procedures.
 \item Property indicators used in term orderings:
       KBOI, KBOW, POASK, POGEQ, POLEQ, POSTS. 
       The use of these global variables is restricted to very few files.
 \item If a global variable is only referenced in the file containing the
  main algorithm and its direct menu-subalgorithms
  its name is only restricted to not clash with that of
  other global variables in the system (e.\,g.\ STPWT in {\tt ac}).
\end{itemize}

\subsection{Macro Techniques} \label{ss:mac}

The \ALDES\ language provides a macro expansion facility which is introduced
\index{macro}
with the key word `const'.
\index{declaration!const{-}}
Const-declarations can be used to define constants and functions by macro
definitions.

\paragraph{Constants}
For all ReDuX algorithms the constants FALSE = 0 and TRUE =1 should always be
declared.
\index{ALDES}
In newer \ALDES\ implementations these constants are already built-in.

There is no constant for the empty list. Even though comments and the
documentation often use the word `nil', `nil' is neither a built-in constant
nor is it a declared constant in any ReDuX module.
The only way to refer to the empty list is by writing `()'.
In ReDuX it is not allowed to refer to the \ALDES\ variable BETA ($\beta$)
\index{BETA ($\beta$)}
in order to refer to the empty list.

If constants are used to describe parts of an abstract data type
(like ASSIDTY in {\tt tpsb}) these parts should be referenced to only via these
constants.
I.\,e., in all modules referring to that part of a data type this constant must
be redeclared.

\paragraph{Functions}
The const-declarations can also be used to define function macros.
\begin{example} 
   const TCONT(T) = FIRST(T). \hfill $\Box$
\end{example}
Then every occurrence of the left-hand side of a constant declaration
(here TCONT(\ldots)) is replaced by the expression in the right-hand side of
the declaration where the function parameters are replaced by the actual 
parameters.
\begin{example} 
 TCONT(INV(L)) becomes FIRST(INV(L)).  \hfill $\Box$
\end{example}
A const-declaration is valid as soon as it appears, thus a const-declaration
can use a previous const-declaration.
\begin{example} \label{ex:sel} \null \\
\begin{tabbing}
 const         \= RED(L)    = SPACE[L-BETA], \\
               \> RED2(L)   = RED(RED(L)), \\
               \> RED3(L)   = RED(RED2(L)), \\
               \> RED4(L)   = RED(RED3(L)), \\
               \> FIRST(L)  = SPACE[L-BETA1], \\
               \> SECOND(L) = FIRST(RED(L)), \\
               \> THIRD(L)  = FIRST(RED2(L)), \\
               \> FOURTH(L) = FIRST(RED3(L)), \\
               \> FIFTH(L)  = FIRST(RED4(L)). 
\end{tabbing}
\hspace*{\fill} $\Box$
\end{example}
In the ReDuX-system function const-declarations are used for the following 
purposes:
\begin{enumerate}
 \item to rename functions in order to give them a more intuitive name,
 \item it speed up the computation, avoiding function call overhead,
 \item to implement (virtual) records and
 \item to provide interfaces for (abstract) data types.
\end{enumerate}
A function $i$th(L) which returns the \ith{i} element of
the list L is called an {\em \ith{i} element selector}.
\index{element selector}
At the beginning of most ReDuX modules the \ith{i} element selectors are 
defined by macro techniques such that each selector is mapped to access to
\index{SPACE}
\index{ALDES}
a field in the array SPACE\footnote{The array SPACE is the dynamic storage 
(heap) of \ALDES.}.
Thereby the standard \ALDES\ and SAC-2/{\it lp} element selectors 
(FIRST, SECOND,
\index{SAC-2}
THIRD and FOURTH) are overwritten.
\index{reductum}
\index{reductum selector}
In the same way the \ith{i} reductum\footnote{Reductum is the \ALDES/SAC-2 
expression for the rest or `cdr' of a list.} selectors are defined by mapping 
them to accesses to SPACE  (see Example~\ref{ex:sel}).
\index{SPACE}

Element- and reductum selectors as defined above do not contain actual 
function calls.
This has the following implications.
Firstly, the overhead of function calls is avoided and secondly 
function macros may occur at the left-hand side of an assignment resulting 
in a convenient way to update parts of a list\footnote{This way of using 
macros is due to Wolfgang K\"{u}chlin.}.
\begin{example}
  SECOND(L) maps to SPACE[SPACE[L-BETA]-BETA1]. Therefore \\
  SECOND(L) := L' maps to SPACE[SPACE[L-BETA]-BETA1] := L' \\
  which is a correct \ALDES\ statement. \hfill $\Box$
\end{example}

In the ReDuX system, data structures which contain a fixed set of 
substructures are stored in lists of fixed length.
Such structures would be implemented as records in languages like Pascal.
To access the components of such a data structure in ReDuX there is a 
const-declaration for each component which renames an \ith{i} element or
reductum selector.
\begin{example} const TNAME(T) = THIRD(T). \hfill $\Box$
\end{example}
Together with the previously explained macro definitions of element and
reductum selectors such constant declarations are equivalent to
record selectors.

It is understood that the data structures defined using macros should always
be accessed through these macros.

Most macros used in \redux\ algorithms are collected in the header files
{\tt tp.h} and {\tt ic.h} (see \RIX.)
In each \redux-source file  comments (e.\,g.\  $[$ \# include tp.h $]$)
state which header files must be concatenated in front of the source file.

\subsection{Naming Conventions}

Here we give an outline for naming conventions used for ReDuX variable and
algorithm identifiers.
It must be noted that these conventions are not strictly followed
throughout the ReDuX system.
However they provide some guidelines for systematically naming program objects
and therefore give raise to mnemonics which after some practice can 
easily be expanded to their full meaning.
This is especially important because \ALDES\ identifiers may only consist of
\index{ALDES}
\index{identifier!ALDES}
at most six upper case letters and digits (with a few exceptions to be
explained later).
Therefore programmers are advised to follow these conventions as strictly as
possible.

The most important convention which {\em may not} be violated is that 
no identifier of an ReDuX algorithm (or a global variable) may be equal to a 
\index{SAC-2}
SAC-2 algorithm (or global variable) name.

\paragraph{Variable Identifiers}
The following restrictions are due to the \ALDES\ language:
\begin{itemize}
 \item a variable identifier may consist of at most six upper case letters
       and digits or,
 \item it is an ornamented one-letter identifier and
\index{identifier!ornamented}
 \item names of global variables have a maximum length of five.
\end{itemize} 
An {\em ornamented one-letter variable name} consists of one letter
(lower or upper case) possibly followed by ornaments and indices, as described 
by the regular expression
\[ \{a,\ldots,z,A,\ldots,Z\}\{\_, \; \hat{}, \; \tilde{}\,\}^*\{*,'\}^*\{0,\ldots,9\}^*. 
\]
Ornamented identifiers may consist of at most six characters, too.
Table~\ref{ta:oi} gives an overview of the ornamented variants of the
identifier T, in publication form, \ALDES\ source code and the source
code of an older version of \ALDES\ --- sometimes called {\em OLDES} 
\index{OLDES}
(\cite{Loos:76}).
\begin{table} 
\begin{center}
\begin{tabular}{|c|c|c|l|}
 \hline
 publication & \ALDES & transliteration to & mnemonic\\
 \ALDES       & source code  & OLDES source code & \\
 \hline \hline
 $T$               & T                 & T     & \\
 $t$               & t                 & TL    & T lower case \\
 $\bar{T}$         & T\_               & TB    & T bar \\
 $\hat{T}$         & T\,$\hat{}$       & TH    & T hat \\
 $\tilde{T}$       & T\,$\tilde{}$     & TT    & T tilde \\
 $T^*$             & T*                & TS    & T star \\
 $T'$              & T'                & TP    & T prime \\
 $T_1$             & T1                & T1    & \\
 $t''$             & t'\,'             & TLPP  & \\
 $T'_3$            & T'3               & TP3   & \\
 $\tilde{t}^*_5$   & t$\,\tilde{}\,$*5 & TLTS5 & \\   
 \hline
\end{tabular}
\caption{Ornamented variants of T} \label{ta:oi}
\end{center}
\end{table}
Note that many translators (like e.\,g., the \ALDES\ to Fortran translator)
transliterate ornamented identifiers as it was done in OLDES.
This may lead to name clashes in the intermediate language!

In \redux\ mostly ornamented identifiers are used for local variables.
Input parameters of algorithms are normally denoted by single letters
(e.\,g., t) and output parameters (or function results) are often
ornamented with a star (e.\,g., t*).
Working copies of a variable are normally primed versions of the 
original variable (e.\,g., t').

If possible a variable name is a mathematical standard identifier of the
\index{identifier!mathematical standard}
object it presents as it is used in most publications (e.\,g., s, t for terms
or v, x, y for variables).
It is common practice in \redux\ to use the OLDES transliteration of ornamented
variables as mnemonics (e.\,g., L$\,\hat{}\,$* = LHS = left-hand side,
D$\,\tilde{}$ = DT = data type).
Unfortunately this may lead to some confusion.
Whereas in newer code objects are normally named by lower case letters
(e.\,g., t) and lists of these objects by corresponding upper case letters
(e.\,g., T), this was done just in the opposite way by OLDES programmers
\index{OLDES}
(TL was interpreted as `T-list'.).

Note that Greek letters (ALPHA, BETA, \ldots) should not be used as
variable names because many such identifiers are used as built-in
global \ALDES\ variables.
\index{ALDES}
For naming conventions of global variables see Section~\ref{ss:gv}.

\paragraph{Warning!} 
Because of clashes with the global variables DLS, DPS, DSS and DTS
the variable names d*, D`*, D**, D$\,\tilde{}\,$* may not be used for local
variables.
To avoid future clashes, the variable names d', d\,$\hat{}$, d\,$\tilde{}$,
d\_ and Dxy where x, y $\in \{$ ', *, $\hat{}$, $\tilde{}$, \_ $\}$
should not be used either.

\paragraph{Algorithm Names}
\index{algorithm name}
Algorithm names are identifiers consisting of at least two and
at most six upper case letters and digits.
The first letters of an algorithm name often describe the types of its main
parameters (see Table~\ref{ta:pt}).
The letters `L', `P' and `S' often mean list, pair and set respectively.
\begin{table}
\begin{center}
\begin{tabular}{|l|l||l|l|}
 \hline
 abbrev. & type(s)                    & abbrev. & type(s) \\
 \hline \hline
  A     & atom or axiom               
                                      &
                                        NTS(T) & node of top set tree  \\
  AX    & axiom                       &                                   
                                        O(P)  & operator / function symbol \\
  C     & constant                    &                                   
                                        P     & polynomial \\             
  CP    & critical pair               &                                   
                                        P(D)  & production rule \\        
  CSC   & completion strategy control &                                   
                                        PTS   & pruned top set tree \\    
  DC    & dictionary                  &                                   
                                        R(R)  & rewrite rule \\           
  DE    & Diophantine equation        &                                   
                                        S     & symbol \\                 
  D(I)  & variable dictionary         &                                   
                                        SB    & substitution \\           
  DT    & data type                   &                                   
                                        STL   & subterm list \\           
  EVC   & evaluation candidate        &                                   
                                        T(N)     & (non-atomic) term \\   
  GR    & grammar                     &                                   
                                        TST   & top set tree \\           
  I     & integer                     &                                   
                                        TX    & mixed term \\             
  INT   & $\gamma$-integer            &                                   
                                        V     & variable \\               
  L     & list                        &                  
                                        X     & external \\               
  NPTS  & node of pruned top set tree &
                                        Y     & type \\                   
 \hline      
\end{tabular} 
\end{center}  
\caption{Abbreviations of parameter types} \label{ta:pt}
\end{table}
In the {\it ac}-subsystem the suffixes `E' and `AC' often mean `modulo an
equational theory' and `modulo AC'.

Names of interactive procedures often begin with `M' (for message, pure output)
or `Q' (for query, dialogue).

Algorithm names often include words where the vowels have been omitted
(e.\,g., WRT for `write') and some times digits are used instead of
words with the same pronunciation (e.\,g., `2' instead of `to' or
`W8' instead of `weight').

\subsection{Interactive Input}
All \redux\ programs use an asterisk ({\tt *}) as prompt for user input.
Input options are given in  brackets ({\tt [}, {\tt ]}) and may be separated by
slashes.
\begin{example}
\begin{verbatim}
 ...
 Enter your choice [B/L/N/E] *
\end{verbatim}
\end{example}

\subsection{The Main Programs}

The main programs normally start with reading a description of an
\index{main program}
algebraic specification from input stream~4\footnote{The stream number may be 
subject to change on systems where this stream is reserved for special
purposes.}.
The algebraic specification is called a {\em data type} in the \redux\
\index{data type}
system. 
A \redux\ data type consists of a description of a many sorted signature,
\index{signature!many sorted}
together with a list of sorted variables and a set of equations
\index{variable!ReDuX}
\index{equation}
\index{axiom}
(called {\em axioms} in the \redux\ system).
Typically this data type is displayed after it has been parsed.
The remaining I/O is done interactively.
The syntax of these data types and the use of the \redux\ demo programs
\index{demo program}
is described in \cite{RUD1.4}.

Modules containing main programs can also contain some procedures
specific to that very program.
For such procedures we do not require that the above mentioned
naming conventions are followed strictly, provided these procedures are
{\em not} included in a library.
Generally main programs are not included in \redux\ libraries.

With the current \ALDES-to-C compiler main programs must contain
local declarations of all global variables used in any called algorithm.
Global SAC-2 system variables need not be declared.
