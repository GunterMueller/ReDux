\section{The TO-Subsystem} \label{se:to}

The {\it to}-subsystem contains term orderings according to which critical
pairs may be oriented such that the resulting term rewriting system is
terminating.
So far, three different term orderings have been implemented:
\begin{enumerate}
 \item the Knuth-Bendix ordering (\cite{KnuthBendix:67}),
\index{ordering!Knuth-Bendix}
 \item the path ordering (with lexicographical l-r and r-l status) described in
\index{ordering!path}
       \cite{KapurNarendranSivakumar:85} and
 \item  the polynomial interpretation ordering described in
       \cite{BenCherifaLescanne:87}.
 \item a polynomial interpretation ordering based on 
\index{ordering!polynomial}
       quantifier elimination (\cite{Weispfenning:88},
       \cite{Loos:90}) has been tested too, but it is not available
       with all \redux-installations.

\end{enumerate}
The Knuth-Bendix ordering should not be used with terms containing AC-operators.
\index{operators!AC{-}}
To use the path ordering with terms containing AC-operators, the input
terms must be guaranteed to be flat.
\index{term!flat}
The {\it to}-subsystem contains the main program {\tt to} as a test environment.
It is described in \RUD.

\index{critical pair}
There is a general driver routine TPPO which orients a critical pair\footnote{A
{\em critical pair} is a list whose second and third elements contain the 
terms of the pair} 
as described in the {\it tc}-subsystem ({\tt tccp}) according to
the control information encoded in a string O.
Each character in O denotes an ordering and the input terms are to be ordered 
w.\,r.\,t.\ the lexicographic combination the orderings listed in the string
O.
TPPO returns a character which corresponds to the users answers for
manual critical pair orientation:
\begin{itemize}
 \item `S' if the left term is greater than the right term (straight ordering),
 \item `R' if the right term is greater than the left term (reverse ordering)
 \item`$=$' if both terms are equivalent or
 \item 'H' if the two terms are incomparable (help requested).
\end{itemize}

Before using a term ordering it must be initialized. 
Among others the following
operations must be performed: 
\begin{description}
 \item[Knuth-Bendix ordering:] the operator weights and - indices must be
  initialized,
 \item[path ordering:] the l-r or r-l status of binary operators must be known,
 \item[polynomial ordering:] the polynomial interpretations of constants
  and operators must be set.
\end{description}
In addition certain symbols must be entered in the global symbol table.
\index{symbol}
\index{symbol table!global}

\subsection{The {\tt toint}-Module}

The module {\tt toint} contains interactive routines to initialize all
\index{initialization}
term orderings needed (TORDIN) and to construct a lexicographical 
\index{lexicographical combination}
combination of the initialized orderings (TORDSE).
There is also a procedure to output the result of an ordering (ORDWRT).
In the current implementation it is not possible to initialize and/or
select more than one Knuth-Bendix- or more than one path-ordering
simultaneously.
However up to nine different polynomial orderings may be installed.

\subsection{The {\tt torm}-module}

The module {\tt torm} contains algorithms to deinstall term orderings.
TORDGT determines which orderings have been installed.
TORDRM removes deinstalls term orderings interactively.

\subsection{The Knuth-Bendix Ordering}
The implementation of the Knuth-Bendix ordering in the ReDuX-system is 
\index{ordering!Knuth-Bendix}
described in \cite{Kuechlin:82a}.
The original implementation has been modified such that only variables with 
the same representation are considered equal.
The Knuth-Bendix ordering associates to each constant, operator, and
variable both an index and a weight.
Two terms are compared according to their cumulative  weights and the
indices of their top operators.

In module {\tt tokb} the procedure BGNKBO initializes the Knuth-Bendix ordering 
by storing indices and weights of 
\index{initialization}
constants, operators and variables under the indicators KBOI (index) and
KBOW (weight) in the property list of the constants, operators and variables
of the data type.
The symbols KBOI and KBOW are entered in the global symbol table.
The indices and weights of the constants and operators are entered
interactively.

There is also a procedure (ENDKBO) to remove an initialized 
Knuth-Bendix-ordering
of the property lists of all variables, constants and operators of a data type.

CPKB is a procedure which orients a critical pair w.\,r.\,t.\ the 
Knuth-Bendix ordering.

\subsection{The Path Ordering}

The original implementation of the path ordering 
\cite{KapurNarendranSivakumar:85} \index{ordering!path}
available in the ReDuX system is described in \cite{Schwaerzler:86}.
The original implementation has been modified to handle operators with
l-r and r-l status and to ensure that only variables with the same 
representation are considered equal.\footnote{We implemented the l-r-
and r-l stati as defined in \cite{KapurNarendranSivakumar:85}.
However this difinition seems not to be complete.}
In addition the global list variable SPECL has been eliminated and the new 
implementation uses recursion.

%In order to use the path ordering the variable SPECL must be declared 
%\index{variable, global}
%externally global in the main program.

The path ordering algorithms are subdivided into three modules:
{\tt tosqol}, {\tt toknsl} and {\tt toknsli}.

%{\tt tosqol} implements a quasi ordering on symbols 
%(operator and constant names).
%The information needed for this ordering is stored in the global list
%SPECL.
%Each element of SPECL has four components: 
%\begin{description}
 %\item[SYMBL:] operator symbol (TNAME of a constant or an operator),
 %\item[STATUS:] operator status (none, l-r, r-l) of binary operators,
 %\item[ASK:] a list of  constant- and operator names whose relation to the
  %symbol in SYMBL is not known and
 %\item[LEQ:]a list of  constant- and operator names which are less than or
  %equal to the symbol in SYMBL.
%\end{description}
%BGNPO initializes SPECL and SYMQUO($f$,$g$) determines whether the operator
%symbol $f$ is greater than or equal to the operator symbol $g$.

{\tt tosqol} implements a quasi ordering $\succ$ over constants and operators.
BGNPO  initializes this quasi-ordering.
It first creates (global) symbols which are used as indicators for 
term property lists describing $\succ$:
\begin{description}
 \item[POASK:] if \( g \in \mbox{TGET}(f,\mbox{POASK}) \) 
               then it is not known whether \( f \succeq g \).
 \item[POGEQ:] if  \( g \in \mbox{TGET}(f,\mbox{POGEQ}) \) 
                then \( g \succeq f \).
 \item[POLEQ:] if  \( g \in \mbox{TGET}(f,\mbox{POLEQ}) \) 
                then \( f \succeq g \).
 \item[POSTS] is used to record the lexicographical status of each
              constant or operator.
\end{description}
To begin with all relations among constants and operators are initialized to
be unknown  and the lexicographical status is initialized interactively.
OPQUO and OPQUOI implements the quasi ordering $\succ$. Missing information
is added interactively.
TSSTAT extracts the lexicographical (none, l-r, r-l) status of a constant
or operator.
ENDPO, finally, removes a previously initialized path-ordering by
removing the indicators POASK, POGEQ, POLEQ and POSTS from the property lists
of all variables, constants and operators of a given data type.

The module {\tt toknsl} implements the path ordering with l-r and r-l status
and provides the procedure CPPO to orient a critical pair.
The data structure {\em path} is represented by a list of terms 
\index{path}
(in reverse order).
%The procedure PO is an example of how to program indirect recursion in
%ALDES if the linker does not allow for mutually dependent procedures.

The module {\tt toknsli} contains the interactive procedure OASKST to query
for the status of a constant or an operator.

\subsection{The Polynomial Interpretation Ordering}

The implementation of the polynomial interpretation ordering 
\index{ordering!polynomial}
available in the ReDuX system is described in \cite{BenCherifaLescanne:87}.

To use the polynomial ordering, the SAC-2 and SACR systems are needed.
\index{SAC-2}
Therefore \redux\ must be initialized such that the SAC-2 polynomial system
is initialized too 
(I.\,e.\ BGNRWD must be called with an  argument greater than 3.).
So far the system computes a formula in prenex normal form 
for two terms to be compared.
This formula can then be decided using a positivity-test.
\index{positivity-test}
(\cite{BenCherifaLescanne:87}).
The algorithms for the polynomial ordering are subdivided into five modules.

The module {\tt topiti} contains interactive algorithms to compute the
(ranked) integral polynomial interpretations of a term or a term pair.
A ranked integral polynomial is a pair $(r,l)$ where $r$ is a $\beta$-integer
denoting the rank (number of indeterminates) of the integral polynomial $p$.
Integral polynomials are represented in recursive form.
PRIP2I and RIPTIV compute the polynomial interpretation of a term and a term
pair respectively and they ask the user for the missing information only
when the information is needed.

The module {\tt topit} contains auxiliary algorithms to build
polynomial interpretation from terms.

The module {\tt tospi} contains algorithms to initialize the
polynomial interpretation ordering and display a polynomial
interpretation. 
The procedure BGNPYO interactively initializes the
polynomial interpretation. All constants and operators are initialized
with empty polynomial interpretations. Missing information is added
interactively. 
For each constant or operator a list of polynomial interpretations
can be stored in its property list under indicator DIS.
ENDPYO removes a previously initialized polynomial
ordering from all constants and operators of a given data type.

The module {\tt tobcl} contains algorithms to order two terms.
The procedure CPPI compares the terms in a critical pair w.\,r.\,t.\ a 
polynomial interpretation ordering. The algorithm IPPT is an
implementation of the positivity-test described in 
\cite{BenCherifaLescanne:87}.
There are two change-coefficient-algorithms CCCLS1, CCCLS2 and one
choose-monomial-algorithm RPCMSD.

The module {\tt SACR} provides some basic algorithms for  polynomials
(IPCONV, PCOMP, PDEC, PFCV, PNT, RPEXP, RPFCV, RPFFML)
and formulas (FCOMP, FDEC, FVARS, INFVIV, ROMIN, RPFFML, TREAD1, VRD).

% right now not supported
%\subsection{The Linear Polynomial Interpretation Ordering}
%
%The implementation of the linear polynomial interpretation ordering
%\index{ordering, polynomial}
%available in the ReDuX system is described in \cite{Joswig:90}.
%
%To use the polynomial ordering, the SAC-2 and SACE systems are needed.
%\index{SAC-2}
%So far the system computes a linear formula in prenex normal form for two terms
%to be compared.
%This formula can then be decided using a linear quantifier elimination
%\index{quantifier elimination}
%(\cite{Weispfenning:88}, \cite{Weber:89}).
%Most ReDuX distributions will not include the needed libraries.
%In this case dummy procedures ({\tt qewdummy}) are provided for the missing
%interface procedures and the polynomial ordering cannot be used.
%The sources of this module are provided anyway to show how to implement
%a map from ReDuX-terms to SAC-2-polynomials.
%The algorithms for the polynomial ordering are subdivided into two modules.
%
%The module {\tt PI} contains algorithms to compute the (ranked) integral
%polynomial interpretations of a term or a term pair.
%A ranked integral polynomial is a pair $(r,l)$ where $r$ is a $\beta$-integer
%denoting the rank (number of indeterminates) of the integral polynomial $p$.
%Integral polynomials are represented in recursive form.
%
%The procedure ARIPDI interactively initializes the polynomial interpretation
%of all constants and variables, whereas ENDPYO removes a previously initialized
%polynomial ordering from all variables, constants and operators of a given
%data type.
%However no support is provided to check whether the polynomial interpretations
%yield a simplification ordering (\cite{Dershowitz:79}).
%PRIP2 and RIPTV compute the polynomial interpretation of a term and a term
%pair respectively.
%PRIP2I and RIPTIV do so without requiring an initializing call of ARIPDI,
%but they ask the user for the missing information by need.
%The procedure CPPI compares the terms in a critical pair w.\,r.\,t.\ a
%polynomial interpretation ordering based on quantifier elimination.
%
%The module {\tt YW} provides the interface to the quantifier elimination.
%QE calls the quantifier elimination and QEFORM computes the input formulas
%derived from the polynomial interpretations of two terms.
%In addition {\tt YW} contains some basic algorithms for constructing and
%destructing polynomials (IPMON, PCOMP, PCONST, PDEC, PINCR)
%and formulas (FDEC, RPFFML, INFVIV).
