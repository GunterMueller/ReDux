\section{The EV-Subsystem} \label{se:ev}

The \redux\ subsystem {\it ev} contains procedures needed for
normalizations supported by {\em partial evaluation domains}
\cite{BuendgenLauterbach:96}.
\index{evaluation domain!partial}
Let $\cal R$ be a complete rewrite specification with a complete 
subspecification $\cal R'$.
Then  a model $\cal D$ of $\cal R'$  is a {\em partial evaluation domain}
for $\cal R$ 
if $\cal D$ is a free algebra in the models of $\cal R'$ over a countable 
set of generators.
In addition we need an efficiently computable mapping from an implementation
of  $\cal D$ to
terms in $\cal R'$-normal form.
This mapping is fixed by a bijection from the generators of $\cal D$ to 
variables and terms outside of the signature of $\cal R'$.
The $\cal R'$-normal form of any term $t$ can be computed by retranslating
the $\cal D$-interpretation of $t$ back to a term.
Given that ${\cal R}'$ is a proper subset of $\cal R$, 
we must alternate evaluation based normalizations \wrt\ ${\cal R}'$ and
reductions \wrt\ ${\cal R}\setminus {\cal R}'$ until a fixed point is
reached in order to compute $\cal R$ normal forms.

A partial evaluation domain is specified by an external type
\index{type!external}
for which the following properties must be specified:
under indicator XFG the name of a binary function that computes
the $n$-th generator of the evaluation domain 
($n$ is provided as first argument and an evaluation dictionary as second
argument),
and under indicator XTERM the name of a ternary function translating an 
external object back into a term.
For tracing purposes the name of an output routine for external objects may
be stored under indicator XWRITE.
Each evaluable operator or constant $f$ 
(i.\,e., each symbol $f$ occuring in $\cal R'$)
must be assigned a corresonding interpretation 
function $f_{\cal D}$ over the partial evaluation domain $\cal D$.
Therefore the name of a wrapper function calling $f_{\cal D}$ is stored as
a property of $f$ under an indicator named by the external type denoting
$\cal D$.
The wrapper function has one more argument than $f_{\cal D}$ to pass
an evaluation dictionary.
Note that the function specified by indicator XTERM determines
the sequence in which certain constants and operators must be declared.

\subsection{The {\tt teini}-Module}

The procedure DTEVIN initializes the data type for evaluation.
For each external type it computes a list of constants $C$ and a list of 
operators $F$ that are evaluable. 
The pair $(C,F)$ is stored as a property of the external type under 
indicator DXCN.
The order of $C$ and $F$ is crucial for the retranslation of external
objects to terms.

In addition for each type a variable of that type is stored as a
property of that type under indicator DVS.

\subsection{The {\tt tedi}-Module}

In order to partially evaluate  terms that contain subterms for which no 
interpretation  is known these non-interpretable subterms must be mapped to
a generator (indeterminate) of the evaluation domain.
These mappings are stored in an \emph{evaluaiton dictionary}
\index{dictionary} 
containing \emph{dictionary entries}.
\index{dictionary entry}
Each dictionary entry contains a term (DCETERM) and its interpretation
(DCEINT).
There is a third field in each entry (DCENAME) where a string 
denoting the interpretation  may be stored.
So far DCENAME  is unused.
Dictionary entries are composed with DCECONS.

A dictionary may contain entries with incomplete information in the DCEINT 
field.
Such dictionaries are called \emph{predictionaries}.
The empty (pre)dictionary is EMPTYDC.
PDCINSE inserts a new possibly incomplete entry in a (pre)dictionary and
An incomplete entry in a predictionary can be completed with PDC2DC.
PDCSIZE determines the size (rank) of a (pre)dictionary.
Some applications call PDCSIZE with a 
\emph{pseudo dictionary} that is a list of empty lists.

The following access functions to dictionaries are available.
DCIE($k$,$D$) returns the $k$-th entry in the dictioanry $D$ if
PDCSIZE($D$) is greater than or equal to $k$.
TDILUP returns the entry in an dictionary that contains a given term
and TDIVAL returns returns the interpretation associated with a given
term in a dictionary.

\subsection{The {\tt teev}-Module}

The module {\tt teev} exports a single function TNMEVI that normalizes
a term \wrt\ evaluation using an innermost-evaluations-first strategy.

An \emph{evaluation candidate}
\index{evaluation candidate}
is a subterm $t|_p$ of $t$ such that
\begin{enumerate}
\item $t(p)$ is an evaluable function symbol for an external sort $x$ and
\item if $p = q.i$ for \( i \in \Nat \) then $t|_q$ is not  an 
      evaluable function symbol for $x$.
\end{enumerate}
An evaluation candidate is represented by a triple ($t|_p$,$r$,$x$) 
where $t|_p$ is a term that
is evaluable to an object of external sort $x$ and that is referenced from
the FIRST field of a list cell $r$ (replacement position).
EVCONS is a constructor for evaluation candidates.
EVCTERM returns the term, EVCRPOS returns the list cell and 
EVCSORT returns the external sort of an evaluation candidate.
EVCDIS displays all components of an evaluation candidate.

The evaluation algorithm computes all innermost evaluation candidates
(TRPIEC) that have not been normalized yet.
For these candidates a common evaluation dictionary  is computed
(EVCLGD/TEGDIC) \wrt\ which the candidates are interpreted and
retranslated (EVCLEV).
This is repeated until all evaluation candidates are normalized.
Marking a normalized evaluation candidate is done by binding it to a 
fresh variable that is replaced into the replacement position of the 
evaluation candidate.
This is dome at the end of EVCLEV).
These marks are removed during the computation of the dictionary (TEGDIC)
and on exit from TNMEVI.

TNMEVI normalizes its argument \wrt\ the built in evaluation domains provided
no function symbol is evaluable for two different evaluation domains.

\subsection{Evaluation Domains}

The following evaluation domains are prepared in th ReDuX system.

\subsubsection{Free Groups}

Free groups are represented by list of integers where positive numbers
represent generators and negative numbers represent inverted generators.
The function computing the $i$-th generator is FGFG. 
The function computing a term from a group object is TFFGE.
The following operations must be declared in the same order as in the
following table.

\begin{center}
\begin{tabular}{l|l|l}
\hline
operator & arity & interpretation \\
\hline
neutral element & 0 & FGIDNT \\
group operation & 2 & FGPROD \\
inversion & 1 & FGINV\\
\hline
\end{tabular}
\end{center}

\subsubsection{Abelian Groups}

Abelian groups with $k$ generators are represented as lists of integers
of length $k$.
The $i$-th entry denotes the multiplicity of the $i$-th generator in a
group element.
A negative value denotes the respective positive multiplicity of the 
inverted generator.
The function computing the $i$-th generator is AGFG. 
The function computing a term from a group object is TFAGE.
The following operations must be declared in the same order as in the
following table.

\begin{center}
\begin{tabular}{l|l|l|l}
\hline
operator & arity & interpretation & remark \\
\hline
neutral element & 0 & AGZERO \\
group operation & 2 & AGSUM \\
inversion & 1 & AGNEG\\
difference & 2 & AGDIF & optional \\
\hline
\end{tabular}
\end{center}

\subsubsection{Commutative Rings}

Commutative rings with $k$ generators are represented as $k$-variate
polynomials over the integers.
The function computing the $i$-th generator (indeterminate) is IPRFG. 
The function computing a term from a group object is TFIPRE.
The following operations must be declared in the same order as in the
following table.

\begin{center}
\begin{tabular}{l|l|l}
\hline
operator & arity & interpretation \\
\hline
zero & 0 & PRZERO \\
one & 0 & IPRONE \\
addition & 2 & IPRSUM \\
multiplication & 2 & IPRPRO \\
additive inversion & 1 & IPRNEG \\
\hline
\end{tabular}
\end{center}

\subsubsection{Boolean Rings}

Boolean rings with $k$ generators  are represented as $k$-variate
Boolean polynomials.
The function computing the $i$-th generator (indeterminate) is IPRFG. 
The function computing a term from a group object is TFIPRE.
The following operations must be declared in the same order as in the
following table.

\begin{center}
\begin{tabular}{l|l|l|l}
\hline
operator & arity & interpretation & remark \\
\hline
zero & 0 & PRZERO \\
one & 0 & IPRONE \\
addition & 2 & BPRSUM \\
multiplication & 2 & BPRPRO \\
additive inversion & 1 & BPRNEG & optional\\
\hline
\end{tabular}
\end{center}

An alternative implementation of Boolean rings is provided by the
FDD-package form J"org Bullmann.
It is accessible through the following functions.
The function computing the $i$-th generator  is FDDFG. 
The function computing a term from a group object is TFFDDE.
The following operations must be declared in the same order as in the
following table.

\begin{center}
\begin{tabular}{l|l|l|l}
\hline
operator & arity & interpretation & remark \\
\hline
zero & 0 & FDDZER \\
one & 0 & FDDONE \\
addition & 2 & FDDXOR \\
multiplication & 2 & FDDAND \\
additive inversion & 1 & FDDNEG & optional\\
not & 1 & FDDNOT & optional \\
or & 2 & FDDOR & optional \\
\hline
\end{tabular}
\end{center}