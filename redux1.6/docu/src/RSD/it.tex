\section{The IT-Subsystem} \label{se:it}

The \redux\ subsystem {\it it} is so far an experimental subsystem to allow
for writing an interpreter which has access to \ALDES\  algorithms
which are compiled and linked to the program.
In other words the {\it it} subsystems provides a portable support for a
functionality typically provided by a dynamic link loader.

{\em Algorithms}
\index{algorithm}
are represented by the symbols of their name which have an
{\em algorithm key} stored as a property under indicator DAK.
\index{algorithm key}
An algorithm key is a list of length four as shown in Table~\ref{ta:ak}
\begin{table}
\begin{center}
\begin{tabular}{|c|c|c|c|}
 \hline
 field  & mnemonic &  contents & type \\
 \hline\hline
 1 & ALGIDNT & algorithm identifier & $\beta$-integer \\
 \hline
 2 & ALGTYPE & algorithm type       & 0 = PROCEDURE \\
   &         &                      & 1 = FUNCTION \\
 \hline
 3 & ALGIPAR & number of input parameters & $\{0, \ldots, \mbox{ITMAXIN} \}$ \\
 \hline
 4 & ALGOPAR & number of output parameters & $\{0, \ldots, \mbox{ITMAXOUT}\}$ \\
 \hline
\end{tabular}
\caption{The fields of an algorithm key} \label{ta:ak}
\end{center}
\end{table}
For each algorithm the entry ALGIDNT must be chosen such that the first
three fields of the `algorithm key' form a key (a unique identification) of
that algorithm.

The symbol table initialization is done in ISTINI (cf.\ Section~\ref{se:ini})
which enters the symbol 
DAK and all symbols for algorithms according to the information on a file
normally called {\it program-name}{\tt .algos}. 
Each line of {\it program-name}{\tt .algos} has the format:
\begin{quote}
  algorithm-name algo.-id. algo.-type \#input-parameters \#output-parameters
\end{quote}
\begin{example} A part of {\tt algos} may look like the following:
\begin{verbatim}
AADV  1  0  1  2
ABS  1  1  1  1
ACOMP  1  1  2  1
ACOMP1  2  1  2  1
ADV2  2  0  1  3
ADV3  3  0  1  4
ADV4  4  0  1  5
AFDIF  3  1  2  1
AFINV  4  1  2  1
AFNEG  2  1  1  1
AFPROD  1  1  3  1
\end{verbatim}
\end{example}
The \redux\ {\it it}-subsystem also provides for pseudo functions that
support calls to ALDES-constants and ALDES-operators.
They must be treated as if they were declared as follows:
\begin{verbatim}
x:=ALDESNIL()          : ALDES constant nil               ()
x:=ALDESZERO()         : ALDES constant zero              0
x:=ALDESONE()          : ALDES constant one               1
x:=ALDESNEG(y)         : ALDES operator unary minus       -
x:=ALDESPLUS(y,z)      : ALDES operator plus              +
x:=ALDESMINUS(y,z)     : ALDES operator minus             -
x:=ALDESTIMES(y,z)     : ALDES operator times             *
x:=ALDESDIV(y,z)       : ALDES operator div               /
x:=ALDESEXP(y,z)       : ALDES operator to the power of   ^
\end{verbatim} 


There is a display procedure ALGDIS for algorithm keys and a function
SGCALL to call the algorithm described by a symbol.
In addition there are functions FCALL0, FCALL1, FCALL2, FCALL3, FCALL,
PCALL0, PCALL1, PCALL2, PCALL3, PCALL which call the functions or procedures
with zero, one, two, three or more input parameters respectively
for a given algorithm identifier.

The file {\tt algos} and the functions FCALL0, \ldots, PCALL can be
generated automatically from a file containing headers of \ALDES\ algorithms
(compare with \RIX) using the lex-program 
{\tt algos.lex}.




