\section{The TP-Subsystem} \label{se:tp}

The \redux\ subsystem {\it tp} contains procedures, describing primitive
operations on terms like measuring terms copying terms comparing
terms (\wrt\ equality, subsumption (match) and unification).
In addition this subsection provides data structures for
critical pairs and (explicit representations of) substitutions.

\subsection{The {\tt tpax}-Module}

The module {\tt tpax} contains a collection of useful auxiliary algorithms.

ASMEMN is a axiom membership test for a list of axioms using
axiom numbers as identifiers for axioms.
FREEC determines whether any left-hand side of a rule in a term rewriting 
system $\cal R$ has a top operator which is a member of a given set of 
operators $C$.
Thus FREEC can be used to decide whether the $\cal R$-ground normal forms
are freely generated by $C$ (if $C$ is the set of constructors).
Given an $n$-ary operator $f$, MKTERM constructs a term $f(x_1,\ldots,x_n)$
with new variables $x_1,\ldots,x_n$.
FREECS computes the set of operators (and constants) that do not occur
as a top symbol of a left-hand side of a rule.
Thus (sub)terms with such an operator at the top can not be replaced by a
rewrite.
OVOET extracts one variable of each type (sort) from a set of variables.
SBTERM computes the position of a subterm $t^*$ in a given term term $t$.
TBVSLI computes the unbound variables and linearity information of a term.
TLRVLL computes the unbound variables and linearity information for a list
of related terms,
\index{terms!related}
i.\,e., terms that share variables.
TMIXED is a boolean function which returns TRUE if its argument contains an
\index{term!external}
external term.
TSOOCC is a member function which compares terms according to their TSIGN.
TOC tests whether the constants and operators of a term belong all to a 
given set.
TSFS skips over leading substitutions of a term. 
\index{substitution!leading}
VINT tests whether a free variable occurs in a term.

\subsection{The {\tt tptm}-Module}

The module {\tt tptm} contains functions for measuring terms.
\index{term!measurements}

DEPTH and DEPTH1 measure the depth of a term according to two different
\index{depth}
definitions of the term depth:
the difference being that \( \mbox{DEPTH}(c) = 0 \) and 
\( \mbox{DEPTH1}(c) = 1 \) for a constant $c$ and
\( \mbox{DEPTH}(v) = \mbox{DEPTH1}(v) = 0\) for variables $v$.
MXDPTH computes the maximum term depth (w.\,r.\,t.\ DEPTH) of a list of terms.

The term size (TSIZE) is the number of non-variable positions of a term and
the term extend (TXTEND) is the number of positions of a term.
The size/extend of an AC-operator is its number of argument minus one.
This allows for sensible sizes/extends in case a term is 
\index{term!semi-flat}
semi-flat.

\subsection{The {\tt tpcy}-Module}

Since terms and axioms are represented by lists using {\em structure sharing}
\index{structure sharing}
there are many possibilities of `copying' terms.
\index{term!copy}
The module {\tt tpcy} provides a variety of copy routines and procedures to
partially modify the internal structure of terms.
Some of the procedures simultaneously collect the set of free variables in a 
term.

The procedure TCMPRS modifies a term $t$ by performing all variable
substitutions in $t$.
TFCVS copies a term $t$ to a term $t'$ such that $t$ is equal to $t'$ modulo 
variable renaming and all substitutions in $t'$ are performed.
External terms are copied by copying their pointers.
\index{term!external}
TBPCPM copies the upper non-variable part of a bound term such that the 
bindings of the variables are shared among the the original term and the new 
term.
TGCVS functions as TFCVS, only that the copied term and the original
term share their free variables.

There are algorithms to copy related terms (ACOPY, ALCOPY, TRPCVS) and to rename the
variables of free terms (TCHGVS, TFRV, TLRRV), i.\,e.\ the
argument terms are modified such that their variables are replaced by new 
variables.

The function DTBKUP copies a data type such that the copy contains
flat copies (via COPY) of the constant, variable and operator lists and a deep
copy (via ACOPY and ALCOPY) of the set of axioms.

\subsection{The {\tt tpct}-Module}

The function TLLCST computes all non-variable subterms $t'$ 
of a free term $t$
that contain a variable that occurs also outside of $t'$ in $t$.
The terms computed by TLLCST are called the
{\em linearly constrained subterms} of $t$.
\index{subterm!linearly constrained}
TLLCST uses a data structure called 
{\em marked inverted position list} consisting of a list
of pairs
\( ((m_1,t_1),\ldots,(m_k,t_k)) \).
Each pair $(m_i,t_i)$ consisting of a mark  $m_i$ (integer) and a term $t_i$
is called a {\em marked position}.
\index{marked position}
The $t_i$ in  marked position list are such that $t_i$ is a
principal subterm of $t_{i+1}$.
LLCST temporarily modifies the bindings of the variables in $t$: they 
are `bound' by a list of one marked inverted position list for each
occurrence of the variable.
Common tails of marked inverted position lists are shared.

\subsection{The {\tt tpu}-Module}

The {\tt tpu}-module contains algorithms to unify, to match and to test two
\index{unification}
\index{match}
\index{term!equality}
terms for structural equality (modulo variable renaming).
External terms are treated like constants.

Most algorithms in the {\tt tpu}-module exist in different versions:
a version for general terms as arguments, a version for {\em non-atomic terms}
\index{term!non-atomic}
and/or a version for {\em related non-atomic terms} where
\index{term!related}
related\footnote{Compare with the meaning of {\em related} in
Section~\ref{se:tp}.} means the top operators of both terms have the same TSIGN.

Both the unification and the matching algorithms do not
compute substitutions (mgus or matchers).
Instead they return the most general common instance (mgci) of two terms
or the matched term on success.
If the terms are not unifiable (matchable) the result is the empty list.

Before calling unification or matching algorithms the free variables of the
argument terms should be known because the unification or matching algorithms
modify the argument terms binding their free variables\footnote{Note that if
an argument is a left-hand side of a rule or equation the variables in the
right-hand side are bound too. This is convenient for reductions and critical
pair computations.}.
Thus the mgu or the matcher can be extracted using the algorithm VLXSB
from the module {\tt tpsb}.
To undo the variable bindings the algorithm UNDO from the module {\tt tpdd}
\index{undo}
should be called.
If the mgci is needed for further computations it must be copied before undoing
the variables.
Because of the side effects it is crucial to follow the
specifications of the algorithms in order to determine whether semi-bound
terms are allowed as arguments\footnote{In the {\tt tc}-Module {\em term}
is generally a synonym for {\em free term}.}.

The unification algorithm TMGCI computes the most general common instance of
\index{unification}
two semi-bound terms if it exists otherwise the result is the empty list.
TMGCI uses two different algorithms depending on whether at least one of the
arguments is linear (TNRMIL) or not (TNRMCI).
If the linearity information is known for two non-atomic terms TNMGCI,
and if in addition the top operators of the two terms are equal,
TNRMCI or TNRMIL can be called directly.
All unification algorithms bind the variables of the argument terms
(even in the non-unifying case) which should be undone afterwards.

The matching algorithm TMATCH tests whether a free term $m$ matches a
\index{match}
semi-bound term $t$.
In the positive case $t$ is returned and otherwise the empty list.
The variables of $m$ will be bound to the matching subterms of $t$
(even in the non-matching case) which should be undone after the operation.
TMATCH calls TNRMCH which matches two non-atomic terms with common top
operator.

There is a group of matching algorithms (TNSMTH, TSMTH, TNPSMP) which
are normally used for reducibility tests.
\index{reducibility test}
These algorithms test whether a term $m$ matches a subterm of a term $t$.
Their respective argument terms must be free unless explicitly stated
($3^{\mbox{\scriptsize rd}}$ argument of TNPSMB).
These algorithms have no side effects.

TASMS and TNASMS return the index of  the first rule (in a rule set) by which a
term $t$ can be rewritten.

The algorithm TEQU (which calls TNREQU and TRXEQU) 
tests whether two semi-bound terms
\index{term!equality}
are structurally equal, i.\,e.\ at each position there is the same operator
(TSIGN), constant or variable (TCONT).
The equality of two external objects under equal coercion operators
\index{object!external}
is tested using the equality function whose key is
stored in TUSTAT of the coercion operator. 
By default pointer equality is tested. 
TLEQIV tests whether two semi bound terms are equal modulo variable renaming
provided at least one of the two terms is linear.

Some statistical information is accumulated in global variables which must be
\index{variable!global}
declared externally safe whenever referenced:
\begin{description}
 \item[unification count:] the number of times either TNRMCI or TNRMIL is
  non-recursively called is stored in UC,
 \item[unification time:] the total time spent during the procedures
  TNRMCI and TNRMIL is stored in UTI,
 \item[match count:] the number of times TNRMCH is
   non-recursively called is stored in MC and
 \item[match time:] the total time spent TNRMCH is stored in MTI.
\end{description}
These global variables should be updated whenever the respective procedures are
called.

\subsection{The {\tt tpcp}-Module}

The module {\tt tpcp} provides the abstract data structure {\em critical pair}.
\index{critical pair}
A critical pair is represented as a list of six elements.
Table~\ref{ta:cp} describes the different fields of a critical pair.
\begin{table}
\begin{center}
\begin{tabular}{|c|c|c|c|c|}
 \hline
 field  & mnemonic & name & contents & type \\
 \hline\hline
 1 &  CPW8 & crit.\ pair weight & weight-information & $\beta$-integer \\
 \hline
 2 &  CPLHS & crit.\ pair left-hand side & $s$ & term \\
 \hline
 3 & CPRHS & crit.\ pair right-hand side & $t$ & term \\
 \hline
 4 & CPLOI & crit.\ pair left & axiom number of `left' rule & $\beta$-integer \\
   &       & origination information & used for computing $(s,t)$ & \\
 \hline
 5 & CPROI & crit.\ pair right & axiom number of `right' rule& $\beta$-integer\\
   &       & origination information    & used for computing $(s,t)$ & \\
 \hline
 6 & CPESS & crit.\ pair & states whether $(s,t)$ is & $\{ \mbox{ESS}, 
                                                           \mbox{INESS} \}$ \\
   &       & essentiality flag & essential or inessential & \\
 \hline
\end{tabular}
\caption{The fields of a critical pair $(s,t)$} \label{ta:cp}
\end{center}
\end{table}
If the critical pair $(s,t)$ was computed by superposing the two rules
\( l_1 \rightarrow r_1 \) and \( l_2 \rightarrow r_2 \) with
\( \mu =\mbox{mgu}(l_1|_p,l_2) \) for some position $p$ in $l_1$,
then \( s = l_1[p \leftarrow r_2]\mu \) and \( t = r_1\mu \) and
\( l_2 \rightarrow r_2 \) is the `left' rule  and
\( l_1 \rightarrow r_1 \) is the `right' rule used for computing $(s,t)$.
The essentiality flag is used for inductive completion
\index{essentiality flag}
(see the {\it ic}-module).

There are critical pair display (CPDIS) and construction (CPCONS) procedures.
CPCONS initializes the weight to 0.
CPLW8 enters the weight information (CPW8) into a list of critical pairs.
This weight may be the size of both terms in the pair (CPSIZ), the size of
the maximal term of the pair (CPMSIZ) or the extend of the maximal term of
the pair (CPMXTN).
CPWRT and CPLWRT are output procedures for a single pair or a list of critical
pairs.

We say a critical pair is {\em oriented} if (it is known that) its
\index{critical pair!oriented}
left-hand side is greater than its right-hand side w.\,r.\,t.\ a terminating
\index{term ordering}
term ordering.


\subsection{The {\tt tpsb}-Module} \label{ss:sb}

The module {\tt tpsb} provides the abstract data structure {\em substitution}.
A substitution 
\index{substitution}
\( \sigma = \{ x_1 \mapsto t_1, \ldots, x_n \mapsto  t_n \} \)
is represented by a list S whose first element is an
{\em assignment list} FIRST(S) $=$ ($x_1, t_1, \ldots, x_n, t_n$) where no
\index{assignment list}
$t_i$ may contain a variable $x_j$ for \( 1 \leq i, j \leq n \).
The reductum of S is reserved to store the image of S but it is not used yet.
SUBDOM(S) $=$ ($x_1, \ldots, x_n$) is the domain of $\sigma$ and
SUBRNG(S) $=$ ($t_1, \ldots, t_n$) are the terms (range) of $\sigma$.
SUBIDTY and ASSIDTY are the empty substitutions and the assignment list.
There are operations to construct substitutions (SUBIDTY, SBCONS, VLTLSB
and VLXSB).
Assignment lists are constructed (ASSIDTY, ASCONS) and displayed incrementally
(ASSVAR, ASSTRM, ASSNXT, ASADV, AASSOC).

Applying the substitution $\sigma$ means to bind each variable $x$ in the domain 
of $\sigma$ to $\sigma(x)$.
In order to safely apply substitutions whose domain variables are bound 
the procedures SBFAPP and SBRAPP store these bindings such that they can
be restored by SBFUDO and SBUNDO respectively.
\index{undo}
SBAPP and SBDAPP apply and undo substitutions without storing and restoring 
variable bindings of the domain variables.
SBCMPS compares two substitutions and VLXSB extracts a substitution from a
list of bound variables.
The last two procedures employ the copy routine TGCVS
which copies a term such that its operators are copied and the original 
term and the copied term share the free variables.

