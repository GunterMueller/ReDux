\section{The AX-Subsystem}

The \redux\ subsystem {\em ax} contains auxiliary algorithms which are not 
specific to term manipulation.

A few routines are meant to hide \ALDES\ specific i/o-features.
These procedures may be subject to changes if \redux\ is ported to
\index{porting \redux}
a different environment like SACLIB.
The input unit may be changed using IUCHNG only.
The \ALDES/SAC-2 assignment ``IUNIT:=n'' should not be used.
BLKW may be used to write a certain number of blanks not exceeding the 
end of the line.
ENDOLN tests whether the character currently read is the last character of
the current line.
SFOL test of a given string still fits on the current line.

Then there is a group of routines to support reading and writing
\ALDES/SAC-2 data types.
The function SCHAR scans the character ignoring blanks.
SWSP scans a white-space (blank, end-of-line).
\index{white-space}
WORDRD reads a continous list of non-white-space characters.
This may be usefull to read Unix path names or other names containing special 
symbols.
The function AREADR is a robust version of AREAD (atom read) designed for
interactive input.
LEREAD reads a list of integers that may contain ranges like ``5--11''.
The string output procedures STROUT (string out) and 
STOUTL (sting out and newline) must be called with string
constants only.
They should replace the SAC-2 procedure CLOUT whenever possible.

HALT stops the program with a division by zero error.
\index{division by zero}
This causes normally a core dump which can be examined by a post mortem
debugger 
(HALT is normally called if unexpected or un-allowed branches of the code
are reached).

There is a group of algorithms operating on lists of $\gamma$-integers,
\index{integer!$\gamma${-}}
i.\,e.\ the list elements are treated as numbers and not as atoms or lists
depending on whether they are less than $\beta$ or not.
We use the notion of $\gamma$-integers, if we do not care whether an object
is a pointer ($ > \beta$) or a $\beta$-integer.
$\gamma$-integers are {\em not} modified to ensure the correctness of the
garbage collector.
\index{garbage collector}
INTDIF (difference), INTINS (insert), INTINT (intersection), 
INTMEM (member), INTSBS (subset) and INTUNI (union) are set operations for 
lists of $\gamma$-integers.
Thus set operations on lists (sets) of pointers are possible.
ILPASS searches through two associated lists of $\gamma$-integers.
LP2EVE and LP2ODD extract the elements at even and odd positions
of a list of period two
and LTLCPY copies destructively a list of $\gamma$-integers into another list.

The module {\tt axsl1} contains sorting algorithms for lists of 
sublists where the first element of each sublist is a $\beta$-integer
(weight) according to which the sublists are compared.
Only the algorithm LOISE which inserts a list of sublists into an
ordered list of sublists should be called outside of {\tt axsl1}.
