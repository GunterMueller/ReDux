\section{The TC-Subsystem} \label{se:tc}

The {\it tc}-subsystem provides all operations needed to build a Knuth-Bendix 
completion procedure (\cite{KnuthBendix:67}).
In particular, there are algorithms for unification, matches, reductions,
\index{unification}
\index{match}
\index{reduction}
critical pair computations, and completion.
\index{critical pair}
\index{completion}
The completion procedures implemented roughly follow the strategy depicted in
\index{strategy}
Figure~\ref{fi:cmpl}.
\begin{figure}[ht]
\begin{center}
\fbox{
\begin{minipage}{5.5in} %necessary if there is a footnote within the figure
\begin{quote}
\begin{center} ${\cal R} \leftarrow$ {\bf COMPLETE}$({\cal E},\succ)$
\end{center}
[Completion procedure. \\
$\cal E$ is a set of equations and $\succ$ is a terminating term ordering.
Then ${\cal R}$ is a canonical term rewriting system with
\( =_{\cal R} \; = \; =_{\cal E} \).]
\begin{description}
 \item[{(1) [Initialise.]}] \(  {\cal R} := \emptyset \).
 \item[{(2) [Simplify.]}] {\bf while} the {\em simplify}-inference rule
   applies {\bf do} \\
  \( ({\cal E};{\cal R}) := Simplify(({\cal E};{\cal R})). \)
 \item[{(3) [Delete.]}] {\bf while} the {\em delete}-inference rule
   applies {\bf do} \\
  \( ({\cal E};{\cal R}) := Delete(({\cal E};{\cal R})). \)
 \item[{(4) [Stop?]}] {\bf if} \(  {\cal E} = \emptyset \) {\bf then return}
      ${\cal R}$ and {\bf stop}.
 \item[{(5) [Orient.]}] Let \( a \leftrightarrow b \in {\cal E} \) then
      \( {\cal E} := {\cal E} \setminus \{ a \leftrightarrow b \} \); \\
      {\bf if} \( a \succ b \) {\bf then} \(  \bgn l := a; r  := b \nd \) \\
      {\bf elsif} \( b \succ a \) {\bf then} \(  \bgn l := b; r  := a \nd \) \\
      {\bf else stop} with failure; \\
      \( {\cal R} := {\cal R} \cup \{ l \rightarrow r \} \).
 \item[{(6) [Collapse.]}] {\bf while} the {\em collapse}-inference rule
     applies {\bf do} \\
    \( ({\cal E};{\cal R}) := Collapse(({\cal E};{\cal R})). \)
 \item[{(6') [Subconnectedness criterion.]}] \null \\
    \( {\cal E} := {\cal E} \setminus \{ \mbox{Equations originating from a
    collapsed rule} \} \).
 \item[{(7) [Compose.]}] {\bf while} the {\em compose}-inference rule
   applies {\bf do} \\
   \( ({\cal E};{\cal R}) := Compose(({\cal E};{\cal R})). \)
 \item[{(8) [Deduce.]}] Compute all critical pairs ${\cal P}$ of 
    $l \rightarrow r$ and rules
    in $\cal R$. \( {\cal E} := {\cal E} \cup {\cal P} \)\footnote{Here we
                                             identify pairs and equations.};
    \\ {\bf continue} with step 2. \hfill $\Box$
\end{description}
\end{quote}
\end{minipage}
}
\end{center}
\caption{Algorithm {\em COMPLETE}} \label{fi:cmpl}
\end{figure}
The {\em simplify}-, {\em delete}-, {\em collapse}-, and {\em compose}
inference rules are those presented in \cite{Dershowitz:89}.
The subconnectedness criterion in step 6' was proposed by K{\"u}chlin
\cite{Kuechlin:86}.

In addition to the standard completion procedures shown in 
Figure~\ref{fi:cmpl}, there are modified 
\index{critical pair!criterion}
completion procedures which employ critical pair criteria:
\begin{enumerate}
 \item in addtion to step 6',
  the subconnectedness criterion is also applied in the Deduce step (8)
  as described in \cite{Kuechlin:86}.
\index{subconnectedness criterion}
 \item the transformation criterion \cite{Buendgen:91b} and
\index{transformation criterion}
 \item the combination of the subconnectedness - and the
  transformation criterion.
\end{enumerate}
For each criterion there are special critical pair computation algorithms and
completion procedures.

The algorithms of the {\it tc} subsystem cannot deal with external terms,
except for the reduction algorithms.

The implementation of the major part of the  algorithms in the
{\it tc}-subsystem is described in \cite{Kuechlin:82a}.

\subsection{The {\tt tcr}-Module}

The {\tt tcr} module contains the term reduction algorithm TIREDS which 
\index{reduction}
normalises a free term $t$ w.\,r.\,t.\ a set of rewrite rules $\cal R$.
\index{normalisation}
The result is $t$ is $t$ if irreducible otherwise TIREDS returns the normal
form $t^*$ of $t$ and $t$ is destroyed.
The innermost reduction strategy used in TIREDS is discussed in
\index{reduction!strategy}
\cite{Kuechlin:82b} and \cite{Stickel:83}.

TAPPLY computes a one step reduction at the top position of a non-atomic term.
\index{term!non-atomic}
The term resulting from TAPPLY is semi-bound, the bound variables serving
as marks.
External terms can be reduced using TXAPLY.
\index{term!external}

A term computed by TAPPLY or TIREDS may have non-variable subterms with the
same representation, i.\,e.\ there is one representation of a subterm
in the memory and more than one pointer pointing to it.
Such terms should normally be avoided in the ReDuX system.
\index{term!copy}
Copying the term (with TFCVS, ACOPY, TRPCVS from {\tt tpcy}) results in a 
correct representation of the term (with variables renamed).
Also TSTDR destructively modifies its argument (a term --- possibly in the 
non-standard representation resulting from TAPPLY or TIREDS) resulting in an
equivalent term in which only constants, variables and extended terms may
be shared.

TIREDS assumes that constants are irreducible, in particular no left-hand side
\index{constant}
of a rule may be a constant.
However there are no such restrictions for nullary function symbols
(operators with ODTYPE equal to the empty list).

The time spent in the non-trivial parts of TIREDS and in TSMTH 
(reducibility test, see {\tt tpu}) is accumulated in the global variable RTI.
The number of successful reductions is counted in the global variable RC.

\subsection{The {\tt tckbq}-Module}

The module {\tt tckbq} contains algorithms to administer lists ({\tt q} for
waiting queues) of axioms and critical pairs used in the Knuth-Bendix
\index{axiom}
\index{critical pair}
\index{queue}
\index{Knuth-Bendix completion}
completion procedure.

Sets of axioms (equations or rewrite rules) are organised as lists 
whose elements are axioms  which are sorted in ascending order of the sizes 
of their respective left-hand sides.
ALINS inserts an axiom into a sorted list of axioms.
ALPUR partitions a set of axioms into a set of axioms with reducible 
left-hand sides and one with irreducible left-hand sides (w.\,r.\,t.\ to
a given rule).
This is used to implement the collapse-inference rule in the completion
algorithm (see Figure~\ref{fi:cmpl}).

To construct queues of critical pairs the algorithm LOISE (from {\tt axsl1})
can be used because the first field of a critical pair contains the 
weight information according to which the pairs must be ordered.
In addition there are algorithms to separate irreducible pairs from
reducible pairs (CPLCLR, CPLOPR), 
\index{critical pair!normalisation}
to normalise a list of critical pairs (CPNLNF, CPOLNF, CPLOPR) and to apply
\index{subconnectedness criterion}
the subconnectedness criterion (\cite{Kuechlin:86}) to old critical pairs
(CPLCCA).
CPNLNF and CPOLNF differ only in their trace output.
CPLOPR combines the separation and the normalisation steps and assumes that
the left-hand sides (CPLHS) of the critical pairs are greater than the
respective right-hand sides.

\subsection{The {\tt tcsrt1}-Module}

The algorithm LAXBMS from the module {\tt tcsrt1} sorts a list of axioms
\index{sorting}
according to their axiom number (ANUM) in ascending order.
It is based on the LBIBMS algorithm from the SAC-2 {\it lp}-subsystem.
This algorithm is parametrised using macro techniques.

\subsection{Critical Pair Computation}

There are four algorithms to compute critical pairs.
\index{critical pair}
All of them essentially compute all critical pairs between a single rule $r$ 
and a set of rules $\cal R$.
This means for all \( r' \in {\cal R} \) to superpose $r$ into $r'$ and
$r'$ into $r$ and to compute all {\em intrinsic critical pairs} between
\index{critical pair!intrinsic}
$r$ and a copy of $r$.

The module {\tt tccpc} contains a standard critical pair computation 
without criteria.
Its main algorithm is ASCPS.

The module {\tt tccpc-s} implements the critical pair computation with
subconnectedness criterion (\cite{Kuechlin:86}).
\index{subconnectedness criterion}
Its main algorithm is ASCPSU.
There is also an algorithm RSDCPS computing all normalised critical pairs of
a term rewriting system which can be used to test whether the system is
\index{confluence}
confluent.

The module {\tt tccpc-t} implements the critical pair computation with
the application of the transformation criterion (\cite{Buendgen:91b}).
\index{transformation criterion}
The interface of the main algorithm ASCPST differs from that of ASCPS and 
ASCPSU.
If the transformation criterion applies the rule which reduces the left-hand 
side of $r$ is returned.
Otherwise the resulting set of critical pairs is partitioned into
\index{critical pair!oriented}
oriented critical pairs and non-orientable critical pairs.
All returned critical pairs are ${\cal R} \cup \{r\}$-normalised.

The module {\tt tccpc-st} uses a combination of the subconnectedness- and the
transformation criteria.
Its main algorithm ASCPTU has the same interface as ASCPTS.

\subsection{The Knuth-Bendix Completion}

There is one completion algorithm for every critical pair computation procedure.
\index{completion}
The completion strategy implemented is roughly that of Figure~\ref{fi:cmpl}.
One difference being that the set of equations $\cal E$ is partitioned into 
three or four waiting queues of axioms or (oriented) critical pairs.
\index{queue}
\index{axiom}
\index{critical pair}
All completion algorithms apply the subconnectedness criterion in the
\index{subconnectedness criterion}
collapse step (CPLCCA).

Each completion procedure can either be run in step mode (global variable 
\index{step mode}
STEPM $ = 1$) or in automatic mode (STEPM $= 0$).
\index{automatic mode}
In {\em automatic mode} the whole completion runs automatically as long as all
equations and critical pairs are orientable.
If the completion is run in {\em step mode} the program is interrupted at 
every orientation step.
The user can then orient the critical pair by hand, toggle traces,
delay the consideration of a critical pair and so on.
The details are described in \RUD.

All completion procedures display statistical and trace output.

The module {\tt tckb0} contains the completion algorithm DTKB which computes
critical pairs without using a criterion.
The set of equations $\cal E$ is stored as
\begin{enumerate}
 \item a queue Q1 for initial equations and collapsed rules,
 \item a standard critical pair queue Q2 and
 \item a queue Q3 where critical pairs are held which 
    were manually but back (option ``B'' in QCA).
\end{enumerate}
Q1 has priority over Q2 which has priority over Q3.

The module {\tt tckb-s} contains the completion procedure DTKBC which computes
critical pairs applying the subconnectedness criterion.
\index{subconnectedness criterion}
Otherwise it differs from DTKB only in some statistical and trace output.

The module {\tt tckb-t} contains the completion procedure DTKBT which computes
critical pairs applying the transformation criterion.
\index{transformation criterion}
I.\,e.\ whenever the left-hand side of the last oriented rule can be reduced 
by a new oriented critical pair, all other critical pairs computed with this 
rule are discarded, the rule is pushed back onto Q0 and the oriented critical 
pair is considered as the next new rule.
The set of equations is stored as
\begin{enumerate}
 \item a queue Q0 for initial equations and collapsed rules,
 \item a queue Q1 of oriented and normalised critical pairs,
 \item a queue Q2 of (initially non-orientable) critical pairs and
 \item a  queue Q3 where critical pairs are held which 
    were manually but back (option ``B'' in QCA).
\end{enumerate}
Q0 has the  highest and Q3 the lowest priority.
The priorities  of Q1 and Q2 are equal.
 
The module {\tt tckb-st} contains the completion procedure DTKBTC which 
combines the subconnectedness and the transformation criterion during the 
critical pair computation.
It differs from DTKBT only in some statistical and in some trace output.

The main program TC %call DTKB, DTKBC, DTKBT and DTKBTC respectively.
first reads a data type from input stream 4, displays the data type.
The the use can select from a menu whether he wants to match or unify
two terms, test for structural equality, prove equality modulo $\cal R$,
normalise a term, define term ordering, perform termination tests and
and run one of the completion procedures (DTKB, DTKBC, DTKBT, DTKBTC)
described above.

There are interactive routines in {\tt tcint}, {\tt mexit} and {\tt qct}.
A leading M (for message) in the procedure name means that information is only
\index{message}
output whereas a leading Q (for query) denotes a procedure containing a 
\index{query}
dialogue.
