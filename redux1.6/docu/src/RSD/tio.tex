\section{The TIO-Subsystem} \label{se:tio}

The \redux\ subsystem {\it tio} contains procedures, describing the
fundamental data structures needed for term rewriting and 
\index{data structure}
some primitives manipulating these data structures.

The prominent data structure of the \redux-system is the data structure 
\index{term}
{\em term}.
Terms may either be atomic or non-atomic where {\em atomic terms}
\index{term!atomic}
are variables, constants, external terms.
\index{variable!print name}
Each constant and each variable is uniquely represented in the system\footnote{
However different variables may have a common {\em print name} as explained
in Section~\ref{ss:tpdd}.}.
A variable may either be free (unbound) or bound.
A {\em free (or unbound) variable} is a variable with no substitution applied 
\index{variable!free}
\index{variable!unbound}
to it.
A {\em bound variable} has a term substituted to it and represents this
\index{variable!bound}
substituted term.
The situation where a variable is bound to a bound variable is sometimes
\index{substitution!chain}
called a {\em chain substitution}.
An {\em external term}
\index{term!external}
consists of a unary {\em coercion operator}
\index{operator!coercion}
\index{object!external}
whose argument is an arbitrary {\em external object}.

{\em Non-atomic terms} consist of a top operator whose principal subterms 
\index{term!non-atomic}
(arguments) are terms.
Note that even though a `coercion operator' is the top symbol of an external
term, external terms are not considered to be non-atomic.
Since constants and variables are uniquely represented, a term is rather a
\index{variable!unique representation of}
directed acyclic graph (DAG) than a tree.
\index{DAG}
\index{directed acyclic graph}
According to whether the variables in a term $t$ are free or bound $t$ is
called a free or (semi-)bound term.
In a {\em free term} no variable is bound, in a {\em semi-bound term} some
\index{term!free}
\index{term!bound}
\index{term!semi-bound}
variables may be bound and in a {\em bound term} all variables are bound.
If a term is referenced by a (chain of) bound variable(s) these binding(s)
are called a {\em leading substitution}.
\index{substitution!leading}
\index{term!related}
A set of terms is {\em related}\,\footnote{Unfortunately some algorithm 
descriptions use the keyword `related' to denote two terms with common
top operator.} if they share common variables.
Thus the two terms of an equation or rewrite rule are generally related.

Operations which bind variables are mgci computations, matching operations 
and operations which {\em apply substitutions}.
\index{substitution!apply}
These operations actually have side effects on their input arguments.
There are also operations to {\em undo} variable bindings,
\index{undo}
i.\,e.\ which make bound variables free.

Sometimes it is necessary to {\em perform a substitution} in a semi-bound
\index{substitution!perform}
term $t$.
Then a bound variable in $t$ will be replaced by its binding.
A term with performed substitutions is equal to the original term up to the fact that it cannot be modified by an undo operation any more.

The argument list of associative commutative (AC) operators may be larger than 
\index{operator!AC{-}}
two meaning that all arguments are combined by AC-operators (with 
arbitrary setting of parenthesis).
Such terms are called {\em semi-flat}.
\index{term!semi-flat}
Terms in which no AC-operator has a principal subterm with that
AC-operator as top symbol are called {\em flat}.
\index{term!flat}
\index{term!flattened}
In \redux\ terms are not generally normalized to flat terms because
flattening semi-bound terms possibly destroys variable bindings.

\subsection{Definitions of Basic Data Structures: The {\tt tpdd}-Module} 
\label{ss:tpdd}

The {\tt tpdd}-module provides for the ReDuX system initialization and the
definition of the basic data structures.
The following data structures are defined in the {\tt tpdd}-module: 
\begin{itemize}
 \item constants,
\index{constant}
 \item operators,
\index{operator}
 \item coercion operators,
\index{operators}
 \item variables,
\index{variable!ReDuX}
 \item external terms
\index{term!external}
 \item terms,
\index{term}
 \item axioms and
\index{axiom}
 \item data type.
\index{data type}
\end{itemize}
All of them are implemented as list structures.
The first six data structures can be divided into two classes:
{\em signature objects} (constants, variables, operators  and
coercion operators) which provide the prototypes of the parts terms are built
from and actual occurrences of the above objects as the  top symbols of terms
(constants, variables, non-atomic terms and external terms).
Note that we {\em call} both the prototype-variable and the variable-term a
variable; and
prototype constants and constant-terms have even the same representation.

The list structure of the first six data structures is the same.
Thus signature objects and terms can be overloaded.
\index{overload}
Table~\ref{tb:tsy} shows the structure of lists denoting 
constants, (coercion) operators, variables or (external) terms.
\begin{table} 
\begin{center}
\begin{tabular}{|l|l|l|l|}
 \hline
 field no. & mnemonic & name & possible entries \\
 \hline\hline
 1 & TCONT & term contents & {\it nil}          \\
   & OARGLST & operator argument list & list of terms \\
   & VARBIND & variable binding & {\it nil}, term \\
   & XOBJECT & external object  & external object     \\
 \hline
 2 & TKIND & term kind     & one of DCS, DOS, DVS, DXT \\
 \hline
 3 & TTYPE &  term type & symbol denoting a sort\\
   & ORTYPE& operator range type & \\
 \hline
 4 & TUSTAT & unification status & {\it nil}, DAC, DCO,  algorithm key \\
 \hline
 5 & ODTYPE & operator domain type & list of symbols denoting sorts\\
 \hline
 6 & TNAME & term name & symbol denoting operator name \\
 \hline
 $>$ 6 & TPROP & term property & property list \\
 \hline
 $>$ 1 & TSIGN & term signature & reductum of a term pointer \\
 \hline
\end{tabular}
\caption{The structure  of constants, operators, variables and terms} 
\label{tb:tsy}
\end{center}
\end{table}
For variables and constants the field ODTYPE is set to {\it nil}. 
\index{nil}
For signature objects the TCONT is always {\it nil} (implying that this is
also true for constant-terms).
If a pointer points to a non-atomic term it points to the top operator
of that term. The term list structure can then be considered as a variant
record according to the contents of the TKIND field.
In this case Table~\ref{tb:tse} displays the meaning of the different
fields of the term list structure.
\begin{table}
\begin{center}
\begin{tabular}{|l|c|c|c|c|}
 \hline
         & \multicolumn{4}{c|}{kind of term} \\
 field   & constant       & variable   & non-atomic   & external \\
         &                &            & term         & term     \\
 \hline\hline
 TCONT   &                & VARBIND    & OARGLST      & XOBJECT \\
         & {\it nil}      & substitution & list of principal & external \\
         &                & value      & subterms     & object \\
 \hline
 TKIND   & DCS            & DVS        & DOS          & DXT \\
 \hline
 TTYPE   & \multicolumn{4}{c|}{sort} \\
 \hline
 TUSTAT  & {\it nil}      & {\it nil}  & free-, AC- or & key of equality \\
         &                &            & commutative op. & function \\
 \hline
 ODTYPE  & {\it nil}      & {\it nil}  & list of domain sorts & list of one \\
         &                &            & of top operator      & external sort \\
 \hline  
 TNAME   & \multicolumn{4}{c|}{name of top symbol} \\
 \hline
 TPROP   & \multicolumn{4}{c|}{property list of top symbol} \\
 \hline
\end{tabular}
\caption{The meaning of the fields of term list structure} \label{tb:tse}
\end{center}
\end{table}

The {\em reductum} (fields 2, 3, \ldots) of a list denoting a constant, 
\index{reductum}
operator, variable or term is called the {\em signature} (TSIGN) of that object.
\index{signature}
The signature is stored only once for every constant, operator or
variable name known to the system whereas the {\em contents} (TCONT) is stored
\index{contents}
for each occurrence of an operator or variable name on some term of the system.
The representation of all constants is unique, so is the pointer to
their contents.

Terms are represented by directed acyclic graphs to ensure the unique
\index{directed acyclic graph}
\index{variable!unique representation of}
representation of a variable which occurs twice in a term or an axiom.
Constants are always shared and external terms may be shared but this is 
not mandatory.
Normally sharing more complex terms than variables (or constants 
or external terms) is avoided. 
Exceptions being substituted variables (see module {\tt tcu} in 
the {\it tp}-subsystem)
and results of reductions (see module {\tt tcr} in the {\it tc}-subsystem).
On the other hand it is often convenient to use the same variable name several 
times in different contexts.
In such a case two variables may share the signature information
(TSIGN) but have different TCONT (alias VARBIND) fields.
\begin{example} Consider
 \[ {\cal E} = \{ *(x,0) \leftrightarrow 0, \;\;
                  *(x,+(y,-(y))) \leftrightarrow *(0,y) \}
 \]
 Here the name $x$ in the first and second equations denotes {\em different}
 variables but the two occurrences of $y$ in the second equation
 denote the same variable.

 From a logical point of view the variables in the two different equations
 are bound by two different quantifiers:
 \[ (\forall x: *(x,0) \leftrightarrow 0), \;\;
    (\forall x, y: *(x,+(y,-(y))) \leftrightarrow *(0,y)).
 \]

 The two occurrences of $x$ will be implemented by two different lists
 with two different first fields (TCONT alias VARBIND) 
 but common reductum (TSIGN).
 The three occurrences of $y$ are represented by pointers to the same list.
 Thus the unique variable representation implements the scope of
 a (quantified) variable.
 The minimal scope of a variable is a term.
 \hfill $\Box$
\end{example}
Even though  TCONT is an overloaded function for all kinds of terms,
its use should be avoided whenever possible.
Instead the appropriate alias (OARGLST for non-atomic terms,
VARBIND for bound and unbound variables and XOBJECT for terms with a
coercion operator at their top) should be used to facilitate
future changes in the representation of argument lists,
variable bindings etc. (cf.\ \cite{BuendgenGoebelKuechlin:94}).

The VARBIND field of a variable term $x$ field may contain a pointer to a 
term $t$ meaning that $x$ is bound by a substitution 
$\{ x \mapsto t, \ldots\}$.
If a variable is unbound  (or  mapped to itself) its VARBIND is set 
to {\it nil}.
The function UNDO can be used to undo the variable bindings of a list
\index{undo}
of (bound) variables.

The signature of each constant,  variable or operator may be augmented by
additional information in an optional property list.
A {\em property list} is a list ($i_1, p_1, \ldots, i_n, p_n$) of period two
\index{property list}
where the $i_j$ are called {\em indicators} and $p_j$ is the 
\index{indicator}
{\em property under indicator} $i_j$.
\index{property}
The following operator properties are known:
\begin{itemize}
 \item The indicator DFS determines the notation used for an operator:
       The property DEF means prefix, DIF means infix, DPF means
       postfix, DRF means roundfix and DLF means Lisp notation.
       The property DCS means constant notation (i.\,e.\ no argument list)
       for nullary operators.
       The default is constant notation for nullary operators and
       prefix notation otherwise for the old parser.
       With the new parser the default notation for constants and nullary
       function symbols is constant (DCS), for unary function symbols
       prefix (DPF), for binary function symbols infix (DIF) and
       function notation (DFF).
 \item Under indicator DRF the closing part of an roundfix operator is
       stored.
 \item The indicator DAV provides a variable of the argument type of the
       operator. This is only used for C- and AC-operators.
 \item The indicator DXIA is used to map an operator to a function
       over external objects. The symbol denoting the algorithm  (accessible
       via the {\it}-module) interpreting
       that function is stored as the property of DXIA.
 \item Further properties are used to define term orderings.
       These properties may also be attached to constants and variables.
 \item For a constant or function symbol we can store
       under the indicator DXCO a coercion operator 
       that maps the external interpretation of a term headed by the
       constant or function symbol to the result type of that constant or
       function symbol.
       It overwrites the default stored as a property of the result type.

\index{term ordering}
       (See Section~\ref{se:to}).
\end{itemize}
The functions and procedures to manipulate these property lists are TGET 
(get property under a given indicator), 
TPUT (insert or update property under a given indicator) and 
TPRRM (remove property under a given indicator).

The data structure {\em axiom} is used to represent equations and rewrite rules.
\index{axiom}
\index{equation}
\index{rewrite rule}
It consists of a list with at least five fields.
Table~\ref{tb:ax} shows the fields of an axiom denoting an equation
$ l \leftrightarrow r$ or a rewrite rule $ l \rightarrow r$ respectively.
\begin{table}
\begin{center}
\begin{tabular}{|l|c|c|c|}
 \hline
 field no. & mnemonic & name & contents \\
 \hline\hline
 1 &  ALHS & axiom left-hand side & $l$ \\
 \hline
 2 & ALHSV & axiom left-hand side variables & list of variables in $l$  \\
 \hline
 3 & ARHS & axiom right-hand side & $r$ \\
 \hline
 4 & ANUM & axiom number & number of axiom  \\
 \hline
 5 & ALLB & axiom left-linearity bit & 1 iff $l$ is linear otherwise 0  \\
 \hline
 $>$ 5 & APROP & axiom properties & property list \\
 \hline
\end{tabular}
\caption{The fields of axiom $ l \leftrightarrow r$ or $ l \rightarrow r$} \label{tb:ax}
\end{center}
\end{table}
If the top operator of the left-hand side of a rewrite rule is associative
and commutative an AC-extension rule \cite{PetersonStickel:81} may
\index{rule!AC-extension}
be stored in the axiom property list under indicator DXR.
The function AGET and the procedure APUT may be used to look at properties
and update property lists respectively.

The data structure for {\em data type} is a list containing its name,
\index{data type}
signature (constant-, variable- and operator sets) and a set of axioms.
The new parser also needs the set of sorts, external sorts and a
list of all symbols occuring in the signature (constants, variables and
function symbols).
See Table~\ref{tb:dt}.
\begin{table}
\begin{center}
\begin{tabular}{|l|c|c|c|}
 \hline
 field no. & mnemonic & name & contents \\
 \hline\hline
 1 &  DTAX & data type axioms & $\cal R$ \\
 \hline
 2 & DTCONS & data type constants & list of constants  \\
 \hline
 3 & DTVARS & data type variables & list of variables \\
 \hline
 4 & DTOPERS & data type operators & list of function symbols  \\
 \hline
 5 & DTNAME & data type name & name symbol \\
 \hline
 6 & DTSORTS & data type sorts & list of sorts \\ \hline
 7 & DTXSORTS & data type ext.\ sorts & list of ext.\ sorts \\ \hline
 8 & DTSYMLST & data type symbol list & list of signature objects \\ \hline
  $>$ 8 & DTPROP & data type properties & property list \\
 \hline
\end{tabular}
\caption{The fields of the data type data structure} \label{tb:dt}
\end{center}
\end{table}

The {\tt tpdd} contains procedures to construct 
\index{construct}
axioms (ACONS), constants (CCONS), data types (DTCONS), operators (OCONS)
terms (TCONS), prototype variables (VCONS), external operators (XOCONS) and 
\index{display}
display all these data structures or parts of them
(ADIS, CDIS, DTDIS, TDIS, VDIS).

\subsection{The Syntax of the Data Type -- Old Parser}
In \redux\ an algebraic specification is called a data type.
In this section, we want present the syntax of data types, axioms and terms.
A \redux\ data type consists of a data type name, external type,
constant, variable and operator declarations with associated types (sorts)
and a set of axioms.
In Figures~\ref{dtgram1}, \ref{axgram} and \ref{dtgram2}
the syntax of a  \redux\ data type is presented in
extended BNF.
%It is the same given in \cite{Kuechlin:82a} and
%\cite{Buendgen:87} with the addition of properties assigned to the
%operators to denote unification states
%(for the {\tt ac} program),
%operator notation (prefix, infix, postfix, roundfix and
%Lisp notation) and to supply additional information
%to deal with external objects.

%If the first symbol of the data type specification file is not
%{\tt TYPE} then this symbol is declared as comment symbol and may be used
%to declare all characters to the right of the comment symbol to be
%a comment in the remaining input.
%This comment symbol is then known to the rest of the program and
%may be used whenever terms or theorems are read in.
%If the first symbol of the data type specification file is {\tt TYPE} then
%no comment symbol is available.
\begin{figure}[htbp]
\begin{center}
\noindent
\rule{\textwidth}{0.6pt}

%\fbox{
\begin{minipage}{6in}
\begin{tabbing}
{\vs PrecedenceDecl} \= \gprod \= \kill
{\vs DataType} \> \gprod \ts{DATATYPE} {\vs DTName} \ts{;} \gstar{{\vs Declaration}} \ts{END} \\
\\
{\vs Declaration} \> \gprod {\vs SortDecl} \galt {\vs ExternalDecl}
  \galt {\vs ConstantDecl} \galt {\vs VariableDecl} \\
  \>\> \galt{\vs OperatorDecl} \galt {\vs CoercionDecl} \galt {\vs PrecedenceDecl} \galt
       {\vs AssocDecl} \\
  \>\> \galt {\vs NotationDecl} \galt {\vs TheoryDecl} \galt {\vs PropertyDecl}
       \galt {\vs AxiomDecl} \\
{\vs SortDecl} \> \gprod \ts{SORT} {\vs SortNames} \ts{;} \\
{\vs ExternalDecl} \> \gprod \ts{EXTERNAL} \gplus{{\vs External} \ts{;}} \\
{\vs ConstantDecl} \> \gprod \ts{CONST} \gplus{{\vs Constant} \ts{;}} \\
{\vs VariableDecl} \> \gprod \ts{VAR} \gplus{{\vs Variable} \ts{;}} \\
{\vs OperatorDecl} \> \gprod \ts{OPERATOR} \gplus{{\vs Operator} \ts{;}} \\
{\vs CoercionDecl} \> \gprod \ts{COERCION} \gplus{{\vs Coercion} \ts{;}} \\
{\vs PrecedenceDecl} \> \gprod \ts{PREC} \gplus{{\vs Precedence} \ts{;}} \\
{\vs AssocDecl} \> \gprod \ts{ASSOC} \gplus{{\vs Associativity} \ts{;}} \\
{\vs NotationDecl} \> \gprod \ts{NOTATION} \gplus{{\vs Notation} \ts{;}} \\
{\vs TheoryDecl} \> \gprod \ts{THEORY} \gplus{{\vs Theory} \ts{;}} \\
{\vs PropertyDecl} \> \gprod \ts{PROPERTY} \gplus{{\vs Property} \ts{;}} \\
{\vs AxiomDecl} \> \gprod \ts{AXIOM} \gplus{{\vs Axiom} \ts{;}} \\
\\
{\vs External} \> \gprod {\vs XSortNames} \ts{:} {\vs XInterface} \gstar{\ts{,} {\vs XInterface}} \\
{\vs XInterface} \> \gprod \group{\ts{XREAD} \galt \ts{XWRITE} \galt \ts{XEQ} \galt \ts{XLT}
  \galt \ts{XFG} \galt \ts{XTERM}} \ts{=} {\vs AlgoName} \\
{\vs Constant} \> \gprod {\vs ConstNames} \ts{:} {\vs SortName} \\
{\vs Variable} \> \gprod {\vs VarNames} \ts{:} {\vs SortName} \\
{\vs Operator} \> \gprod {\vs OpNames} \ts{:} {\vs DomainList} \ts{->} {\vs Range} \\
{\vs DomainList} \> \gprod {\vs Domain} \gstar {\ts{,} {\vs Domain}} \galt \gepsilon \\
{\vs Coercion} \> \gprod {\vs CoercionOpName} \ts{:} {\vs Domain} \ts{->} {\vs Range} \\
{\vs Precedence} \> \gprod {\vs OpNames} \gplus{\group{\ts{<} \galt \ts{=}} {\vs OpNames}} \\
{\vs Associativity} \> \gprod {\vs OpNames} \ts{:} 
  \group{\ts{LEFT} \galt \ts{RIGHT} \galt \ts{NONE}} \\
{\vs Notation} \> \gprod {\vs OpNames} \ts{:} {\vs Fixity} \\
{\vs Fixity} \> \gprod \ts{PREFIX} \galt \ts{POSTFIX} \galt \ts{INFIX} \galt \ts{CONSTANT} \\
  \>\> \galt \ts{FUNCTION} \galt \ts{LISP} \galt \ts{ROUNDFIX} {\vs ClosingRoundfixName} \\
{\vs Theory} \> \gprod {\vs OpNames} \ts{:} \group{\ts{AC} \galt \ts{COM}} \\
{\vs Property} \> \gprod {\vs ObjectNames} \ts{:} {\vs PropList} \\
{\vs PropList} \> \gprod {\vs Prop} \gstar{\ts{,} {\vs Prop}} \\
{\vs Prop} \> \gprod {\vs Indicator} \ts{=} {\vs Value} \\
{\vs Indicator} \> \gprod \ts{XINT} \galt \ts{KBINDEX} \galt \ts{KBWEIGHT} \galt \ts{COERC}
   \galt \ts{XFG} \galt \ts{HASH} \galt {\vs Ident}\\
\end{tabbing}
\end{minipage}

\noindent
\rule{\textwidth}{0.6pt}
%} % end fbox
\caption{The Syntax of \redux\  Data Types, Part 1}
\label{dtgram1}
\end{center}
\end{figure}

\begin{figure}[htbp]
\begin{center}
\noindent
\rule{\textwidth}{0.6pt}
 
%\fbox{
\begin{minipage}{6in}
\begin{tabbing}
{\vs PrecedenceDecl} \= \gprod \= \kill
{\vs Axiom} \> \gprod \ts{[} {\vs Number} \ts{]} {\vs Term} \ts{==} {\vs Term} \\
\\
{\vs Term} \> \gprod {\vs Var} \gopt{\ts{\_} {\vs Number}} \galt {\vs Const} 
       \galt \ts{(} {\vs Term} \ts{)} \galt {\vs PrefixTerm} \\
  \>\> \galt {\vs PostfixTerm} \galt {\vs InfixTerm} \galt {\vs FunctionTerm} \\
  \>\> \galt {\vs LispTerm} \galt {\vs RoundfixTerm} \galt {\vs CoercionTerm} \\
{\vs PrefixTerm} \> \gprod {\vs PrefixOp} {\vs Term} \\
{\vs PostfixTerm} \> \gprod {\vs Term} {\vs PostfixOp} \\
{\vs InfixTerm} \> \gprod {\vs Term} {\vs InfixOp} {\vs Term} \\
{\vs FunctionTerm} \> \gprod {\vs FuctionOp} \ts{(} \gopt{{\vs Term} 
  \gstar{\ts{,} {\vs Term}}} \ts{)} \\
{\vs RoundfixTerm} \> \gprod {\vs RoundfixOp} {\vs Term} \gstar{\ts{,} {\vs Term}}
   {\vs RoundfixOp} \\
{\vs LispTerm} \> \gprod \ts{(} {\vs LispOp} \gstar{{\vs Term}} \ts{)} \\
{\vs CoercionTerm} \> \gprod {\vs CoercionRoundfixOp} {\vs ExternalObject} {\vs CoercionRoundfixOp} \\
  \>\> \galt {\vs CoercionPrefixOp} {\vs ExternalObject} \\
  \>\> \galt {\vs CoercionFunctionOp} \ts{(} {\vs ExternalObject} \ts{)} \\
  \>\> \galt \ts{(} {\vs CoercionLispOp} {\vs ExternalObject} \ts{)}
\end{tabbing}
\end{minipage}
 
\noindent
\rule{\textwidth}{0.6pt}
%} % end fbox
\caption{The Syntax of ReDuX Data Types, Part 2}
\label{dtgram2}
\end{center}
\end{figure}
\begin{figure}[htbp]
\begin{center}
%\fbox{
\noindent
\rule{\textwidth}{0.6pt}

\begin{minipage}{6in}
\begin{tabbing}
{\vs ClosingRoundfixName} \gsep {\vs DTName} \gsep {\vs OpName} \gsep {\vs Value} \= \kill
{\vs ClosingRoundfixName} \gsep {\vs DTName} \gsep {\vs OpName} \gsep 
   {\vs Value} \> \gprod {\vs XIdent} \\
{\vs AlgoName} \gsep {\vs Domain} \gsep {\vs Range} \gsep {\vs SortName} \> \gprod {\vs Ident} \\
{\vs CoercionOpNames} \gsep {\vs ConstNames} \gsep \\
   {\vs ObjectNames} \gsep
   {\vs OpNames} \gsep {\vs VarNames} \> \gprod {\vs XIdents} \\
{\vs SortNames} \gsep {\vs XSortNames} \> \gprod {\vs Idents} \\
\\
{\vs InfixOp} \gsep {\vs LispOp} \gsep 
   {\vs PostfixOp} \gsep {\vs PrefixOp} \gsep {\vs RoundfixOp} \gsep
   {\vs Var} \= \kill
{\vs CoercionPrefixOp} \gsep {\vs CoercionRoundfixOp} \gsep {\vs Const} \gsep
   {\vs FunctionOp} \gsep \\
   {\vs InfixOp} \gsep {\vs LispOp} \gsep 
   {\vs PostfixOp} \gsep {\vs PrefixOp} \gsep {\vs RoundfixOp} \gsep
   {\vs Var} \> \gprod {\vs XIdent} \\
\\
{\vs XIdents} \= \gprod {\vs XIdent} \gstar{\ts{,} {\vs XIdent}} \\
{\vs Idents} \> \gprod {\vs Ident} \gstar{\ts{,} {\vs Ident}} \\
\\
{\vs Number} \> \gprod \gplus{{\vs Digit}}
\end{tabbing}
\end{minipage}
%} % end fbox

\noindent
\rule{\textwidth}{0.6pt}
\caption{Definition of Auxiliary Grammar Symbols}
\label{axgram}
\end{center}
\end{figure}

%In the current implementation the term-syntax has the
%following limitations that will be removed in a later revision.
%For operator names may start with a special symbol and end with an
%identifier or digit and on the other hand may start with an identifier
%and end with an identifier or digit,
%it is impossible for the parser to uniquely interpret a term
%\verb+(A*B)+ as  either \verb+(A * B)+ or \verb+(A *B ...)+, where
%``{\tt *}'' is an infix operator. Therefore infix operators must be
%separated by blanks from their arguments.
%Roundfix Operators also have to be separated by blanks from their argument
%list. Otherwise \verb+<A,B>+ might be interpreted as \verb+<A,+\ldots
%instead of as \verb+< A,B >+, where ``{\tt <}'' and ``{\tt >}'' form a
%roundfix operator. A later
%revision might restrict operator names to consist either only of
%special symbols or of identifiers and digits with a leading
%identifier.

The following
semantic constraints are given: the argument list of an operator in a term must
be as long as the type list of its domain and the arguments must be of the
appropriate type. Also the name of an operator variable or constant
may not be the same as another operator's, variable's or constant's name.
Furthermore, no left-hand side of a Rule may consist only of a ConstantName
or a VariableName. If a constant is needed as a left-hand side of a
rule, a nullary operator can be used instead.

After reaching the {\tt END} symbol, the programs stop reading the
algebraic specification. This feature can also be used to put
comments at the end of algebraic specification files.

Algebraic specifications are called {\em data types} in the ReDuX system.
The data type describing lists over the elements 0 and 1 can be described as
shown in Figure~\ref{fi:exa}.

\begin{figure}[htbp]
\begin{center}
%\fbox{
\noindent
\rule{\textwidth}{0.6pt}

\begin{minipage}{5.9in} %necessary if there is a footnote within the figure
\begin{quote}
\begin{verbatim}
DATATYPE LIST;
%%  STATUS completed
%%  AUTHOR Reinhard Buendgen
%%  DESCRIPTION List({0,1}): cons, app, rev
%%  ORIGIN  folkore
SORT
        EL, LIST;
CONST
        0, 1: EL;
        nil: LIST;
VAR
        A, B, C: EL;
        L, M, N: LIST;
OPERATOR
        @: LIST, LIST -> LIST;  %%  append 
        [: EL, LIST -> LIST;    %%  cons
        rev: LIST -> LIST;      %%  reverse
NOTATION
        rev: FUNCTION;
        @: INFIX;
        [: ROUNDFIX ];
AXIOM
   [1] nil @ L     == L;
   [2] [A, L] @ M  == [A, L @ M];
   [3] rev(nil)    == nil;
   [4] rev([A, L]) == rev(L) @ [A, nil];
END
\end{verbatim}
\end{quote}
\end{minipage}
%} %end fbox

\noindent
\rule{\textwidth}{0.6pt}
\end{center}
\caption{An example of a ReDuX data type specification} \label{fi:exa}
\end{figure}
The name of the data type is {\tt LIST}, the constants {\tt 0} and {\tt 1}
are of sort {\tt EL} and {\tt NIL} is of sort {\tt LIST}.
There are variables of sort {\tt EL} ({\tt A, B, C}) and of sort {\tt LIST}
({\tt L, M, N}) and operators {\tt APP, CONS} and {\tt REV} with range sorts
{\tt LIST} $\times$ {\tt LIST}, {\tt EL} $\times$ {\tt LIST} and
{\tt LIST} respectively, and result sorts {\tt LIST}.

For further information on input to \redux-programs see 
\RUD.
The user's point of view of the new parser is is described in
\NPUG.

\subsection{The Scanner: The {\tt tpsc}-Module -- Old Parser}

The \redux scanner provides a procedure GETSYM which reads a token from
\index{scanner}
the input.
This token is looked up in the global symbol table\footnote{The global
system variable SYMTAB points to the symbol table used by the SAC-2 symbol 
system.} %end footnote
\index{symbol table!global}
and if needed a new symbol is entered.
The symbol scanned is stored in the global variable SYM.
In GETSYM, the symbols are assigned to different symbol classes which 
are stored as a property under indicator KIND:
\begin{description}
 \item[integer:] unsigned integers (property: INT),
 \item[identifier:] a letter possibly followed by letters or digits
                    (property: IDENT),
 \item[extraordinary identifier:] string of special (non-white-space) 
                    characters (property: XIDENT)
 \item[grammar symbol:] key word and key symbols of the \redux\ interactive
                    language \\ (property: GRSYM).
\end{description}
There are also a lot of procedures to scan symbols of certain classes
(identifiers: SID, integers: SIS), combination of
classes  (SII, SIX, SIIX, SON, SSYM) and special tokens
(SCB, SLS, SNS, SOB, SPS, SSS, SWSP).

The scanner of the new parser is based on {\it flex}-program. 
It is described in \NPPG.

\subsection{The Parser: The {\tt tpprs}-Module -- Old Parser}

The {\tt tpprs}-module provides input procedures for the main data structures
of the \redux\ system:
terms (TPRSC, TEPRS), operators (OPRS), operator sets (OSPRS), constant sets
(CSPRS), axioms (AXPRS), axiom sets (ASPRS) and a  whole data type
(DTPRS).
To parse a single independent term TEPRS should be used.
If several related terms (i.\,e.\ term that share variables) are to be read in
TPRSC must be used.
Among others TPRS has an input parameter that describes a list
of variables that are to be shared.
This list will be updated and output together with the term.
The procedure OROLRD parses relations between a constant or
operator and a list of constants or operators.

The new parser also features the aforementioned interface procedures.
It is based on a {\it yacc}-program that is described in \NPPG.

\subsection{The {\tt tpprs2}-Module}

The module {\tt tpprs2} contains procedures from the old scanner and 
parser that are also used in the new parser.
SCONS enters a symbol into the global symbol table and adds its string as
a property under indicator FLAT to its property list.
It contains consistency checks for associative-commutative operators
(OACC) and commutative operators (OCOMC).
The operation OSUCF is needed to post-process forward references to
coercion operators in the declaration of COERC-properties.
STOCC searches in a list of signature objects (constants, operators, variables,
top-symbols of terms)
for the first element with a given TNAME-value.

\subsection{The {\tt tperr}-Module}

The module {\tt tperr} contains routines to output error messages of the scanner
and the parser.

\subsection{The {\tt tpwrt}-Module -- Old Parser}

The output procedures in the module {\tt tpwrt} write axioms
(ASWRT, AXWRT), data types (DTWRT), rewrite rules (RRWRT),
substitutions (SBWRT) and (related) terms (TLWRT, TDIWRT, TWRT)
such that different variables with common signature (TSIGN) are output
with different print names, adding \_$i$ to the name (TNAME) for the
\index{variable!print name}
$(i+1)^{\mbox{\scriptsize st}}$ different variable with that name.

The output procedures use a data structure called {\em variable dictionary}.
\index{variable!dictionary}
The first component of such a dictionary D contains a list of the form
VDNAME(D) = ($v^*_1,n_1, \ldots, v^*_k,n_k$) which means that so far $n_i-1$
different variables with signature TSIGN($v^*_i$) have been found.
The second component of D is VDINDEX(D)=($v_1,i_1, \ldots, v_m,i_m$)
meaning that $v_j$ is the $i_{j_1}^{\mbox{\scriptsize th}}$ variable with
signature TSIGN($v_j$).

TDICT and TDIWRT update a variable dictionary w.\,r.\,t.\ a term.
VDWTS computes the string output when writing a variable.

If operators are specified as infix-, prefix-, postfix-, roundfix- or
Lisp-operators, they are output accordingly where prefix is the default
notation.
Semi-flat terms are output in a semi-flat manner according to their internal
representation.

External objects which are arguments of coercion operators are written using 
the procedure whose algorithm key is stored under the property DXWA of
the coercion operator.

There are also output procedures used in the context of \redux\ data type
declarations:
for constants (CONWRT, CSWRT), operators (OPWRT, OSWRT), symbols 
(SWRT, SLWRT), variables (VAWRT, VSWRT) and external types
(YXWRT, YXSWRT).

With the new parser a new write module ({\tt write}) must be used.
It offers all of the procedures of {\tt tpwrt}.

