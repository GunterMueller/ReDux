\section{The CO-Subsystem}

The ReDuX subsystem {\em co} contains `coroutines' for combinatorial
\index{coroutine}
\index{combinatorial algorithm}
algorithms.

Because of the complexity of problems like the computation of all subsets of
a set, it may be prohibitive to compute the result at once due to
space limitations.
Therefore the algorithms in {\em co} simulate coroutines which compute 
only one element each time they are called and
`remember' which elements they have already computed.
Hence the actual result is encoded in the status of the coroutine which 
must be passed as input to that routine the next time it is called.

MSNSSC (in {\tt coms}) computes one by one all multi-subsets $M'$ of a
given multi-set $M$ such that $M'$ is $c$ times contained in $M$
\index{multi-set}
(for an integer $c$).
If $e_1,\ldots,e_k$ are pairwise distinct elements, $M$ is encoded as a list
$(e_1, n_1, \ldots, e_k, n_k)$ if element $e_i$ occurs $n_i$ times in $M$.
\begin{example}
 The following piece of \ALDES\ code show how to compute the
 multi-subsets $M'$ of $M$ such that $M'$ is $c$ times contained in $M$.
 \begin{description}
      \item[{(1) [Initialize status of coroutine.]}] \( S:=() \). 
      \item[{(2) [Enumerate multi-subsets.]}] {\bf repeat} \\
        \null [ \( S^* = \{ s_1,s_3, \ldots, s_{|S|-1} \} \) is
         a multi-subset of $M$ which fits $c$ times in $M$
            where \( s_i = \mbox{FIRST}(S_i)  \)
          and \( S = (S_1, \ldots, S_{|S|}) \). ] \\
        MSNSSC$(M,S,c; S,b)$; \\
        \null [ \( b = 1 \) if the length of $S$ has changed. ] \\
       {\bf until} S = (). \hfill $\Box$
 \end{description}
\end{example}

INCGL one by one computes all positive $k$-tuples $C^{(k)}$ whose elements 
stem from a list $C$.
\begin{example} 
 The following piece of  \ALDES\ code show how to compute all
 $k$-tuples in $C \times \cdots \times C$.
 \begin{description} 
      \item[{(1) [Initialize status of coroutine.]}] \( G:=() \);  
      {\bf for} \( i = 1, \ldots, k \) {\bf do} \( G:=\mbox{COMP}(C,G) \). 
      \item[{(2) [Generate $k$-tuples.]}] 
         {\bf while} \(G \neq () \) {\bf do} \\ 
        $\bf \{$ [ \( (g_1,\ldots, g_k) \in C \times \cdots \times C \),
        where $g_i= \mbox{FIRST}(G_i)$ and \( G = (G_1, \ldots, G_k) \).] \\
      \( G:=\mbox{INCGL}(C,G) {\bf \}} \). \hfill $\Box$
 \end{description}
\end{example}

Given $k$ lists $C_1,\ldots,C_k$, GGLINC one by one computes all $k$-tuples 
$(c_1,\ldots,c_k)$ whose elements $c_i$ stem from $C_i$ 
(for $1 \leq i \leq k$).
\begin{example}
 The following piece of  \ALDES\ code show how to compute all 
 $k$-tuples in $C_1 \times \cdots \times C_k$.
 \begin{description}
      \item[{(1) [Initialize status of coroutine such that 
             \( G =(C_1,\ldots,C_k) \).] }]
       \ldots. 
      \item[{(2) [Generate $k$-tuples.]}] {\bf while} \(G \neq () \) {\bf do} \\
        $\bf \{$ [ \( (g_1,\ldots, g_k) \in C_1 \times \cdots \times C_k \),
        where $g_i= \mbox{FIRST}(G_i)$ and \( G = (G_1, \ldots G_k) \).] \\
        \( G:=\mbox{GGLINC}(C,G) {\bf \}} \). \hfill $\Box$
 \end{description}
\end{example}
