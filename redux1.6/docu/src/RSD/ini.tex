\section{ReDuX System Initalization: The INI-Subsystem}
\label{se:ini} %\label{ss:tpbgn}

The \redux\ subsystem {\em ini} contains procedures needed to
initialize the \redux\ system.

The system initialization routine BGNRWD must be called at the beginning of
\index{system initialization}
each ReDuX program.
Depending on its argument it calls the initialization routines 
of other SAC-2 subsystems
\index{SAC-2}
including at least the basic- ({\tt bs}), the list processing- ({\tt lp})
and the symbol processing systems ({\tt sy}).
Optionally the SAC-2 arithmetic-, - polynomial-, - polynomial factorization- 
and the \redux\ interpreter systems (see Section~\ref{se:it})
can be initialized (ISTINI).

BGNRWD also opens a file from which the \redux\ program expects to read
a data type.
The default name of this file is {\it program-name}{\tt .in}.
It may be overwritten by the first command line argument.
In addition BGNRWD opens a file on which the current session is recorded.
The default name of this file is {\it program-name}{\tt .log}.
It may be overwritten by the last command line argument.
Note that if the interpreter system is initialized by BGNRWD
then a file is opened that contains symbol table entries for each algorithm 
to be callable by the interpreter (cf.\ Section~\ref{se:it}).
By default this file is called {\it program-name}{\tt .algos} unless it
is overwitten by the second command line argument.

Then BGNRWD initializes the global symbol table of grammar symbols 
\index{symbol table, global}
\index{symbol!grammar}
(see Table~\ref{tb:sym}).
The interpreter initialization also defines a new global vaiable DAK.
With the new parser it initializes further variables calling the routine
ITGLPR (see \NPPG).

\begin{table}
\begin{center}
\begin{tabular}{|l|l|l|l|}
 \hline
 symbol name & print name & kind & meaning\\
 \hline\hline
  DAC        & ``AC''     & grammar symbol & AC-unification status \\
  %DAK        &            & indicator      & algorithm key (interpreter) \\
  DAS        & ``AXIOMS'' & grammar symbol & axiom symbol \\
  DAV        & ``ARGVAR'' & indicator      & variable of argument type \\
  DCB        & ``)''      & grammar symbol & closing bracket \\
  DCM        &            & grammar symbol & comment symbol \\
  DCO        & ``COM''    &  grammar symbol & commutative-unification status \\
  DCR        & ``ROUND''  & indicator      & $2^{nd}$ roundfix 
                                             operator symbol \\
  DCS        & ``CONST''  & grammar symbol & constant symbol \\
  DEF        & ``PREFIX'' & property       & operator notation: prefix \\
  DES        & ``END''    & grammar symbol & end symbol \\
  DFS        & ``FIX''    & indicator      & operator notation \\
  DIF        & ``INFIX''  & property       & operator notation: infix \\
  DIS        & ``POLY''   & indicator      & polynomial interpretation \\
  DLF        & ``LISP''   & property       & operator notation: Lisp \\
  DLS        & ``--''     & grammar symbol & link symbol \\ 
  DNS        & ``:''      & grammar symbol & enumeration symbol \\  
  DOB        & ``(''      & grammar symbol & opening bracket \\ 
  DOS        & ``OPS''    & grammar symbol & operator symbol \\  
  DPF        & ``POSTFIX'' & property       & operator notation: postfix \\
  DPS        & ``==''     & grammar symbol & production symbol \\   
  DRF        & ``ROUNDFIX'' & property       & operator notation: roundfix \\
  DRS        & ``,''      & grammar symbol & repeater symbol \\    
  DSH        & ``HASH''   & indicator      & suitable for hashing \\
  DSS        & ``.''      & grammar symbol & separator symbol \\     
  DST        & ``CSBTERM'' & indicator     & constrained subterms \\
  DTS        & ``TYPE''   & grammar symbol & type symbol \\      
  DUS        & ``USTAT''  & indicator      & unification status \\
  DVS        & ``VARS''   & grammar symbol & variable symbol \\      
  %DXDA       & ``XDEC''   & indicator      & ext.\ type decomposition \\
  DXCN       & ``XCONSTRUCTORS'' & indicator & ext.\ type constructors \\
  DXCO       & ``COERC''  & indicator      & ext.\ type coercion operator \\
  DXFG       & ``XFG''    & indicator      & ext.\ type free generator algo. \\
  DXIA       & ``XINT''   & indicator      & ext.\ type interpretation \\
  DXLA       & ``XLT''    & indicator      & ext.\ type less-than algo. \\
  DXQA       & ``XEQ''    & indicator      & ext.\ type equal algorithm \\
  DXRA       & ``XREAD''  & indicator      & ext.\ type read algorithm \\
  DXTA       & ``XTERM''  & indicator      & term from ext.\ object algo. \\
  DXWA       & ``XWRITE'' & indicator      & ext.\ type write algorithm \\
  DXR        & ``EXT.RULE'' & indicator    & extension rule \\
  DXT        & ``EXTERNAL'' & grammar symbol & external term symbol \\
  KIND       &            & indicator      & symbol kind \\
  IDENT      &            & property       & kind: identifier \\
  XIDENT     &            & property       & kind: extraordinary identifier \\
  INT        &            & property       & kind: integer \\
  GRSYM      &            & property       & kind: grammar symbol \\
  FLAT       &            & indicator      & external name \\
  \hline
\end{tabular}
\caption{Global symbols initialized by BGNRWD} \label{tb:sym}
\end{center}
\end{table}

The routine SCONS called by BGNRWD constructs a symbol and stores its
string under indicator FLAT.