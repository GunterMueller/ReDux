\section{The ReDuX Indices}

The ReDuX indices consist of four parts. 
First there are copies of all ReDuX header files
followed by an overview of the ReDuX module architecture.
Then there is one index for each subsystem
that shows in which module each algorithm is located and last but not least
there is KWIC-index for all algorithms in the ReDuX system.
The KWIC-index is a permuted index which contains algorithm names followed 
by the expansions of the abbreviated name.
The rightmost column consists of a two letter abbreviation of the ReDuX
subsystem the algorithm is in.


The ReDuX system consists of the following eleven subsystems:
\index{subsystem}
\begin{quote}
\begin{tabbing}
 {\bf oo} \underline{co}mbinatorial algorithmsxxxxxxxxxxxxxx, \= \kill
 {\bf io} \underline{i}nput-\underline{o}utput primitives, \>
 {\bf ax} \underline{a}u\underline{x}iliary algorithms, \\
 {\bf ini} ReDuX system \underline{ini}tialization, \>
 {\bf tio} \underline{t}erm \underline{i}nput-\underline{o}utput \\
 {\bf it} \underline{i}n\underline{t}erpreter \>
 {\bf tp} \underline{t}erm \underline{p}rimitives, \\
 {\bf co} \underline{co}mbinatorial algorithms, \>
 {\bf to} \underline{t}erm \underline{o}rderings, \\
 {\bf tc} \underline{t}erm \underline{c}ompletion, \>
 {\bf ic} \underline{i}nductive \underline{c}ompletion, \\
 {\bf ev} \underline{ev}aluation domains, \>
 {\bf ac}({\bf n}) \underline{AC} completion (\underline{n}ew), \\
 {\bf rd} \underline{r}andom term generation, \>
 {\bf uc} \underline{u}nfailing \underline{c}ompletion \\
\end{tabbing}
\end{quote}
