\section{The Random Term Generation Program}
The \redux\ random term generation demo program {\tt trd} is used to
experiment with the random term generation algorithms, to produce files
with random terms and to make statistics on the algorithms. The random
term generator is based on the SAC--2 random number generator {\tt
IRAND} and algorithms which establish two bijective mappings from the
natural numbers to the set of terms over a given signature:
\begin{itemize}
\item For the first mapping algorithm ({\tt NT2TE}), the function
   \(n \rightarrow |O(\mbox{\tt NT2TE}(n))|\) is monotonous,
   i.\,e.\ the terms are sorted according to their extents.
\item For the other mapping algorithm ({\tt NT2TD}) the function
   \(n \rightarrow \mbox{DEPTH}(\mbox{\tt NT2TD}(n))\) is monotonous,
   i.\,e. the terms are sorted according to their depths.
\end{itemize}

The program uses the same standard files ({\tt trd.in}, {\tt trd.log}
and {\tt algos}) as the other \redux\ demo programs. Additionally, the
file {\tt file.1} is used if the global data arrays are written to a
file (using menu option {\tt w}), {\tt file.2} is used to store
generated random terms and {\tt file.3} is used to hold the output of
the statistics algorithms.

\subsection{Program Start}
Since it can use data types with external operators, the random term
generation program must first read the file {\tt algos} which holds data for
the \redux\ interpreter module.  Subsequently, it reads a \redux\ data type
which is presented to the user.

When the user strikes the $\langle$return$\rangle$ key to proceed, the program
may issue messages concerning the unification state of some
operators:
\begin{quote}
\begin{verbatim}
Operator +: Unification state AC is ignored!
Operator *: Unification state COM is ignored!
\end{verbatim}
\end{quote}
The program is able to generate terms over signatures containing AC--
or commutative operators, but it ignores those properties.  This means
that two terms which are equivalent with respect to an unification
state are considered to be distinct by the program. This changes
the distribution of terms in a generated term sequence.

If the data type contains external operators, the program will
display messages like the following:
\begin{quote}
\begin{verbatim}
External operator " is treated like a constant (with fixed TCONT-field)!
Please enter the value of the TCONT-field in a format suitable for IREAD. *
\end{verbatim}
\end{quote}
The random term program can work with data types containing external
operators in a restricted fashion, i.\,e., it can generate terms with
external operators, but each of their {\tt TCONT}--fields holds a
constant value which must be entered by the user.  This value is read
by a function which is determined in the data type via the {\tt XREAD}
specifier in the {\tt EXTERNAL} part of a \redux\ data type
specification.

\subsection{The Main Menu}
After those program initializations, the main menu is displayed:
\begin{quote}
\begin{verbatim}
You have following choices: 
  s - show signature descriptor
  w - write global data
  i - initialize global data
  e - enumerate terms
  r - generate random terms
  R - generate random terms, using TRANDE or TRANDD
  S - statistics
  t - toggle term generation trace
  g - toggle garbage collection messages
  o - determine output mode
  h - help, print menu
  q - quit
 
[s/w/i/e/r/R/S/t/g/o/h/q]  * 
\end{verbatim}
\end{quote}
For most options it does not matter if letters are entered in upper or
lower case.  Exceptions are {\tt r} and {\tt s}.

\begin{itemize}
\item{\bf Show signature descriptor (s)}\\
   The signature of the data type is shown. The printed values
   include the signature name, a list of all types (sorts), a list of
   all domains (the Cartesian product is denoted by an {\tt x}), a list
   of operators sorted according to their return types and a list of all
   atomic terms (constants, variables, nullary operators and external
   operators (with fixed {\tt TCONT}--field)) sorted according to their
   type.

\item{\bf Write global data (w)}\\
   The algorithms which map natural numbers to terms use several data
   arrays. This menu item allows to show these arrays. They can either
   be displayed on the console or written to {\tt file.1}. Some
   knowledge of the involved algorithms is needed to benefit from the
   output. It is mainly for debugging purposes.

\item{\bf Initialize global data (i)}\\
   This menu item allows to initialize the aforementioned data arrays
   manually. First, the user must choose which arrays shall be
   initialized, either those for generation of terms sorted according
   to their extent or those for generation of terms sorted according to
   their depth.  Then the extent (depth) up to which the initialization
   should take place has to be entered. All algorithms in this program
   initialize the arrays automatically, so there is no real need to
   select this menu item.  It can be used to compare the initialization
   time necessary for different data types.

Every time the program enhances the global data, it issues a message like
\begin{quote}
\begin{verbatim}
Initialized global data from 4 positions to 24 positions
in 0 ms (+ 0 ms in GC).
\end{verbatim}
\end{quote}
(In this case, the initialization took less than one clock tick.)

\item{\bf Enumerate terms (e)}\\
   If this item is selected, the program maps a sequence of natural
   numbers to terms which are displayed.  Again, the user must choose
   which of the two mapping algorithms should be used.  Then the first
   and the last number in the sequence must be specified.  The
   generated terms are written to the console, each one on a new line.

\item{\bf Generate random terms (r)}\\
   This option allows to generate a sequence of random terms.  Once
   more the user has to enter which of the two mapping algorithms
   should be used. Then the number of terms to generate, the maximal
   extent (depth) per term and if the term sequence should be displayed
   on the console or written to {\tt file.2} have to be specified.
   The terms will be separated by lines containing a single character.
   Therefore, the program asks for this character. If the terms are to
   be read by another \redux\ program, a semicolon must be chosen.

   Since there are more terms with large extent (depth) than such
   with smaller ones, the probability to get terms with small extents
   (depths) is negligibly small.

\item{\bf Generate random terms, using TRANDE or TRANDD (R)}\\
   This menu item performs the same task as the previous one. The
   difference is that here the high-level algorithms {\tt TRANDE} and
   {\tt TRANDD} are used, whereas in the former item the algorithms
   {\tt NT2TE} and {\tt NT2TD} are called directly. The direct call
   leads to a slightly better performance, so usually the previous menu
   item should be used. This item is intended to compare the
   performance of both algorithms.

\item{\bf Statistics (S)}\\
   This menu item is a facility to make some statistics on the random
   term generation algorithms. There are statistics on the term
   generation time, the distribution of term depths and the
   distribution of term extents.  See
   Section~\ref{StatsOnRandomTermGenerator} for more information on
   this topic.

\item{\bf Toggle term generation trace (t)}\\
   The term generation algorithms can print some trace information.
   With this option, the user can turn on and off these messages. Note
   that the trace output is quite voluminous. Further information on
   the term generation trace can be found in
   Appendix~\ref{TermGenerationTrace}.

\item{\bf Toggle garbage collection messages (g)}\\
   This option allows to turn on and off the messages which are
   displayed by the garbage collector.

\item{\bf Determine output mode (o)}\\
   This menu item allows the user to change the term write mode. The
   term write mode is described by 10 flags. They are explained in {\em
   A New Parser for \redux\  --- User Guide} in the section about the
   parser demo program.

\item{\bf Help, print menu (h)}\\
   By selecting this menu item, the user can display the menu. This may
   be useful if the short form of the menu is displayed.

\item{\bf Quit (q)}\\
   This terminates the program.
\end{itemize}

\subsection{Statistics on the Random Term Generator}
\label{StatsOnRandomTermGenerator}
If the statistics option has been selected in the main menu, the
programs shows another menu with the different possibilities to make
statistics:
\begin{quote}
\begin{verbatim}
You have the following choices:
 
  Concerning terms distributed according to extent:
  t - Statistics on generation time
  d - Statistics on term depth
  e - Statistics on term extent
 
  Concerning terms distributed according to depth:
  T - Statistics on generation time
  D - Statistics on term depth
  E - Statistics on term extent
 
  x - Exit, return to main menu
  *
\end{verbatim}
\end{quote}
There are the same statistics for both mapping algorithms.  In either
case, there is the possibility to write a copy of the results to {\tt
file.3} in GNUPLOT--readable format.

\subsubsection{Statistics on Term Generation Time}
First the user must enter a range of extents (depths) and the number of
terms to test.  For each extent (depth) in the given range, the program
generates the specified number of terms.  Note that in spite of the
fact that the generated terms can have any extent (depth) from one
(zero) up to the specified upper border, most of the generated terms
will have an extent (depth) near to the upper border. (The distribution
of the term extents (depths) can also be examined with the statistics
part of the program, look below for more information.)

Then the program prints a table of the results to the console. The
first column contains the term extent (depth), the second column the
CPU--time needed for the creation of the terms with that maximal
extent (depth).

\subsubsection{Statistics on Term Depth}
Here the program will generate a number of terms with a fixed maximal
value for the term extent (depth).  Then the depth of each term is
determined and a table of the distribution of term depths is shown. Its
first column contains term depths, the second column contains the
number of generated terms with that depth.

Note that the definition of term depth realized by the \redux\ algorithm
{\tt DEPTH} is used, i.\,e., variables and constants have depth 0.

\subsubsection{Statistics on Term Extent}
If one of these menu items has been chosen, the program will generate
a number of terms with a fixed maximal value for the term extent
(depth).  Then the extent of each generated term is determined and a
table of the distribution of term extents is shown. Its first
column contains term extents, the second column contains the number of
generated terms with that extent.

If the second mapping algorithm is chosen (menu item {\tt E}), the term
extent can become quite large even for small term depths.  The program
therefore must store the number of terms of each extent for large
extents.  To avoid undesired large tables, the maximal
term extent which will be considered separately must be specified. The
numbers of terms with larger extents are not stored separately but are
added.

\subsubsection{Displaying Graphs of Statistics Using GNUPLOT}
If the user chooses to create a file with the results which were gained
in the statistics part of the program, it can be used to display a
graph using GNUPLOT. The file {\tt file.3} is read by GNUPLOT using the
statement {\tt load "file.3"} from within GNUPLOT.  GNUPLOT will
display the graph on the default device, usually a window on the
console. Other output devices can be selected with the {\tt set output}
and {\tt set terminal} statements. Note that the position of the {\tt
ylabel}-string depends on the output device. The random term program
produces output files in which the {\tt ylabel}-string is shifted
about ten characters downwards form the default position. This is
suitable for most terminals. If a graphic should be printed on a
PostScript printer, the default position is better suited, i.\,e., the
position in the {\tt set ylabel} command should be deleted. If it should
be output in \LaTeX-format, the y--coordinate should be set to zero and
the x-coordinate to a small negative value. Additionally, line break
commands (\verb|\\|) should be inserted in the {\tt ylabel}-string.

Due to the fact that a GNUPLOT--file usually cannot hold both commands
and data, in {\tt file.3} the data lines are marked as comments with a
{\tt \#}--sign. The plot statement (in the last lines of {\tt file.3})
filters the file through {\tt sed} to get rid of all non-data lines and
comment characters. If {\tt file.3} is renamed, the argument {\tt
file.3} in the plot statement has to be replaced by the new name.

If the statistics routines are called more than once during one program
run, all output is collected in one file. GNUPLOT cannot handle such
files correctly, so they should be splitted.

