\section{The EV-Rewrite Laboratory}

The EV-rewrite laboratory is a modification of the AC-rewrite laboratory
that allows to support normalizations by partial evaluation domains and
memorization.

If a rewrite specification $\cal S$  contains a complete subspecification 
$\cal S'$  a free algebra $\cal A'$ in the set of models of $\cal S'$ 
(on a countable set of generators) is a \emph{partial evaluation domain} 
for $\cal S$. 
Built-in partial evaluation domains may be used to speed up the normalization 
process.
For more background on partial evaluation domains see
\cite{BuendgenLauterbach:96a}.

It is also possible to memorize the normal form of  term.
This memorized normal form may then be retrieved instead of computing it
a second time.
ReDuX allows to memorize  normal forms of all ground terms with 
specified top operators.

\subsection{Specifications}

In order to specify a partial evaluation domain an external sort must be 
declared with a property denoting a function to compute the $i$-th generator
of the evaluation domain (indicator \texttt{XFG}) and a property denoting
a function that translates an external object from the evaluation domain
back into a term (indicator \texttt{XTERM}).
Each constant or operator which can be interpreted in an external sort must 
be assigned the interpreting function name as property under an indicator that
has the name of the respective external sort.
Figure~\ref{fi:ed} shows an example specification for a signature with two
partial evaluation domains: functional decision diagrams 
to model Boolean rings and an implementation of a free group.
\begin{figure}[htbp]
\begin{center}
\noindent
\rule{\textwidth}{0.6pt}

\begin{verbatim}
DATATYPE BOOLGROUP;
SORT
        Q4, BOOLEAN;
CONST
        1, a, b: Q4;
        F, T:    BOOLEAN;
OPERATOR
        *: Q4, Q4 -> Q4;
        I: Q4     -> Q4;
        h: Q4     -> Q4;

        #:  BOOLEAN, BOOLEAN -> BOOLEAN;
        /\: BOOLEAN, BOOLEAN -> BOOLEAN;
        ~:  BOOLEAN          -> BOOLEAN;
        =>: BOOLEAN, BOOLEAN -> BOOLEAN;
        A, B, C, D:          -> BOOLEAN;

        ?: Q4 -> BOOLEAN;
THEORY
        #, /\: AC;
PREC
        # < /\;
NOTATION
        *, I, ?: FUNCTION;
        ~: PREFIX;
EXTERNAL
        FGROUP: XTERM=TFFGE, XFG=FGFG;
        FDD:    XFG=FDDFG, XTERM=TFFDDE;
PROPERTY
        I: FGROUP=FGINV;
        *: FGROUP=FGPROD;
        1: FGROUP=FGIDNT;

        F:  FDD=FDDZER;
        T:  FDD=FDDONE;
        #:  FDD=FDDXOR;
        /\: FDD=FDDAND;
        ~:  FDD=FDDNOT;
VAR
        ...
AXIOM 
        ...
END
\end{verbatim}
 
\noindent
\rule{\textwidth}{0.6pt}
%} % end fbox
\caption{Specification with Evaluation Domains}
\label{fi:ed}
\end{center}
\end{figure}

The following evaluation domains are prepared in the ReDuX system.
The tables show the functions needed in the properties of the external sorts
and for the interpretation of constants and/or operators.
Note that constants and/or operators must be declared in the same 
order as in the following tables.
\begin{description}
\item[free groups] \null

\begin{tabular}[t]{l|l|l}
\hline
\texttt{XFG} & \multicolumn{2}{l}{\texttt{FGFG}} \\
\texttt{XTERM} & \multicolumn{2}{l}{\texttt{TFFGE}} \\
\hline
\hline
operator & arity & interpretation \\
\hline
neutral element & 0 & \texttt{FGIDNT} \\
group operation & 2 & \texttt{FGPROD} \\
inversion & 1 & \texttt{FGINV}\\
\hline
\end{tabular}
\item[Abelian groups]  \null

\begin{tabular}[t]{l|l|l|l}
\hline
\texttt{XFG} & \multicolumn{3}{l}{\texttt{AGFG}} \\
\texttt{XTERM} & \multicolumn{3}{l}{\texttt{TFAGE}} \\
\hline
\hline
operator & arity & interpretation & remark \\
\hline
neutral element & 0 & \texttt{AGZERO} \\
group operation & 2 & \texttt{AGSUM} \\
inversion & 1 & \texttt{AGNEG}\\
difference & 2 & \texttt{AGDIF} & optional \\
\hline
\end{tabular}
\item[commutative rings] \null

\begin{tabular}[t]{l|l|l}
\hline
\texttt{XFG} & \multicolumn{2}{l}{\texttt{IPRFG}} \\
\texttt{XTERM} & \multicolumn{2}{l}{\texttt{TFIPRE}} \\
\hline
\hline
operator & arity & interpretation \\
\hline
zero & 0 & \texttt{PRZERO} \\
one & 0 & \texttt{IPRONE} \\
addition & 2 & \texttt{IPRSUM} \\
multiplication & 2 & \texttt{IPRPRO} \\
additive inversion & 1 & \texttt{IPRNEG} \\
\hline
\end{tabular}
\item[Boolean rings] There are two alternatives: Boolean polynomials 

\begin{tabular}[t]{l|l|l|l}
\hline
\texttt{XFG} & \multicolumn{3}{l}{\texttt{IPRFG}} \\
\texttt{XTERM} & \multicolumn{3}{l}{\texttt{TFIPRE}} \\
\hline
\hline
operator & arity & interpretation & remark \\
\hline
zero & 0 & \texttt{PRZERO} \\
one & 0 & \texttt{IPRONE} \\
addition & 2 & \texttt{BPRSUM} \\
multiplication & 2 & \texttt{BPRPRO} \\
additive inversion & 1 & \texttt{BPRNEG} & optional\\
\hline
\end{tabular}

or functional decision diagrams (FDDs)\footnote{FDDs are not available
with the default installation. You must modify the Makefiles in the
\texttt{mak} and \texttt{src/it/PRESRC} if you want to use them.}

\begin{tabular}[t]{l|l|l|l}
\hline
\texttt{XFG} & \multicolumn{3}{l}{\texttt{FDDFG}} \\
\texttt{XTERM} & \multicolumn{3}{l}{\texttt{TFFDDE}} \\
\hline
\hline
operator & arity & interpretation & remark \\
\hline
zero & 0 & \texttt{FDDZER} \\
one & 0 & \texttt{FDDONE} \\
addition & 2 & \texttt{FDDXOR} \\
multiplication & 2 & \texttt{FDDAND} \\
additive inversion & 1 & \texttt{FDDNEG} & optional\\
not & 1 & \texttt{FDDNOT} & optional \\
or & 2 & \texttt{FDDOR} & optional \\
\hline
\end{tabular}
\end{description}

In order to specify that a the normal form of a ground term with a certain
top operator shall be memorized that operator must have a property 
\texttt{yes} under indicator \texttt{HASH}.
The following declaration indicates that the normal forms of ground terms 
with top operators \texttt{+} and \texttt{*} are worth memorizing.
\begin{verbatim}
 PROPERTY
   +, *: HASH = yes;
\end{verbatim}

\subsection{Program Start}
The EV-program does not automatically compute the AC-extension rules of the 
data type at the beginning of the program.
These extensions are only available after initializing the
normalization procedures.

\subsection{The Main Menu}

The program displays the following main menu.
\begin{verbatim}
You have the following choices:
Syntactic relations among terms
---------------------------------------------------------
  e/m/u - test equality of / match / unify two terms
Term orderings and termination
---------------------------------------------------------
  i - install (deinstall) term orderings
  s - select term ordering
  o/O - order axioms / of current data type
Proof w.r.t. current TRS (assumed to be complete)
---------------------------------------------------------
  N - select normalization procedure
  n - normalize a term
  p - prove equational theorem
Deduction / completion
---------------------------------------------------------
  c - compute critical pairs
  C - complete current set of axioms
Output/Options/Statistics
---------------------------------------------------------
  w/W - write current data type / to a file
  r - reset times and counters
  d - display times and counters
  t - set/unset trace options
Control menu
---------------------------------------------------------
  h/? - help, print this menu
  q - quit
Go on [e|m|u|i|s|o|O|N|n|p|c|C|w|W|r|d|t|h|q]? *
\end{verbatim}
Even though the menu format has changed compared with the AC-laboratory
only the menu items ``select normalization procedure (N)'',
``complete current set of axioms (C)'',
``write current data type / to a file (w/W)'',
``display times and counters (d)'' and
``set/unset trace options (t)'' have changed or are new.
\begin{itemize}
\item \textbf{select normalization procedure (N)}

 Before using any normalization procedure --- directly using the item 
 ``normalize a term (n)''
 or indirectly using items ``prove equational theorem (p)'' or 
 ``compute critical pairs (c)'' --- the normalization should be initialized 
 by that option. 
 Without an initialization of the normalization standard innermost 
 normalization is used. Yet no AC-extension rules will be available
 at this point. 
 See Section~\ref{se:evN} for details on initializing the normalization
 procedure.
\item \textbf{complete current set of axioms (C)}

 This option run the Peterson-Stickel completion procedure for
 commutative and associative-commutative theories.
 This procedure may take advantage of normalizations supported by
 partial evaluation domains.
 As usual orderings must be installed and selected before choosing
 item ``C'' if automated orderings are to be used.
 For more details on running the completion process see  
 Sections~\ref{AC-KBCompletion} and \ref{se:evC}.
\item \textbf{write current data type / to a file (w/W)}

 Item ``w'' writes the current data type to the standard output.
 This option was formerly called ``d''.
 Item ``W'' allows to write  the current data type to file that is requested
 from the user
\begin{verbatim}
   Input filename! *
\end{verbatim}
\item \textbf{display times and counters (d)}

 Displays the timers and counters. This item was formerly called ``S''.
\item \textbf{set/unset trace options (t)}

 A new trace option for evaluation traces (see Appendix~\ref{Traces})
 is available in addition to those
 available in the AC-laboratory.

\end{itemize}

\subsection{Initializing the Normalization Procedure}
\label{se:evN}

Before using the normalization procedure it must be initialized.
This initialization allows the user to choose among different normalization
procedures:
\begin{verbatim}
Current normalization strategy: pure (inner-most) normalization
Choose a new normalization strategy
 n - pure (inner-most) normalization
 e - normalization with evaluation support
 m - normalization with memorization
 q - quit
Enter you choice [n|e|m|q] (verbose [N|E|M]) *
\end{verbatim}
Option ``n'' uses the standard inner-most normalization strategy used
by all other ReDuX programs.

Option ``e'' allows support by the evaluation domains specified by the data
type.
If this option is chosen the user must enter the rules which become redundant
using the built-in evaluation domain.
\begin{verbatim}
Enter the list of rules that are built-in by evaluation domains! *
\end{verbatim}
The expected answer is a list of axiom numbers where ranges may be
specified using hyphens
(e.\,g., \texttt{(2,4,10-13,8)}).
The user must make sure that the specified rules are really subsumed by the
evaluation domain.
There is no means to automatically check this.

Option ``m'' allows to memorize the normal forms of ground term with top
operators that have the \texttt{HASH} property specified.
The memorization technique may be combined with using evaluation domains
and therefore  the user must enter the rules which become redundant
using the built-in evaluation domain. 
If no such support is needed the empty list \texttt{()} should be entered.

After the choice of the normalization procedure the AC-extension rules 
are computed.
If verbose mode was selected (``N'', ``E'' or ``M'' respectively)
then the set of non-built-in rules is displayed.

\subsection{Running the Peterson-Stickel Procedure}
\label{se:evC}

The completion process of the EV-laboratory it the same as the one of the 
AC-laboratory (see Section~\ref{AC-KBCompletion}).
The only difference is that the user may choose support the normalizations
during the completion by evaluation domains if a subset of the rules
is complete and equivalent to the evaluation domains specified in the
data type.
\begin{verbatim}
Do you want evaluation support [y|n] ? *
\end{verbatim}
If this question is answered positively the rules equivalent to the evaluation
domains must be entered as a list of rule numbers.
\begin{verbatim}
The following set of convergent rules should be
subsumed by the evaluation process!
Is a subset of the rules in the data type known to be canonical [y/n]?  *
y
Enter list of rule numbers of canonical rule set! *
\end{verbatim}
Only the rules specified here will be considered redundant in the normalization
process.
The remaining completion works as described in Section~\ref{AC-KBCompletion}.