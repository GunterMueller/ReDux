\section{Ground Normal Form Analysis}
The {\tt ts} demo program allows to analyze the set of ground normal 
forms of an algebraic specification described by a term rewriting system.
It works for left-linear term rewriting systems and
{\em well-behaved} non-left-linear systems.
In case of a method failure a warning is output.
The program further assumes that for each sort occurring in the data
type at least one variable is declared.

\subsection{The Main Menu}
After reading an algebraic specification, the program shows the main
menu:

\begin{quote}
\begin{verbatim}
You have the following choices: 
 
  l - compute a top set tree (by levels),
  p - compute a pruned top set tree,
  i - interleaved top set tree computation,
  d - display data type,
  s - show times and counters,
  r - reset times and counters,
  m - reclaim memory,
  t - set/unset trace options
  h - help, print menu
  q - quit.
[l/p/i/d/s/r/m/t/h/q] *
\end{verbatim}
\end{quote}

\begin{itemize}
\item{\bf Compute Top Set Tree (by Levels) (l)}\\
This allows the computation of top set trees.  See Section~\ref{TopSetTrees}.

\item{\bf Compute Pruned Top Set Tree (p)}\\
This algorithm performs the same tasks as the previous one, except that 
it computes pruned top set trees and pruned top sets described in
\cite{Buendgen:91b}, \cite{Eckhardt:91}, \cite{BuendgenEckhardt:92}.

\item{\bf Interleaved Top Set Tree Computation (i)}\\
This menu item allows the computation of top set trees in an interleaved
way.  See Section~\ref{InterleavedTopSetTrees}.

\item{\bf Display Data Type (d)}\\
The program displays the current data type.

\item{\bf Show Times and Counters (s)}\\
The counters (number of matches, created top set nodes, reductions 
and unifications) and
the timers (cpu-time used so far, cpu-time spent for ground reduction
tests, matching, reduction and unification) are displayed.
Furthermore, the number of available cells in {\tt SPACE} is shown.

\item{\bf Reset Times and Counters (r)}\\
The counters and timers are reset, either altogether or selectively.

\item{\bf Reclaim Memory (m)}\\
The number of available cells in {\tt SPACE} is displayed. Then
the garbage collector is started. The garbage collector prints the
number of available cells after execution.

\item{\bf Set/Unset Trace Options (t)}\\
Five traces can be selected, namely the counter example trace, the
substitution trace, the procedure name trace, the tree extension trace 
and the node deletion trace. A description of these traces is given in
Appendix~\ref{Traces}.

\item{\bf Help, Print Menu (h)}\\
This item prints the main menu. It can be useful if the program
displays the short form  of the main menu.

\item{\bf Quit (q)}\\
The program can be stopped by selecting this menu item.
\end{itemize}

\subsection{Computing (Pruned) Top Set Trees}
\label{TopSetTrees}%
If the menu item ``compute top set tree (by levels)'' has been 
selected, the program computes \(d\)-top sets $GR_{\cal R}(d)$, top set 
trees and ground normal form grammars (GNF grammars) for
the the given algebra according to \cite{Buendgen:91},
\cite{BuendgenEckhardt:92}.

The user must enter the deepest level of the top set tree.
Usually the depth of the
deepest left-hand side in the set of rewrite rules is be  chosen.
If the chosen level is too small, it is possible that a node of the computed
top set tree is constrained.

If the top set tree is unconstrained, the program prints the computed 
top set 

\begin{quote}
\begin{verbatim}
Topset: 
0 / OPP(SUCC(0)) / OPP(SUCC(SUCC(A))) / SUCC(SUCC(0)) /
                                              SUCC(SUCC(SUCC(A))) / SUCC(0) /
Top set size is 6.
\end{verbatim}
\end{quote}

and the top set tree.
For each node the corresponding label is printed, starting with the root
at the left margin.
The  child of a node is shifted two positions to the right-hand side.

\noindent
{\bf Example:} (In this example, the data type {\tt spec/canopp.rdx} is used.)
\begin{quote}
\begin{verbatim}
Prefix Traversal: 
A
  SUCC(A)
    SUCC(0)
    SUCC(SUCC(A))
      SUCC(SUCC(SUCC(A)))
      SUCC(SUCC(0))
  OPP(A)
    OPP(SUCC(A))
      OPP(SUCC(SUCC(A)))
      OPP(SUCC(0))
  0
\end{verbatim}
\end{quote}

Then the GNF grammar is output by the program:
\begin{quote} 
\begin{verbatim}
GNF grammar for data type OP
<OP> ::= <A>
<SUCC(A)> ::= SUCC(<SUCC(A)>)  |
     SUCC(0)
<A> ::= 0  |
     OPP(<SUCC(A)>)  |
     SUCC(<SUCC(A)>)  |
     SUCC(0)
\end{verbatim}
\end{quote}

In the grammar, the usual meta symbols are used, {\tt ::=} is the
production symbol, nonterminal symbols are enclosed in \verb+<+ \verb+>+ and 
alternatives are separated by \verb+|+.

If the top set tree is constrained, the program prints the labels of
the constrained nodes:
\begin{quote}
\begin{verbatim}
OPP(A) is constrained!
\end{verbatim}
\end{quote}
Then it prints the top set tree. The output of a GNF grammar is omitted.

\subsection{Interleaved Top Set Trees Computation}
\label{InterleavedTopSetTrees}
The procedure for interleaved top set trees partitions the signature of an 
algebraic
specification into a set of {\em free constructors} and {\em defined function
symbols} according to the algebra separation algorithm in 
\cite{BuendgenKuechlin:89}, \cite{Buendgen:91b}, \cite{Eckhardt:91},
\cite{BuendgenEckhardt:92}.

\noindent
{\bf Example:}
\begin{quote}
\begin{verbatim}
Data type composition:
constructors: 
0: OP
SUCC: OP -> OP
OPP: OP -> OP
defined function symbols: 
+: OP, OP -> OP
\end{verbatim}
\end{quote}

Then it computes the ground reduction test set of the given algebra.
\begin{quote}
\begin{verbatim}
ground reduction test set: 
SUCC(0)
SUCC(SUCC(SUCC(A)))
SUCC(SUCC(0))
OPP(SUCC(SUCC(A)))
OPP(SUCC(0))
0
\end{verbatim}
\end{quote}

Subsequently it prints the top set tree and the GNF grammar.
