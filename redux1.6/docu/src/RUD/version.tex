\section{Version Log}

In this  section we give a brief summary of the main changes
of the \redux\ programs from a user's point of view.
Thus corrections, changes or improvements of algorithms
w.\,r.\,t.\ efficiency  will not be mentioned here.

\paragraph{Version 1.0}
\redux\ (Version 1.0) is described in \cite{WalterBuendgen:91}.

\paragraph{Version 1.1}
The changes of Version 1.1 mainly concern the organization of the \redux\ 
sources.
Comments have been introduced to the data type descriptions.

\paragraph{Version 1.2}
A System description of \redux\  (Version 1.2) is published in
\cite{Buendgen:93}.
The demo programs TS, ITS and ICP were removed.
The term syntax allows for postfix, infix, roundfix and operators and
the LISP notation for terms.
The recursive path ordering has been extended to be compatible with
AC-operators.

\paragraph{Version 1.3}
The programs TC0, TCS, TCT and TCTS are combined to a new program TC.
The programs TST, ITST and PTST are  combined to a new program TS.
The user interface of all \redux\ programs has a uniform style.
Constant notation for nullary operators, purely commutative operators
and external terms have been  introduced.
A term ordering based on polynomial interpretations has been added and
the recursive path ordering has been augmented to allow for a
lexicographical right-left status.

\paragraph{Version 1.4}
\redux\ version 1.4 is compiled with the \ALDES-C compiler version 3.7.
The random term generator (TRD) and an experimental unfailing completion 
(UC) has been added.
The AC-completion has been modified to allow for user defined
strategy control.
It also supports the computation of generalized critical pairs and
critical pair subsumption.

\paragraph{Version 1.5}
The most important new feature of \redux\ version 1.5 is the new syntax of the
language for equational specifications which allows for a more
convenient and flexible term syntax.

Properties may now be assigned to constants too. Terms containing
auxilliary variables generated by the system may be parsed now.

The inductive completion procedure has been improved to take advantage
of free constructors even if not all constructors are free.

\paragraph{Version 1.6}
The most improtant new feature of version 1.6 is the new EV-rewrite
laboratory that allows to support normalization by evaluation domains
and memorization.


