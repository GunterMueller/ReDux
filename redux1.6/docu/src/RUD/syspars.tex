\section{Adjusting System Parameters}

Although this manual is intended to be read by users of the \redux\
demo programs that do not write programs, we include this section where
we describe how to adjust several system parameters.

One reason for this section is that even non-programming \redux\ users should
be able to modify these parameters is that they indicate how much RAM is
consumed by the demo programs. On computers with smaller RAM size,
it might be necessary to reduce the RAM consumtion to prevent degradation
of system performance.
The other reason is that it needs only a neglibly small knowledge of
the \ALDES\ programming language.

The system parameters are set in the source code of the \redux\ demo programs.
Therefore, one must have write access to the \redux\ system.
Table~\ref{tab:ReDuXMainProgramFiles} contains the file where modifications
must take place for each of the demo programs. The paths are relative to
the \redux\ installation directory.

\begin{table}[htbp]
\begin{center}
\begin{tabular}{|c||l|}
\hline
Demo Program & Source Code of Main Program\\ \hline
ac & src/ac/MAIN/ac.front\\
ev & src/acn/MAIN/ev.fornt\\
ic & src/ic/MAIN/ic.front\\
tc & src/tc/MAIN/tc.front\\
to & src/to/MAIN/to.front\\
trd & src/rd/MAIN/trd.front\\
uc & src/uc/MAIN/uc.front\\ \hline
\end{tabular}
\caption{\redux\ demo programs}\label{tab:ReDuXMainProgramFiles}
\end{center}
\end{table}

Then the \redux\ system must be recompiled. This task is accomplished
by making the \redux\ installation directory the current directory
and activating the installation {\tt make} file. To activate the
{\tt make} file, enter {\tt make all} or simply {\tt make}.

There are two system parameters that are available in all demo programs.
Both are set in a pragma declaration, i.\,e.\ a declaration of the form
\begin{quote}
\begin{verbatim}
pragma TNU=100000.
\end{verbatim}
\end{quote}
The two parameters are:
\begin{description}
\item[{\bf TNU}:] This parameter controls the size of the demo program's
heap space, i.\,e.\ the RAM where dynamic memory allocation takes place.
It indicates the number of machine words (4 bytes on a 32-bit computer)
that are provided for the heap space.
On the heap, the machine words are allocated pairwise, so {\tt TNU}
must be set to an even number.

\item[{\bf TMU}:] To facilitate garbage collection, \ALDES\ maintains
an own stack different from the system stack. The parameter
{\tt TMU} indicates the number of machine words that are provided
for this stack.

\end{description}

The random term generator (demo program {\tt trd}) maintains several
global arrays which determine the ``size'' of data types for which
random terms can be computed. To change the size of those arrays, {\tt
const}-declarations at the beginning of the header file
{\tt src/include/rd.h} must be adjusted. These constants are
{\tt MAXPOS}, {\tt MAXDEPTH1}, {\tt MAXSORTS}, {\tt MAXDOMS},
{\tt MAXOPPS}, {\tt MAXARITY}, {\tt MAXATOM} and {\tt MAXPOSINSTATS}.
For each of those constants, there is a comment in the source file
that describes its meaning.

Another possibility to reduce the RAM consumtion of the demo programs
{\tt ac}, {\tt trd} and {\tt tc} if no external terms are needed is to 
resign of the interpreter. To do so, one must modify the source code where
the main algorithm is defined. These file can be determined by taking
the file names from Table~\ref{tab:ReDuXMainProgramFiles} and
changing the file name extensions from {\tt front} to {\tt end}.
Within the file, one must search for step~1 of the main algorithm.
In this step, the function {\tt BGNRWD} is invoked. The argument of
{\tt BGNRWD} must be set to 4 instead of the standard value 5.
Before the \redux\ system is recompiled, the file {\tt Makefile}
in the {\tt demo} directory should be modified to link a dummy library
instead of the interpreter module. One has to edit the definition
of the macros {\tt ACLIB}, {\tt RDLIB} and {\tt TCLIB}. Each occurrence
of \verb!$(LIBDIR)/libint.a! must be replaced by
\verb!$(LIBDIR)/libintdummy.a!.
