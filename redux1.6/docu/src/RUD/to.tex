\section{The Term Ordering Program}
\subsection{Program Invocation}
The {\tt to}-program is used to experiment interactively with the 
term orderings implemented in the \redux\ system and the data type
in {\tt to.in}.

When the program {\tt to} is started, it prints a start message and 
the time needed for the program initialization.
Then it reads a data type and prints the time needed
for the acception of the data type. Subsequently, the data type is output,
again with the time needed for this procedure:
\begin{quote}
\begin{verbatim}
Loading your data type ...
Acception time: 34 ms.
 
DATATYPE GROUP;
SORT
        G;
CONST
        1: G;
VAR
        x, y, z: G;
OPERATOR
        1/: G -> G;
        *: G, G -> G;
NOTATION
        1/: PREFIX;
        *: INFIX;
AXIOM
 [1] 1 * x == x;
 [2] (1/x) * x == 1;
 [3] (x * y) * z == x * (y * z);
END
Display time: 0 ms.
\end{verbatim}
\end{quote}
 
(For this example, the data type {\tt ./spec/group.rdx} has been used.)


\subsection{The Main Menu}
After the data type has been loaded, the main menu is displayed:
\begin{quote}
\begin{verbatim}
        --  Menu  -- 
  1 -- add orderings
  2 -- remove orderings
  3 -- select an automatic ordering
  4 -- order the axioms of data type
  5 -- order an axiom
  6 -- toggle traces
  7 -- list the installed orderings
  8 -- display the data type
  0 -- exit program
 
Enter your choice [0-8]*
\end{verbatim}
\end{quote}

Before axioms can be oriented with the ordering algorithms, a 
particular ordering must be {\em selected}. This is done with menu
item 3.
Before orderings can be selected for usage, they must be {\em installed
(added)}, i.\,e.\  
information about a particular ordering has to be specified. This is
done in item 1 of the main menu. Therefore, this item should be selected
first.

\begin{enumerate}
{\bf \item{Add Orderings}}\\
The user must choose one or more term orderings for initialization.
There are algorithms for the Knuth-Bendix ordering, a
polynomial ordering and a path ordering (with l-r and r-l status).
See Section~\ref{InitializationOfOrderings}.


{\bf \item{Remove Orderings}}\\
This item allows the user to remove one or more of the installed
orderings:

\begin{quote}
\begin{verbatim}
Remove term orderings.
The following term orderings are installed:
P -- Path ordering (with l-r and r-l status)
3 -- 3. polynomial ordering

Which of these orderings shall be removed ?
Input string of {P,3} *
\end{verbatim}
\end{quote}

If one or more orderings are removed, {\em all} orderings will be deselected
automatically.

{\bf \item{Select an Automatic Ordering}}\\
This item is used to select an ordering. If at least one polynomial
ordering has been installed, the program allows to display the definitions
of these orderings:

\begin{quote}
\begin{verbatim}
The following term orderings are installed:
P -- Path ordering (with l-r and r-l status)
3 -- 3. polynomial ordering
S -- Straight (pseudo) ordering
R -- Reverse (pseudo) ordering

D -- Display polynomial ordering(s)

Input a string consisting of at most one occurrence of the above
                                                             orderings to
select the lexicographic combination of these orderings.
Input string or D to display *
\end{verbatim}
\end{quote}

(In the example, the path ordering and one polynomial ordering
is installed.)

The user is asked to
enter a lexicographic combination of the orderings chosen in 
menu item 1 (Knuth-Bendix,
polynomial or path ordering), and the straight respectively reverse
pseudo orderings.
Straight ordering means that for the left-hand side
$l$ and the right-hand side $r$ of an axiom, $l \succ r$.
Reverse ordering sorts the terms in the opposite direction.
Obviously, the straight and reverse pseudo orderings cannot
be used to prove the termination property of a term rewriting
system.

The simplest case of a lexicographic combination of orderings is
one of the four possible orderings. In this case, only the letter {\tt P},
{\tt 3}, {\tt S} or {\tt R} is entered.

If a lexicographic combination of more 
than one ordering is selected, then first the leading ordering is
applied. If the two terms are not comparable with this ordering, they
are not comparable with the combination. If they are comparable and not
equal, they are ordered with the leading ordering. If they are
comparable and equal, the next ordering of the lexicographic combination
of orderings is used, and so on. Such a combination is entered by 
concatenating the letters of the desired orderings; the leading ordering 
is the first letter.

{\bf \item{Order the Axioms of Data Type}}\\
This menu item is used to order the axioms of the data
type according to the selected ordering. If the path ordering
or a polynomial ordering has been initialized to query information by
need, the program will ask questions concerning these orderings.

Then the oriented axioms are displayed. If the two sides of an axiom are 
comparable, this is indicated by a {\tt <}-, {\tt >}- or {\tt =}-sign.
If they are not comparable, this is written explicitly:
\begin{quote}
\begin{verbatim}
1 * x   >   x
(1/x) * x   >   1
(x * y) * z   >   x * (y * z)
\end{verbatim}
\end{quote}

{\bf \item{Order an Axiom}}\\
This item allows to order an arbitrary axiom
that is entered by the user. The axiom must be entered in the form 
$l${\tt ==}$r$, where $l$ and $r$ are the left-hand side and the right-hand
side of the axiom. $l$ and $r$ are two arbitrary terms built of the
constants, variables and operators that have been defined in the data
type.

{\bf \item{Toggle Traces}}\\
By selecting this item, the trace status can be changed. After
starting the program, no trace information is displayed. When this item
is chosen once, the trace is turned on, the second time the trace is
turned off, and so on.
For the three orderings, there are different traces implemented:
\begin{itemize}
\item{\bf The Knuth-Bendix Ordering}\\
There is no trace facility for the Knuth-Bendix ordering.

\item{\bf The Path Ordering (with l-r and r-l status)}\\
All comparisons of terms (indicated by {\tt TERMS}), paths (indicated by
{\tt PATHS} and path sets ({\tt PSETS}) are output.

\item{\bf The Polynomial Ordering}\\
For each comparison of two polynomial interpretations, the following
output is displayed:

\begin{quote}
\begin{verbatim}
CPPI Polynomial interpretation LHS:
x3^2x1^2+2x3^2x2x1+x3^2x2^2
CPPI Polynomial interpretation RHS:
x3^2x1^2+x3^2x2^2
CPPI LHS-RHS:
2x3^2x2x1
\end{verbatim}
\end{quote}

The polynomial interpretations of the left-hand side and right-hand side of an
axiom is printed together with the difference of left-hand side minus
right-hand side.
\end{itemize}

{\bf \item{List Installed Orderings}}\\
The installed and selected orderings are shown. The string indicating the
selected orderings is interpreted as the lexicographic combination
of the selected orderings if it contains more than one letter. 


{\bf \item{List Data Type}}\\ 
The data type fed in at the program start is displayed.

\setcounter{enumi}{-1}
{\bf \item{Exit Program}}\\
This, of course, terminates the program after displaying some
statistical information.
\end{enumerate}

\subsection{Initialization of Term Orderings}
\label{InitializationOfOrderings}%
If the user has chosen to install one or more orderings, the following 
menu is displayed:

\begin{quote}
\begin{verbatim}
Already installed orderings:
P3
Initialization of term orderings.
The following term orderings are supported:
P  -- Path ordering (with l-r and r-l status)
K  -- Knuth-Bendix ordering. Not AC-compatible!
1. -- polynomial ordering
2. -- polynomial ordering
3. -- polynomial ordering
4. -- polynomial ordering
5. -- polynomial ordering
6. -- polynomial ordering
7. -- polynomial ordering
8. -- polynomial ordering
9. -- polynomial ordering

Which of these orderings shall be added ?
Input string of {P,K,1,2,3,4,5,6,7,8,9} *
\end{verbatim}
\end{quote}

Only one Knuth-Bendix or path ordering, but up to nine
polynomial orderings can be installed. The latter are labeled {\tt 1}
to {\tt 9}.
When the user has selected the orderings to work with, further information for
the initialization of the chosen orderings must be entered:

\subsubsection{Knuth-Bendix Ordering}
The Knuth-Bendix ordering \cite{KnuthBendix:67}
is based on an ordering over the set of all
operators and constants. This ordering is established by adjoining an index
(a non-negative integer) to each operand and constant so that
$f \succ g \ {\rm iff} \  {\rm index}(f) > {\rm index}(g)$
for two operators or constants $f$ and $g$ when ${\rm index}(f)$ denotes
the index of $f$.
The implemented
Knuth-Bendix ordering  needs a total ordering over the constants and
functions, i.\,e.\  ${\rm index}(f) \neq {\rm index}(g)$ for every pair
$(f,g)$ of constants or operators with $f \neq g$.

Furthermore, a weight (also a non-negative integer) must be assigned to
each operand and constant.
The weights of constants  and unary operators are restricted to positive
integers,
with the possible exception of the operator with largest index, whose
weight can be zero.

For each constant and operator, the user is asked for the appropriate
numbers. The positivity or non-negativity of weights and indices is
not checked.

\subsubsection{Path Ordering (with l-r and r-l Status)}
If the path ordering \cite{KapurNarendranSivakumar:85} is used,
the arguments of an operator are treated as multi set.
If the l-r (r-l) status is set, they are regarded as sequence
ordered from left to right (right to left).
The program asks for each operator with arity greater than 1 if it has 
the l-r or r-l-status.\footnote{We implemented the definition in
\cite{KapurNarendranSivakumar:85} for the l-r- and r-l stati.
However these definitions seem not to be complete.}
All commutative operators (with unification status {\tt AC} or {\tt COM})
are initialized without status.
Then the program displays the following menu:
\begin{quote}
\begin{verbatim}
Initialization of quasi-ordering over signature:
You have following choices: 
 n - initialize constants and operators by need
 q - query ordering relation for each constant and operator
 l - initialize quasi-ordering using list notation   [n/q/l]? *
\end{verbatim}
\end{quote} 
The menu item (n) ends the initialization and the relation between operators
will only be asked when needed.

The menu item (q) allows the user to totally initialize the whole ordering. 
All possible relations will be queried by the program:
\begin{quote}
\begin{verbatim}
* >= 1 [Y/N]?
y
1 >= * [Y/N]?
n
\end{verbatim}
\end{quote}

In this example, $+>0$. If both questions are answered with {\tt Y},
{\tt +} and {\tt 0} are equal in the applied ordering, if both questions are 
answered with {\tt N}, the two terms are not comparable.

If the user chooses (l) 
the program allows to input relation lists specifying that 
an operator or constant is less than (\verb+<+), greater than (\verb+>+), 
less than or equal (\verb+<=+), greater than or equal (\verb+>=+),
equivalent (\verb+~+) or incomparable (\verb+#+) to a list of
operators and/or constants.
For example a possible input is:
\begin{quote}
\begin{verbatim}
* >= 1/, 1;
\end{verbatim}
\end{quote}
If the user inputs only a semicolon ``;'' the initialization ends.
Note, that a previously given relation will not be overwritten by a second one. 

\subsubsection{Polynomial Orderings}
For a polynomial ordering, a multivariate integral polynomial
must be assigned to each constant and operator. The rank of a
polynomial is the same as the arity of the corresponding operator in the data
type, the polynomial of a nullary operator is a constant. The
indeterminate $x_i$ corresponds to the \(i\)-th argument of the operator.
The grammar of the possible inputs is given in
Figure~\ref{fi:SyntaxOfPolynomials}.

\begin{figure}[htbp]
\begin{center}
\noindent
\rule{\textwidth}{0.6pt}
 
%\fbox{
\begin{minipage}{5.9in} %necessary if there is a footnote within the figure
\begin{tabbing}
\vs{Indeterminate} \= \gprod \= \kill
\\
\>\vs{Polynomial}\' \gprod \>
      \gopt{\ts{+} \galt \ts{-}} \vs{Monomial}
      \gstar{\group{\ts{+} \galt \ts{-}} \vs{Monomial}}\\
\>\vs{Monomial}\' \gprod \> \vs{Exp}
      \gstar{\group{\ts{*} \vs{Exp}} \galt \group{\ts{/} \vs{Number}}}\\
\>\vs{Exp}\' \gprod \> \vs{Num} \gopt{\ts{\^} \gplus{\vs{Digit}}}\\
\>\vs{Num}\' \gprod \> \ts{(} \vs{Polynomial} \ts{)} \galt
      \vs{Indeterminate} \galt \gstar{\vs{Digit}} \\
\>\vs{Number}\' \gprod \> \ts{(} \gopt{\ts{+} \galt \ts{-}}
      \gplus{\vs{Digit}} \ts{)}\galt \gplus{\vs{Digit}} \\
\>\vs{Indeterminate}\' \gprod \> \ts{x}\gplus{\vs{Digit}} \\
\>\vs{Digit}\' \gprod \>
         \ts{0} \galt $\ldots$ \galt \ts{9} \\
\>\vs{Letter}\' \gprod \>
         \ts{a} \galt $\ldots$ \galt \ts{z} \galt  
         \ts{A} \galt $\ldots$ \galt \ts{Z} \\
\end{tabbing}
\end{minipage} 

\noindent
\rule{\textwidth}{0.6pt}
%} end fbox
\end{center}
\caption{The Syntax of Polynomials}
\label{fi:SyntaxOfPolynomials}
\end{figure}
First, the program asks if all polynomials shall be entered altogether
or if they should be queried by need. Then, the user is asked for
the configuration of the positivity test:

\begin{quote}
\begin{verbatim}
Configuration of the positivity test:
You have the following choices for the change of coefficients: 
 o - use the solution 1 of Ben Cherifa and Lescanne
 t - use the solution 2 of Ben Cherifa and Lescanne
   [o/t]? *
\end{verbatim}
\end{quote}

Solution~2 is slower but stronger than solution~1,
solution~2 may orient a particular axiom if solution~1 fails.
The solutions referred to can be found in \cite{BenCherifaLescanne:87}.

After those queries, the polynomial interpretations can be entered,
either instantly or later during the program.
Coefficients and indeterminates must be separated by a multiplication star,
exponents are indicated by a hat symbol (\verb!^!) and the whole polynomial
is terminated by a period. The indeterminates are written as {\tt x1},
{\tt x2}, {\tt x3} and so forth.

\noindent
{\bf Example:}

\begin{quote}
\begin{verbatim}
Input polynomial interpretation of operator:
*: G, G -> G.
Use variables x1, x2 for arguments 1 to 2.   *
5*x1^7+x2+1.
\end{verbatim}
\end{quote}

Note that the polynomial interpretations should be non-negative for all
possible instantiations with interpretations of ground terms and they
should be monotonously ascending in each variable. These properties are
not checked so far.

The positivity-test tries to decide the positivity of a polynomial to
determine the orientation of an axiom. For example, if there occur
polynomial interpretations of the left-hand side of an axiom, say
\(x_3^2x_1^2+2x_3^2x_2x_1+x_3^2x_2^2\), and the right-hand side, say
\(x_3^2x_1^2+x_3^2x_2^2\), the difference left-hand side minus right-hand side
is \(2x_3^2x_2x_1\). In this case there is no negative monomial and therefore
the difference is positive and the axiom is ordered left to right. In a
case where a negative monomial occurs the positivity-test tries to
find a positive monomial which bounds it. For this work there is so
far only one algorithm implemented (choose-monomial-algorithm).
If such an bound exists the negative monomial will deleted and the
coefficient of the positive monomial will be changed by an
change-monomial-algorithm. This modified polynomial is the study of
the positivity-test now. If at the end the polynomial is positive the
axiom can be ordered.
