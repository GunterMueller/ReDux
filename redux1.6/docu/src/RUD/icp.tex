\section{The Inductive Completion Programs}
\subsection{The icp Program}
The {\tt icp} demo program allows the user to experiment with the
inductive completion algorithm \cite{Buendgen:87}.
As all of the ReDuX demo programs, it reads an algebra specification
from the file {\tt fort.4}. For this program the set of axioms in the 
algebra specification must be ground confluent and terminating.

After the data type is read, the user must initialize the term orderings.
(See Section \ref{InitializationOfOrderings} for more information on the
term orderings.)
Then he can decide if he wants to trace the program. If tracing has been
chosen,
the user can select among the seven traces, namely the 
theorem derivation trace, the proved theorems trace, the formula reduction 
trace, the inessential critical pair trace, the procedure name trace, 
the counter example trace and the substitution trace.

Now the program computes a decomposition of the data type into constructors 
and defined
function symbols and a ground reduction test set as is performed in the
{\tt its}, and {\tt itst} programs (see Sections \ref{ProgramIts} and 
\ref{ProgramItst}).
The decomposition and the test set are shown to the user.

After these prerequisites the main menu is displayed:
\begin{quote}
\begin{verbatim}
You have following choices: 
  R - reduce term,
  P - prove theorem,
  B - display basic axioms,
  C - display current data type,
  O - select term ordering,
  E - exit.
[R/P/B/C/O/E] *
\end{verbatim}
\end{quote}
The user must enter one of the menu items by selecting the corresponding
letter.

\begin{itemize}
\item {\bf Reduce Term (R)}\\
This menu item allows the user to enter a term consisting of the
constants, variables and functions of the used data type.
The program tries to reduce the term by the axioms (interpreted as
rewrite rules) in the current data
type. The normalized term (which can be the same as the entered term) 
is printed,
together with some statistical information concerning the reduction process,
including the reduction rate (the number of rewrites per second).
The reduction rate is omitted if 
the whole reduction process happens between two consecutive clock ticks.
(For the SunOS 4.0 the period between two clock ticks is $16\frac{2}{3}$ms.)

\item {\bf Prove Theorem (P)}\\
The user must enter an axiom (consisting of two terms over the used data
type that are separated by {\tt ==}).

The program checks first if the entered theorem is an equational 
consequence of the algebra specification (i.e. the terms on both sides of
the entered theorem can be reduced to the same term by application of the 
axioms in the algebra). If this is true, a message like
\begin{quote}
\begin{verbatim}
The theorem  0) +(0,A) == A
is valid in all models of the data type INT.
Both terms can be reduced to A.
\end{verbatim}
\end{quote}
is displayed.

If the theorem cannot be proved by application of the axioms in the data type,
the program prints a message like:
\begin{quote}
\begin{verbatim}
The theorem  0) +(+(A,B),C) == +(A,+(B,C))
 is not an equational consequence, 
it may however be an inductive consequence of
INT data type.
\end{verbatim}
\end{quote}
Then the program tries to prove the theorem by applying the extended 
Knuth-Bendix algorithm for inductive completion given in \cite{Kuechlin:89}.
The entered 
theorem is oriented and added to the set of axioms. If the new set of
axioms  can 
be completed by the algorithm, the entered theorem is a inductive consequence
of the initial set of axioms, otherwise it cannot be proved by inductive
completion.

To start the completion, the user must choose if trace information shall be displayed.
The same seven traces as at the program start can be selected.
Then the program asks if the inductive completion shall be done in step 
mode or in automatic mode. If step mode has been chosen, the user must 
orient the computed critical pairs manually.

In the completion process terms may be proved to be inductively 
reducible. In general, a term can be inductively reducible at several
positions. 
These inductively reducible terms are displayed, together with the
positions at which they are reducible and the axioms for which they
are inductively complete.
Positions are printed as a sequence of positive numbers
that are separated by points. The empty position $\lambda$ is displayed 
as \verb+/\+. The user must choose one of the given positions.
In the following example the term is reducible only
at one position, therefore the user has no real choice:

\begin{quote}
\begin{verbatim} 
 Old axiom no 14,
+(A,NEG(B)) == NEG(+(NEG(A),B)),
 considered. Proposing old axiom no 14,
+(A,NEG(B)) == NEG(+(NEG(A),B)),
 as new axiom no 14.NEG(+(NEG(A),B)) is inductively reducible.
1. The set of positions: 
NEG(A) at pos. 1.1
is inductively complete with rules: 
 7) NEG(S(A)) == P(NEG(A))
 8) NEG(P(A)) == S(NEG(A))
 6) NEG(0) == 0
 
Choose set of positions [1 - 1]. *
\end{verbatim}
\end{quote}

Then the number of critical pairs computed with the new axioms are 
printed. The programs differentiates between essential and inessential
critical pairs. The inessential critical pairs are not 
necessary for correctness reasons, but may be helpful to ensure the 
termination of the completion process.
If an essential critical pair can be reduced by an inessential critical
pair, the program asks if the inessential critical pair shall be treated next:

\begin{quote}
\begin{verbatim}
The essential critical pair
( 5 and  8) ( 5)  OPP(SUCC(+(A,SUCC(0)))) == OPP(SUCC(SUCC(A)))
can be reduced by 
the inessential critical pair
( 8 and  2) ( 3)  SUCC(A) == +(A,SUCC(0))
using Straight and Reverse ordering.
do you want to treat the inessential critical pair next [Y/N]?
\end{verbatim}
\end{quote}
The user must answer with {\tt Y} or {\tt N}. If he entered {\tt Y}, the
inessential critical is the next critical pair that is oriented,
otherwise the essential critical pair is oriented next.

If the the completion succeeds, the program prints the proven theorem
together with a message that it is an inductive theorem of the data 
type. Then the program displays the basic axiom set and a list of
the new theorem and the lemmas that have
been proved while the completion procedure. Now the program prints 
some statistics and a menu that allows the user to choose the data 
type for the following computations:

\begin{quote}
\begin{verbatim}
You have following choices: 
 B - Continue using basic axiom set,
 L - Continue using last axiom set,
 N - Continue using new axiom set
 E - Exit program.
Enter your choice [B/L/N/E] *
\end{verbatim}
\end{quote}

\begin{itemize}
\item Continue using basic axiom set (B)\\
The basic axiom set (the algebra specification read-in at the program 
start) is used for further computations.
\item Continue using last axiom set (L)\\
The current data type before the last completion process is used.
\item Continue using new axiom set (N)\\
The new data type (including the new theorem and lemmas) is used. Therefore the 
program must compute a new decomposition into constructors and defined 
function symbols and a new ground reduction test set.
\item Exit program (E)\\
This menu item allows to leave the program after some statistical 
information concerning the garbage collector has been displayed.
\end{itemize}
 
If the completion does not succeed, this is told to the user.
Then he has the choice to use the basic axioms or the last axioms for the
next computation or to leave the program.

\item {\bf Display Basic Axioms (B)}\\
The basic axioms, i.e. the axioms read-in at the program start are 
displayed.

\item {\bf Display current data type (C)}\\
This item displays the current data type, i.e. the initial data type 
that has been extended with all inductively proven theorems.

\item {\bf Select term ordering (O)}\\
The user can select a lexicographical combination of the initialized 
term orderings and the straight and reverse pseudo orderings.

\item {\bf Exit (E)}\\
This item allows the user to quit the program.
\end{itemize}

\subsection{The lab Program}
The {\tt lab} demo program is also written to allow the user to experiment with 
the inductive completion algorithm. In contrast to the {\tt icp} program,
the input data type need not to be canonical. Therefore the program first
tries to determine if the axioms (oriented from left to right)
are terminating. There are three cases:

\begin{itemize}
\item The axioms can be proven to be non-terminating.
In this case the program prints a message and runs the Knuth-Bendix 
algorithm (see Section \ref{KBCompletion}).

\item The axioms cannot be proved to be terminating or not terminating. 
The the programs assumes
that they are terminating and prints a warning message. Then it does the
same as if the axioms could be proved to be terminating.

\item If the axioms can be proved to be terminating, the program tests
the convergence of axioms interpreted as term rewriting system.

\begin{itemize}
\item 
If the data type is convergent, the program prints this explicitly and
continues. 

\item If it is not convergent, the program 
prints a list of the divergent critical pairs.
Then the user is given the choice to run the Knuth-Bendix algorithm.
Otherwise it is assumed that the data type is ground confluent.
\end{itemize}
\end{itemize}

After these initial tests have been done, the program does the same as
the {\tt icp} program, i.e. it continues with the decomposition of the 
data type and goes to the main menu.
