\section{The Term Completion Program}
\subsection{Introduction}
The {\tt tc}-program is used to experiment with the term completion algorithms
\cite{KnuthBendix:67}
implemented in the \redux\ system:

\begin{itemize}
\item The Knuth-Bendix procedure
\item The Knuth-Bendix procedure with subconnectedness criterion
\item The Knuth-Bendix procedure with transformation criterion
\item The Knuth-Bendix procedure with subconnectedness and
transformation criterion
\end{itemize}

The function of the completion procedures is to transform a set of
equations  into a canonical set of rules.
While this process is executed, critical pairs 
are computed. 
The canonical set of rules is constructed by orienting equations and critical
pairs and adding them incrementally to this set.
All four completion procedures follow roughly the strategy 
depicted in Figure~\ref{fi:cmpl}.
\begin{figure}[ht]
\begin{center}
\fbox{
\begin{minipage}{5.5in} %necessary if there is a footnote within the figure
\begin{quote}
\begin{center} ${\cal R} \leftarrow$ {\bf COMPLETE}$({\cal E},\succ)$
\end{center}
[Completion procedure. \\
$\cal E$ is a set of equations and $\succ$ is a terminating term ordering.
Then ${\cal R}$ is a canonical term rewriting system with
\( =_{\cal R} \; = \; =_{\cal E} \).]
\begin{description}
 \item[{(1) [Initialize.]}] \(  {\cal R} := \emptyset \).
 \item[{(2) [Simplify.]}] {\bf while} the {\em simplify}-inference rule
   applies {\bf do} \\
  \( ({\cal E};{\cal R}) := Simplify(({\cal E};{\cal R})). \)
 \item[{(3) [Delete.]}] {\bf while} the {\em delete}-inference rule
   applies {\bf do} \\
  \( ({\cal E};{\cal R}) := Delete(({\cal E};{\cal R})). \)
 \item[{(4) [Stop?]}] {\bf if} \(  {\cal E} = \emptyset \) {\bf then return}
      ${\cal R}$ and {\bf stop}.
 \item[{(5) [Orient.]}] Let \( a \leftrightarrow b \in {\cal E} \) then
      \( {\cal E} := {\cal E} \setminus \{ a \leftrightarrow b \} \); \\
      {\bf if} \( a \succ b \) {\bf then} \(  \bgn l := a; r  := b \nd \) \\
      {\bf elsif} \( b \succ a \) {\bf then} \(  \bgn l := b; r  := a \nd \) \\
      {\bf else stop} with failure; \\
      \( {\cal R} := {\cal R} \cup \{ l \rightarrow r \} \).
 \item[{(6) [Collapse.]}] {\bf while} the {\em collapse}-inference rule
     applies {\bf do} \\
    \( ({\cal E};{\cal R}) := Collapse(({\cal E};{\cal R})). \)
 \item[{(6') [Subconnectedness criterion.]}]
    \( {\cal E} := {\cal E} \setminus \{ \mbox{Equations originating from a
    collapsed rule} \} \).
 \item[{(7) [Compose.]}] {\bf while} the {\em compose}-inference rule
   applies {\bf do} \\
   \( ({\cal E};{\cal R}) := Compose(({\cal E};{\cal R})). \)
 \item[{(8) [Deduce.]}] Compute all critical pairs ${\cal P}$ of
    $l \rightarrow r$ and rules
    in $\cal R$. \( {\cal E} := {\cal E} \cup {\cal P} \)\footnote{Here we
                                             identify pairs and equations.};
    \\ {\bf continue} with step 2. \hfill $\Box$
\end{description}
\end{quote}
\end{minipage}
}
\end{center}
\caption{Algorithm {\em COMPLETE}} \label{fi:cmpl}
\end{figure}
The {\em simplify}-, {\em delete}-, {\em collapse}-, and {\em compose}
inference rules are those presented in \cite{Dershowitz:89}.
The equation set $\cal E$ is realized as two queues.
A queue of {\em old axioms} containing the initial equations of the 
input data type and collapsed rules (step~6), and a second queue of 
{\em theorems} which contains the critical pairs deduced in step~8.
The rule set $\cal R$ is implemented as a list of {\em (new) axioms}.
The equation selection strategy in step~5 gives old axioms priority 
over theorems and selects the equations in the same queue according to
their size.

When the program {\tt tc} is started, it prints a start message and 
the time needed for the program initialization.
Then it reads a data type and prints the time 
for the acception of the data type. After that, the data type is output,
again with the time needed for this procedure.

\subsection{The Main Menu}
After that the program's main menu is displayed:
\begin{quote}
\begin{verbatim}
You have following choices: 
  u - unify two terms 
  m - match two terms 
  e - test equality of two terms 
  p - prove equational theorem
  n - normalize term 
  k - run Knuth-Bendix completion procedure
  i - de/install term ordering
  s - select term ordering
  o - order axiom
  O - order axioms of data type
  d - display data type
  S - show times and counters
  r - reset times and counters
  h - help, print menu
  q - quit
Go on [u/m/e/n/k/i/s/o/O/d/S/r/h/q]?  *
\end{verbatim}
\end{quote}

\label{ShortMainMenu}
After one of the above menu items has been processed, the program
prints the main menu in its short version:
\begin{quote}
\begin{verbatim}
Go on [u/m/e/n/k/i/s/o/O/d/S/r/h/q]?  *
\end{verbatim}
\end{quote}
The letters have the same meaning as in the long form of the menu.
The long form can be called by selecting {\tt h} or by striking
the $\langle$return$\rangle$ key.

If one of the menu items ``unify'', ``match'', ``test equality of two terms''
or ``prove equational theorem''
have been chosen previously, for each of the former three items an additional
item indicated by an upper case letter is displayed. These
items perform the same task as the original menu items, but they 
do not read two terms from the keyboard but use the previously
entered terms. Again, a long 
version of the menu can be called with {\tt h} or $<$return$>$.

\begin{itemize}
\item {\bf Unify Two Terms (u})\\
The user is prompted to enter two terms.
If the terms are unifiable, the program prints the
most general common instance (mgci) of the two terms:
\begin{quote}
\begin{verbatim}
1 * 1/1 = mgci( 1 * 1/1 , x * y )
\end{verbatim}
\end{quote}

If the terms are not unifiable, this is printed explicitly.

\item {\bf Match Two Terms (m)}\\
This item allows the user to match two terms, i.\,e.\  the program attempts
to find an instance of the first term which equals the second term.

\item {\bf Prove Equational Theorem (p)}\\
This item allows to prove a equational theorem w.\,r.\,t.\  the
current algebraic specification. The algebraic specification must be
canonical.

\item {\bf Test Equality of Two Terms (e)}\\
The user is prompted to enter two terms $t_1$ and $t_2$. Then the program 
prints a message like $t_1$ \verb/=/ $t_2$ if the terms are equal
or $t_1$ \verb/#/ $t_2$ if the terms are not equal.

\item {\bf Normalize a Term (n)}\\
The user must enter a term. Then the program 
computes the corresponding normal form w.\,r.\,t.\  to the current data 
type. The normal form can be equal to the entered term if no axiom in
the data type can be applied to the term.

\item {\bf Run Knuth-Bendix completion procedure (k)}\\
See Section~\ref{KBCompletion} for a description of the
Knuth-Bendix procedure. A term ordering must be installed and selected
if the Knuth-Bendix algorithm shall be used in automatic mode.

\item{\bf (De-)Install Term Ordering (i)}\\
This item allows the user to install or de-install a term ordering. See 
Section~\ref{InitializationOfOrderings} for more information
on term orderings.

\item{\bf Select Term Ordering (s)}\\
The user can select a lexicographical combination of the initialized
term orderings and the straight and reverse pseudo orderings.
To initialize a term ordering use the previous menu item.

\item{\bf Order Axioms (o)}\\
This item allows to order an arbitrary axiom that is entered by the user.
The axiom must be entered in the form $l${\tt ==}$r$, where $l$ and $r$
are the left-hand side and the right-hand side of the axiom. $l$ and $r$
are two arbitrary terms built of the constants, variables and operators
that have been defined in the data type.

\item{\bf Order Axioms of Data Type (O)}\\
This item allows the user to order the axioms of the data type according
to the selected ordering.

\item {\bf Display Data Type (d)}\\
The program displays the current data type. If the Knuth-Bendix
completion procedure has been applied, the current data type may be
different from the initial data type that was fed in at the program
start.

\item {\bf Show Times and Counters (S)}\\
The counters (number of matches, reductions and unifications) and the 
the timers (cpu-time used so far, cpu-time spent for matching,
reduction and unification) are displayed.

\item {\bf Reset Times and Counters (r)}\\
The counters and timers are reset, either altogether or selectively.

\item {\bf Help, Print Menu (h)}\\
This item prints the main menu. It can be useful if the program
displays the short form  of the main menu. The same action is performed
if only $\langle$return$\rangle$ is struck.

\item {\bf Quit (q)}\\
The program can be aborted by selecting this menu item.
\end{itemize}

\subsection{The Knuth-Bendix Completion}
\label{KBCompletion}

If the Knuth-Bendix completion procedure is selected in the main
menu, the user is prompted to choose one of the available variants 
of the Knuth-Bendix algorithm:
\begin{quote}
\begin{verbatim}
Which kind of completion procedure do you want to run?
  p - plain completion
  s - completion with subconnectedness criterion
  t - completion with transformation criterion
  c - completion with both criteria
  q - quit completion
Choose one of [p/s/t/c/q] *
\end{verbatim}
\end{quote}

\begin{itemize}
\item {\bf Plain Completion}\\
The basic Knuth-Bendix completion presented in Figure~\ref{fi:cmpl}  is run.

\item {\bf Completion with Subconnectedness Criterion}\\
In addition to step 6', 
the subconnectedness criterion is also applied in the Deduce step (8)
as described in \cite{Kuechlin:86}.

\item {\bf Completion with Transformation Criterion}\\
The transformation criterion as given in \cite{Buendgen:91,Buendgen:91b}
is used.
The procedure tries to orient all computed critical pairs (theorems) 
during the critical pair computation step.
The program prints the number of orientable and non-orientable critical pairs
it has computed after a new axiom is taken to the set of new axioms:
\begin{quote}
\begin{verbatim}
4 orientable critical pairs and 0 non-orientable critical 
                                                  pairs have been derived.
\end{verbatim}
\end{quote}
In this message only unproved theorems are considered.

Furthermore, the program prints a message each time  the transformation
criterion has been applied.

If this completion procedure  is used in step mode, the user must orient
the proposed new axioms {\em and} all computed 
theorems (critical pairs). The program displays a message like the following:
\begin{quote}
\begin{verbatim}
 Theorem
z == (1/x) * (x * z)
 (from 2 and 3) has been computed.
Orient [S/R/N]  *
\end{verbatim}
\end{quote}
The theorem is ordered straight (reverse) if the user enters {\tt s}
({\tt r}). If {\tt n} is entered, the theorem will be considered as
non-orientable. If the user only strikes the $<$return$>$-key, the program
will display a help menu:
\begin{quote}
\begin{verbatim}
 Your input options are -
     (S)traight / (R)everse
    (T)race  / Do (N)ot orient
  auto(M)atic ordering proposal
         (P)rompt input
Orient [S/R/N]  *
\end{verbatim}
\end{quote}
The other options (T, M and P) are the same as described in
Section~\ref{StepMode}.

\item {\bf Completion with both Criteria}\\
In this completion procedure, the subconnectedness criterion and the
transformation criterion are combined.
\end{itemize}

After choosing an completion algorithm,  the user can select if trace
information shall be displayed.
An arbitrary combination of the 
six built-in traces, namely the theorem derivation trace, the proved theorems
trace, the formula reduction trace, the general ordering trace, the polynomial 
ordering trace and the path ordering trace can be chosen. Each of the traces
is turned on by typing {\tt 1} or off by typing {\tt 0}.
The trace options are described in Appendix~\ref{Traces}.

Then the program asks if the user wants to work in step mode or automatic mode. 
(The step mode will be described in the Section~\ref{StepMode}.)

Subsequently, the programs starts with the completion. For each rule added
into the set of new axioms some lines of information are displayed:
                                                                               
\begin{quote} 
\begin{verbatim}
 Old axiom no 1,
1 * x == x,
 considered Proposing old axiom no 1,
1 * x == x,
 as new axiom no 1.
1 * x   >   x
 No axiom can be reduced by new axiom no 1.

 No theorems derived so far.
\end{verbatim}
\end{quote}
This means that the set of new axioms has now one element, namely the
first axiom of the set of old axioms, ordered straight.
 
After putting a theorem/old axiom to the set of new axioms, it is checked
if an earlier axiom can be reduced with the new axiom. In our example,
this is not possible. Furthermore, we can see that the system was not
able to derive any theorems.
 
Another possible output is:
\begin{quote}
\begin{verbatim} 
 Proposing theorem
1/x == y * 1/(x * y)
 (from 12 and 4) as new axiom no 13.
1/x   <   y * 1/(x * y)
 1 axiom can be reduced by new axiom no 13.
 13 new theorems can be derived by new axiom no 13.
  6 of them must be retained unproved.

 79 theorems derived in total,
  8 of them still unproved.

 Old axiom no 12,
x * (y * 1/(x * y)) == 1,
 considered and proved.
\end{verbatim}
\end{quote} 

In this example, one of the derived theorems (i.\,e.\  critical pairs) is put
into the set of new axioms. It was derived from the new axioms twelve and
four and is the thirteenth new axiom. With the new axiom, one of the earlier
new axioms can be proved. In this case, it is axiom 12 as is indicated some
lines below.

Thirteen new theorems could be derived by computing critical pairs between
new axiom 13 and the remaining new axioms,
but only six must  be retained
unproved. Together with these theorems, 79 theorems have been derived so far,
eight of them must be kept unproved.

If the chosen ordering cannot order an old axiom or theorem, a message
of the following form is output:
\begin{quote}
\begin{verbatim}
 Proposing theorem
1 == x * 1/x
 (from 2 and 6) as new axiom no 8. The automatic ordering failed.  *
\end{verbatim}
\end{quote}
Now the user can manually orient the axiom or theorem straight by typing 
{\tt s} or
reverse by typing {\tt r}. If the user just types $\langle$return$\rangle$,
a menu of the
available functions is displayed. This menu is described in 
Section~\ref{StepMode}. If the user orients an axiom or theorem 
manually the termination 
property of the resulting term rewriting system may be violated. The system 
has no way to check this\footnote{As a matter of fact, the problem to 
decide whether a term rewriting system has the termination property is
undecidable.}.

If the term completion algorithm succeeds, the new data type is
displayed together with some statistical information, including a
histogram of the length of the theorem queue (the set of all 
unproved theorems).
The axioms in the displayed data type are rules, although they are
printed as equations ({\tt ==} separates the both sides of each axiom).
Therefore {\tt ==} must be read as $\rightarrow$.

Finally, the user is asked if the old or the new data type shall
be used in further computations.

\subsection{Working in Step Mode}
\label{StepMode}
If the completion procedure runs in step mode, old axioms and theorems
are not oriented automatically. The user is prompted with the
old axiom or theorem to be oriented and entered into the set of new axioms.
There are several options that can be chosen. If the $<$return$>$-key is struck,
a menu of the possible actions is displayed:
\begin{quote}
\begin{verbatim}
 Your input options are -
     (S)traight / (R)everse
      (T)race  / (D)isplay
  auto(M)atic ordering proposal
  (I)ntroduction of new operator
    (P)rompt input / (F)ertilize
         (Q)ueue / (E)nd
  *
\end{verbatim}
\end{quote}

\begin{itemize}
\item {\bf Straight (S)}\\
If this item is chosen, then the proposed new axiom is ordered straight
i.\,e.\  the left-hand side of the displayed axiom or theorem will become the
left-hand side of the rule in the set of new axioms.
\item {\bf Reverse (R)}\\
By typing {\tt R} (or {\tt r}) the proposed new axiom is ordered
reverse, i.\,e.\  the left-hand side of the displayed equation will be the
right-hand side of the new axiom.
\item {\bf Trace (T)}\\
This item allows to select the six built-in traces. The possibilities 
to change the traces are exactly the same as at the program start.
\item {\bf Display (D)}\\
If `Display' has been selected, the program shows a menu:
\begin{quote}
\begin{verbatim} 
 Your options for display are -
         (U)nproved theorems
      (C)onsts / (V)ars / (O)ps
     old (A)xioms / new a(X)ioms
  *
\end{verbatim}
\end{quote}
{\tt U} shows the queue of unproved theorems, each theorem is displayed together
with  its weight and the new axiom it is derived from.

With {\tt C}, {\tt V}, {\tt O} and {\tt A}, the user can choose the constants, 
variables, operators and axioms of the read-in data type to be displayed.

The option {\tt X} allows the set of new axioms at the current state to
be displayed. Again, {\tt ==} must be read as $\rightarrow$.
\item {\bf Automatic Ordering proposal (M)}\\
The program shows how the installed and selected ordering orients the proposed
axiom.
\item {\bf Introduction of new operator (I)}\\
This feature is not implemented yet.
\item {\bf Prompt Input (P)}\\
The considered old axiom or theorem is shown again.
\item {\bf Fertilize (F)}\\
This feature is not implemented yet.
\item {\bf Queue (Q)}\\
The queue of unproved theorems is cyclically permuted, i.\,e.\  the proposed
theorem is appended to the end of the queue and its the former first entry is
proposed as new axiom.
\item {\bf End (E)}\\
This item allows to terminate the program. The data type with the new
axioms at the current state is displayed, together with some statistical
information.
\end{itemize}
