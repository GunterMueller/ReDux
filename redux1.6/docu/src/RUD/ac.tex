\section{The AC-Rewrite Laboratory}
\subsection{Program Start}
The AC-rewrite laboratory allows the user to experiment with
a data type that may contain commutative and/or 
associative and commutative operators.
As all \redux\  demo programs, {\tt ac} reads an algebra specification
and shows it to the user.

After the initial data type has been printed, the program
may extend the data type with {\em AC-extension rules} to
ensure its AC-compatibility \cite{PetersonStickel:81}.  
When a data type is displayed later during the program, each of the additional
AC-extension rules is printed under the corresponding main axiom:
\begin{quote}
\begin{verbatim}
 [2] +(I(X),X) == 0;
  AC-ext: +(x,I(X),X) == x;
\end{verbatim}
\end{quote}
(Note that the left-hand side of the AC-extension rule is printed
in flattened form.)
%### Die Beispiele wurden mit spec/ac/acring1.rdx gemacht.

\subsection{The Main Menu}
Then the program displays the main menu: 

\begin{quote}
\begin{verbatim}
You have the following choices: 
  u - unify two terms 
  m - match two terms 
  e - test equality of two terms 
  n - normalize a term
  x - compute an external term
  t - set/unset trace options
  d - display data type
  c - compute critical pairs
  k - run (AC)-Knuth-Bendix completion procedure
  p - prove equational theorem
  i - de/install term ordering
  s - select term ordering
  o - order axioms of data type
  O - order axioms
  S - show times and counters
  r - reset times and counters
  h - help, print menu
  q - quit
Go on [u/m/e/n/x/t/d/c/k/p/i/s/o/O/S/r/h/q]?  *
\end{verbatim}
\end{quote}

If one of the menu items ``unify'', ``match'', ``test equality of two terms''
or ``prove equational theorem''
have been chosen previously, for each of the former three items an additional
item indicated by an upper case letter is displayed. These
items perform the same task as the original menu items, but they 
do not read two terms from the keyboard but use the previously
entered terms. Hitting $\langle$return$\rangle$ without any preceding
letters will do the same as menu item {\tt h}.

\begin{itemize}
\item {\bf Unify Two Terms (u)}\\
The user is prompted to enter two terms.
If the terms are unifiable modulo the unification states specified
in the algebraic specification, 
the program prints the most general
unifier (mgu) and the most general common instance (mgci) of the two 
terms. The mgu is printed as one or more substitutions. (If 
AC-operators or commutative operators are involved, there may be more
than one possible
substitution.) Each substitution is printed as a set of domain variables
and the corresponding terms that replace that variables:
\begin{quote}
\begin{verbatim}
mgu( +(0,1) , +(X,Y) )  mod AC :
{ X <- 1, Y <- 0, x <- 1, x_1 <- 0 }
{ X <- 0, Y <- 1, x_2 <- 1, x_3 <- 0 }
 
mgci(+(0,1),+(X,Y)) mod AC :
+(0,1)
\end{verbatim}
\end{quote}
The variables \verb/x/, \verb/x_1/, \verb/x_2/ and \verb/x_3/ are
auxiliary variables. In this example they can be ignored by the user,
but it is possible that a variable is substituted by a term containing 
an auxiliary variable.

If the terms are not unifiable, this is printed explicitly.

\item {\bf Match Two Terms (m)}\\
This item allows the user to match two terms modulo the unification
states, i.\,e.\  the
program attempts to find an instance of the first term which equals the
second term modulo the unification states.
If the first term matches the second one, the matching substitutions
are displayed.  They are printed like the mgu, namely as one or more sets of
domain variables and corresponding terms.

If the first term does not match the second one, this is printed
explicitly.

\item {\bf Test Equality of Two Terms (e)}\\
The user is prompted to enter two terms $t_1$ and $t_2$. Then the program 
prints a message like $t_1$ \verb/=/ $t_2$ if the terms are equal modulo
the unification states or $t_1$ \verb/#/ $t_2$ if the terms are not equal.

\item {\bf Normalize a term (n)}\\
The user must enter a term. Then the program 
computes the corresponding normal form according to the current data 
type.
Normal forms of terms with unification states are equivalence classes
of terms modulo the unification states.
The normalization procedure computes a representative of this
equivalence class.

This representative can be equal to the entered term if no axiom in
the data type can be applied to the term.

\item {\bf Compute an External Term (x)}\\
The user is asked to enter a (mixed) term. A copy of that term is
first normalized and then computed (see Section~\ref{ExternalObjects}).
Another copy is first computed and then normalized.

\item {\bf Set/Unset Trace Options (t)}\\
This item allows the user to select the traces in the {\tt ac} program,
namely the formula reduction trace, the critical pair trace and the 
procedure name trace. A description of these traces can be found in 
Appendix~\ref{Traces}.

After the traces have been selected, the program allows the user to
change the term write mode. The term write mode is described by 10
flags. 
They are explained in 
Appendix~\ref{ap:output} and in
{\em A New Parser for \redux\  --- User Guide}
in the section about the parser demo program. 

\item {\bf Display Data Type (d)}\\
The program displays the current data type. If the Knuth-Bendix
completion procedure has been applied or AC-extension rules have been
computed, the current data type may be different from the initial data
type that was fed in at the program start.

\item {\bf Compute Critical Pairs (c)}\\
The user is prompted to enter the numbers of two axioms in the data
type. If an axiom (two terms separated by {\tt ==}) shall be read from
the terminal instead of taken from the data type, zero must be entered
instead of an axiom number.  The the user must decide if the computed
critical pairs are to be normalized w.\,r.\t.\ the current data type
before they are displayed.

Then the program computes all critical pairs between the two given
axioms.
The computed critical pairs are printed, together with their origins and their
weights. 

\item {\bf Run (AC)-Knuth-Bendix Completion Procedure (k)}\\
See Section~\ref{AC-KBCompletion} for a description of the
AC-Knuth-Bendix completion. A term ordering must be installed and selected
if the Knuth-Bendix algorithm shall be used in automatic mode.        

\item {\bf Prove Equational Theorem (p)}\\
This item allows to check if an axiom is an equational consequence of
the current data type. The user is asked to enter an axiom (two terms
separated by {\tt ==}). Then the program computes representatives of
the normal forms for the two terms of the axiom and investigates if the
resulting terms are equal modulo the unification states.  The result of
this comparison is printed.  If the terms are equal, the program
displays a representative of the normal form, otherwise both normal
forms are displayed.

To yield a reasonable result, the current data type must be canonical
before this menu item is called.

\item {\bf (De-)Install Term Ordering (i)}\\
This item allows the user to install or de-install term orderings. See 
Section~\ref{InitializationOfOrderings} for more information on the
term orderings.

\item {\bf Select Term Ordering (s)}\\
The user can select a lexicographical combination of the installed
term orderings and the straight and reverse pseudo orderings.
To install term orderings, use the previous menu item.

\item {\bf Order Axioms of Data Type (o)}\\
This item allows the user to order the axioms of the data type 
according to the selected ordering.

\item {\bf Order Axioms (O)}\\
This item allows to order an arbitrary axiom that is entered by the user.
The axiom must be entered in the form $l${\tt ==}$r$, where $l$ and $r$
are the left-hand side and the right-hand side of the axiom. $l$ and $r$
are two arbitrary terms built of the constants, variables and operators
that have been defined in the data type.

\item {\bf Show Times and Counters (S)}\\
The counters (number of matches, reductions and unifications) and the 
the timers (cpu-time used so far, cpu-time spent for matching,
reduction and unification) are displayed.
The reduction time includes the matching time, so the sum
of the three latter times is greater than the cpu-time spent so far.)

\item {\bf Reset Times and Counters (r)}\\
The counters and timers are reset, either altogether or selectively.

\item {\bf Help, Print Menu (h)}\\
This item prints the main menu. It can be useful if the program
displays the short form (see Section~\ref{ShortMainMenu}) of the main menu.
The same action is performed if only $\langle$return$\rangle$ is struck.

\item {\bf Quit (q)}\\
The program can be stopped by selecting this menu item.
\end{itemize}

\subsection{Running the Peterson-Stickel Procedure}
\label{AC-KBCompletion}
By choosing the menu item ``run (AC)-Knuth-Bendix completion procedure'',
the user can run the Peterson-Stickel 
AC-completion procedure given in \cite{PetersonStickel:81}.
First the user is asked if a subset of the data type is known to be
canonical. 
In this case, the first phase of the completion procedure can be omitted.
If the user answers with {\tt y}, he must enter a list
consisting of the axiom numbers of the axioms known to be canonical.
The list is entered as a comma-separated sequence of numbers
and ranges (two numbers separated by a minus sign) within
a pair of parentheses, i.\,e., {\tt (1-4,7)} is a correct list
specification if the data type contains at least 7 axioms. 
When the list has been entered, the program asks if the rule priority
should be preserved.
If the rule priority is preserved, the axioms known to be canonical are
put in the list of rewrite rules in the order of their occurrence in
the data type, otherwise they are sorted according to their left-hand
side's size.\footnote{The order of the axioms determines their priority
in the normalization process.}

Then the user must specify the completion strategy control parameters.
To describe the completion strategy control parameters we give a sketch
of the AC-completion procedure in Figure~\ref{fi:accmpl}.
Figure~\ref{fi:accmpl}.
\begin{figure}[ht]
\begin{center}
\fbox{
\begin{minipage}{5.5in} %necessary if there is a footnote within the figure
\begin{quote}
\begin{center} ${\cal R} \leftarrow$ {\bf AC-COMPLETE}$({\cal E},\succ)$
\end{center}
[Completion procedure. \\
$\cal E$ is a set of equations and $\succ$ is a terminating term ordering.
Then ${\cal R}$ is a canonical term rewriting system with
\( =_{\cal R} \; = \; =_{\cal E} \).]
\begin{description}
 \item[{(1) [Initialize.]}] \(  {\cal R} := \emptyset \).
 \item[{(2) [Simplify.]}] {\bf while} the {\em simplify}-inference rule
   applies {\bf do} \\
  \( ({\cal E};{\cal R}) := Simplify(({\cal E};{\cal R})). \)
 \item[{(3) [Delete.]}] {\bf while} the {\em delete}-inference rule
   applies {\bf do} \\
  \( ({\cal E};{\cal R}) := Delete(({\cal E};{\cal R})). \)
 \item[{(4) [Stop?]}] {\bf if} \(  {\cal E} = \emptyset \) {\bf then return}
      ${\cal R}$ and {\bf stop}.
 \item[{(5) [Orient.]}] Let \( a \leftrightarrow b \in {\cal E} \) then
      \( {\cal E} := {\cal E} \setminus \{ a \leftrightarrow b \} \); \\
      {\bf if} \( a \succ b \) {\bf then} \(  \bgn l := a; r  := b \nd \) \\
      {\bf elsif} \( b \succ a \) {\bf then} \(  \bgn l := b; r  := a \nd \) \\
      {\bf else stop} with failure; \\
      \( {\cal R} := {\cal R} \cup \{ l \rightarrow r \} \).
 \item[{(6) [Extend.]}] 
    \( {\cal R} := {\cal R} \cup \{ l \rightarrow r \} \; \cup \;
    \) AC-extension of $l \rightarrow r$
 \item[{(7) [Collapse.]}] {\bf while} the {\em collapse}-inference rule
     applies {\bf do} \\
    \( ({\cal E};{\cal R}) := Collapse(({\cal E};{\cal R})). \)
 \item[{(8) [Subconnectedness criterion.]}] \null \\
    \( {\cal E} := {\cal E} \setminus \{ \mbox{Equations originating from a
    collapsed rule} \} \).
 \item[{(9) [Compose.]}] {\bf while} the {\em compose}-inference rule
   applies {\bf do} \\
   \( ({\cal E};{\cal R}) := Compose(({\cal E};{\cal R})). \)
 \item[{(10) [Deduce.]}] Compute all (generalized) $\cal R$-normalized
      AC-critical pairs ${\cal P}$ of     $l \rightarrow r$ and rules
    in $\cal R$. 
 \item[{(11) [Subsumption test.]}] Delete pairs from $\cal P$ that are
      subsumed by other pairs in $\cal P$.
 \item[{(12) [Update $\cal E$.]}]
        \( {\cal E} := {\cal E} \cup {\cal P} \)\footnote{Here we
                                             identify pairs and equations.};
     {\bf continue} with step 2. \hfill $\Box$
\end{description}
\end{quote}
\end{minipage}
}
\end{center}
\caption{Algorithm {\em AC-COMPLETE}} \label{fi:accmpl}
\end{figure}

\begin{quote}
\begin{verbatim}
Select completion strategy control parameters.
Do you wish to use
  d - default settings, 
      (or previous settings for symmetrization resp.)
  q - a quick query or
  v - a verbose query? 
Enter your choice [d/q/v]. *
\end{verbatim}
\end{quote}
\begin{itemize}
\item
If the {\bf verbose query} is selected, the program prints a question
for each of the parameters.
\begin{itemize}
 \item The user may determine whether he wants to orient rules manually
        or using the term ordering.
       In the second case he can chose whether he wants to orient
       non-orientable rules manual (and take advantage of the
       display, trace and exit options) or automatically in step~5.
       In step mode, each critical pair must be oriented manually. 
       In automatic mode, the programs tries to orient critical pairs
       and only if the ordering algorithms fail, the user is asked to
       choose how to proceed. 
       In strict automatic mode, non-orientable
       critical pairs are put to the end of the critical pair queue.
       The last option risks non-termination.
 \item The user may determine whether the input equations shall be
       selected first in step~5  or not.
 \item The user may determine whether only the selected equation or
        all equations in $\cal E$ shall be  kept normalized in step 2.
 \item The user may determine whether collapsed rules are to be
       considered for orientation with priority or whether they are 
       to be inserted  into $\cal E$ using the standard strategy.
 \item The user may choose to apply the subconnectedness criterion in
       step~8 or not.
 \item The user may choose to compute standard critical pairs or
       generalized critical pairs \cite{Buendgen:94}.
 \item The user may choose to delete subsumed critical pairs in step~11
       or not.
       Note that this is not compatible with the subconnectedness
       criterion of step~8. So steps 8 and 11 may not be used
       simultaneously.

\end{itemize}

\item
In the {\bf quick query}, the same information as in the verbose query
must be entered, but in a more compact format: The input consists of a
list of 7 numbers.  The first number is 0 if the the completion shall
run in step mode, 1 if it shall run in automatic mode, and 2 if it
shall run in strict automatic mode. The other numbers are 1 if the
corresponding question was answered by ``yes'', 0 otherwise.
(The control parameters are printed in the compact format at the
beginning of the completion.)

\item
By choosing the {\bf default settings}, the completion control parameters
are set to (1,1,1,0,1,0,0).
\end{itemize}

After those parameters have been entered, the completion begins.  The
interactive input of the completion algorithm is similar to that of the
conventional Knuth-Bendix algorithm. Therefore
Section~\ref{KBCompletion} may be referred to for more detailed
information on the completion process.

In contrast to the other completion algorithms in the \redux\  system, the
fertilization procedure is implemented in this algorithm.  It can be
selected by typing {\tt f} when the program asks how to orient an axiom
or theorem.

If the fertilization was selected, the user must first orient the
considered critical pair:
\begin{quote}
\begin{verbatim}
 How do you want to orient the following critical pair
( 0 and  2) ( 0)  +(I(X),X) == 0
      A - automatic ordering
      S - straight
      R - reverse
      Q - quit fertilization
 Input character [A/S/R/Q]:  *
\end{verbatim}
\end{quote}
He can choose to orient the critical pair straight or reverse.  If a
term ordering has been selected, this ordering can be applied by typing
{\tt A} or {\tt a}.  {\tt Q}/{\tt q} stops the fertilization process.

If the critical pair was extended with an AC-extension rule to ensure
AC-compatibility, the user can choose if he wants to use the unextended
rule, the extension rule or both rules to compute critical pairs.  He
can also leave the fertilization procedure by typing {\tt 0}:

\begin{quote}
\begin{verbatim}
 The critical pair is turned into a rule with extension rule:
 [0] +(I(X),X) == 0;
  AC-ext: +(x,I(X),X) == x; Which of the following shall be used for
critical p
air computation?
      1 - only unextended rule
      2 - only extension rule
      3 - both
      0 - quit fertilization
 Input number [0-3]:  *
\end{verbatim}
\end{quote}

Then the user must choose the number of the (new) axiom with which critical 
pairs shall be created.
A list of all axioms can be obtained by typing {\tt -1}.
If {\tt -2} is entered, the program computes the intrinsic critical
pairs, i,\,e.~the critical pairs that can be obtained by matching
the left-hand side of the considered rule with itself.
The fertilization can be quit by typing {\tt 0}.
\begin{quote}
\begin{verbatim}
 Which axiom do you want to create critical pairs with?
       n -  Axiom number n
      -1 -  List axioms
      -2 -  Intrinsic pair
       0 - quit fertilization
 Input integer:  *
\end{verbatim}
\end{quote}

Now the program tries to create critical pairs between the given
rules.  If no unresolvable critical pair can be created, the user must
choose another axiom to create critical pairs with, otherwise he must
select one of the created critical pairs:

\begin{quote}
\begin{verbatim}
 2 critical pairs were created.
 Choose a critical pair in order to continue the
 completion with! You have the following choices: 
       T - Take the presented critical pair
       N - go on to Next critical pair
       S - Skip 5 critical pairs
       H - Help, this menu
       Q - Quit fertilization
 Beginning presentation.
 Presenting  ( 0 and  0) ( 0)
                  x == I(I(x))
 Input choice [T)ake, N)ext, S)kip, H)elp, Q)uit]  *
\end{verbatim}
\end{quote}

\begin{itemize}
\item Take the presented critical pair (T/t)\\
The considered critical pair is added to the list of theorems.  The
program continues with the completion procedure. The next theorem that
has to be oriented by the user is the chosen critical pair in the
fertilization procedure.

\item Go on to the next critical pair (N/n)\\
The next of the created critical pairs is displayed, together with a
short version of the menu above. If the last critical pair has been
reached, the program continues with the first one.

\item Skip 5 critical pairs (S/s)\\
The next 5 critical pairs are displayed, together with the short
version of the menu above. The user can select the fifth of the printed
critical pairs.

\item Help (H/h)\\
The full menu is printed. This can be useful if the short version of
the menu is displayed.

\item Quit fertilization (Q/q)\\
The fertilization procedure stops, the program continues with the
completion procedure. The current theorem is the same as before the
fertilization procedure.
\end{itemize}
