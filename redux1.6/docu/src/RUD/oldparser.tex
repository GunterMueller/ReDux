\section{The Old Parser}
\label{OldParser}
In this Appendix, the old \redux\  specification parser is described. Versions
of the \redux\  demo programs that use the old parser can be found
in subdirectory {\tt olddemo} in the \redux\  main directory.

\subsection{Algebraic Specifications}
The input file  $\langle$prog$\rangle${\tt .in} must contain a one or 
many sorted algebraic specification.
The algebra name, constants and variables with associated type (sort) belong 
to that specification. All function symbols (operators) must be 
listed together with their domains and ranges, all constants and variables 
together with their types. 
And, last but not least, we need the set of axioms.
In Figure~\ref{fi:syn.old}, we present the syntax of an algebraic specification in 
extended BNF.
It is the same given in \cite{Kuechlin:82a} and 
\cite{Buendgen:87} with the addition of properties assigned to the
operators to denote unification states 
(for the {\tt ac} program),
operator notation (prefix, infix, postfix, roundfix and
Lisp notation) and to supply additional information
to deal with external objects.

If the first symbol of the data type specification file is not
{\tt TYPE} then this symbol is declared as comment symbol and may be used
to declare all characters to the right of the comment symbol to be 
a comment in the remaining input.
This comment symbol is then known to the rest of the program and
may be used whenever terms or theorems are read in.
If the first symbol of the data type specification file is {\tt TYPE} then
no comment symbol is available.
\begin{figure}[htbp]
\begin{center}
\fbox{
\begin{minipage}{5.9in} %necessary if there is a footnote within the figure
\begin{tabbing}
ConstantName\hspace{\tabbingsep}\=::=\hspace{\tabbingsep}\=\kill
\>AlgebraSpec\'::=\>NameSpec ``{\tt.}'' ExtSpec ConstSpec VarSpec OperatorSpec
      RuleSpec ``{\tt END}''\\
\>NameSpec\'::=\>``{\tt TYPE}'' AlgebraName\\
\>ExtSpec\'::=\>$\varepsilon$ $|$ ``{\tt EXTERNAL}'' Externals \{ ``{\tt.}''
      Externals \}$^\ast$\\
\>Externals\'::=\>TypeName ``{\tt XREAD:}'' AlgorithmName
                               ``{\tt XWRITE:}'' AlgorithmName\\
\>    \>      ``{\tt  XEQ:}'' AlgorithmName ``{\tt  XLT:}'' AlgorithmName \\
\>ConstSpec\'::=\>$\varepsilon$ $|$ ``{\tt CONSTS}'' Constants \{ ``{\tt.}''
      Constants \}$^\ast$\\
\>Constants\'::=\>ConstantName \{ ``{\tt ,}'' ConstantName \}$^\ast$
      ``{\tt -}'' Typename\\
\>VarSpec\'::=\>$\varepsilon$ $|$ ``{\tt VARS}'' Variables \{ ``{\tt .}''
      Variables \}$^\ast$\\
\>Variables\'::=\>VariableName \{ ``{\tt,}'' VariableName \}$^\ast$
      ``{\tt -}'' Typename\\
\>OperatorSpec\'::=\>$\varepsilon$ $|$ ``{\tt OPS}'' Operator \{ ``{\tt .}''
      Operator \}$^\ast$\\
\>Operator\'::=\>OperatorName ``{\tt(}'' Domain ``{\tt )}'' ``{\tt -}''
      Range PropertyList\\
\>Domain\'::=\>$\varepsilon$ $|$ Typename \{ ``{\tt,}'' Typename \}$^\ast$\\
\>Range\'::=\>Typename\\
\>PropertyList\'::=\>\{ Indicator ``{\tt :}'' PropertyValue \}$^\ast$\\
\>Indicator\'::=\>{\tt USTAT}
%      \footnote{The unification state
%      {\tt AC} indicates an associative and commutative (binary)
%      operator, the unification state {\tt COM} indicates a commutative
%      operator. In the future, this may 
%      be extended to other unification states.}
      $|$ {\tt FIX} $|$ {\tt ROUND} $|$ {\tt XINT}\\
\>RuleSpec\'::=\>$\varepsilon$ $|$ ``{\tt AXIOMS}'' \{ Rule \}$^+$\\
\>Rule\'::=\>Number ``{\tt )}'' Term ``{\tt ==}'' Term\\
\>ArgumentList\'::=\>Term \{ ``{\tt ,}'' Term \}$^\ast$\\
\>LispArgs\'::=\>$\varepsilon$ $|$ Term \{ ``$\null\;\;\null$''
               Term \}$^\ast$ \\
\>Term\'::=\>ConstantName $|$ VariableName $|$
          PrefixTerm $|$ PostfixTerm $|$ \\
\>        \>InfixTerm $|$ RoundfixTerm $|$ OperatorName $|$ LispTerm \\
\>PrefixTerm\'::=\>OperatorName ``{\tt (}'' ArgumentList ``{\tt )}''\\
\>PostfixTerm\'::=\>``{\tt (}'' ArgumentList ``{\tt )}'' OperatorName\\
\>InfixTerm\'::=\>``{\tt (}'' Term OperatorName Term ``{\tt )}''\\
\>RoundfixTerm\'::=\>OperatorName  ArgumentList OperatorName\\
\>LispTerm\'::=\> ``{\tt (}'' OperatorName LispArgs ``{\tt )}''\\
\>PropertyValue,\'\>\\
\>OperatorName\'::=\>SpecialSymbol \{ Digit $|$ SpecialSymbol $|$
          Letter \}$^\ast$ $|$ \\
\>        \> Letter \{ Digit $|$ Letter \}$^\ast$ $|$ Digit$^+$ \\
\>ConstantName\'::=\>\{ Digit \}$^+$ $|$ Letter \{ Letter $|$
          Digit \}$^\ast$\\
\>AlgebraName,\'\>\\
\>TypeName,\'\>\\
\>VariableName\'::=\>Letter \{ Letter $|$ Digit \}$^\ast$ \\
\>Number\'::=\>\{ Digit \}$^+$ \\ 
\>SpecialSymbol\'::=\>
         ``\verb+=+'' $|$ ``\verb-+-'' $|$ ``\verb+*+'' $|$
         ``\verb+/+'' $|$ ``\verb+\$+'' $|$ ``\verb+!+'' $|$
         ``\verb+"+'' $|$ ``\verb+\#+'' $|$ ``\verb+\%+'' $|$
         ``\verb+\&+'' $|$ ``\verb+'+'' $|$ ``\verb+;+'' $|$
         ``\verb+<+'' $|$ \\
\> \>    ``\verb+>+'' $|$ ``\verb+?+'' $|$
         ``\verb+@+'' $|$ ``\verb+[+'' $|$ ``\verb+]+'' $|$
         ``\verb+$\wedge$+'' $|$  ``\verb+\_+'' $|$ ``\verb+`+'' $|$
         ``\verb+\{+'' $|$ ``\verb+|+'' $|$ ``\verb+\}+'' $|$
         ``\verb+$\sim$+'' \\
\>Digit\'::=\>
         ``{\tt 0}'' $|$ $\ldots$ $|$ ``{\tt 9}'' \\
%         ``{\tt 0}'' $|$ ``{\tt 1}'' $|$ ``{\tt 2}'' $|$ 
%         ``{\tt 3}'' $|$ ``{\tt 4}'' $|$ ``{\tt 5}'' $|$ 
%         ``{\tt 6}'' $|$ ``{\tt 7}'' $|$ ``{\tt 8}'' $|$
%         ``{\tt 9}'' \\
\>Letter\'::=\>
         ``{\tt a}'' $|$ $\ldots$ $|$ ``{\tt z}'' $|$  
         ``{\tt A}'' $|$ $\ldots$ $|$ ``{\tt Z}'' \\
%         ``{\tt a}'' $|$ ``{\tt b}'' $|$ ``{\tt c}'' $|$
%         ``{\tt d}'' $|$ ``{\tt e}'' $|$ ``{\tt f}'' $|$
%         ``{\tt g}'' $|$ ``{\tt h}'' $|$ ``{\tt i}'' $|$
%         ``{\tt j}'' $|$ ``{\tt k}'' $|$ ``{\tt l}'' $|$
%         ``{\tt m}'' $|$\\
%\> \>    ``{\tt n}'' $|$ ``{\tt o}'' $|$
%         ``{\tt p}'' $|$ ``{\tt q}'' $|$ ``{\tt r}'' $|$
%         ``{\tt s}'' $|$ ``{\tt t}'' $|$ ``{\tt u}'' $|$
%         ``{\tt v}'' $|$ ``{\tt w}'' $|$ ``{\tt x}'' $|$
%         ``{\tt y}'' $|$ ``{\tt z}'' $|$ \\
%\> \>    ``{\tt A}'' $|$ ``{\tt B}'' $|$ ``{\tt C}'' $|$
%         ``{\tt D}'' $|$ ``{\tt E}'' $|$ ``{\tt F}'' $|$
%         ``{\tt G}'' $|$ ``{\tt H}'' $|$ ``{\tt I}'' $|$
%         ``{\tt J}'' $|$ ``{\tt K}'' $|$ ``{\tt L}'' $|$
%         ``{\tt M}'' $|$ \\
%\> \>    ``{\tt N}'' $|$ ``{\tt O}'' $|$
%         ``{\tt P}'' $|$ ``{\tt Q}'' $|$ ``{\tt R}'' $|$
%         ``{\tt S}'' $|$ ``{\tt T}'' $|$ ``{\tt U}'' $|$
%         ``{\tt V}'' $|$ ``{\tt W}'' $|$ ``{\tt X}'' $|$
%         ``{\tt Y}'' $|$ ``{\tt Z}'' \\
\end{tabbing}
\end{minipage} 
}
\end{center}
\caption{The Syntax of the input format of \redux\ data types for the old
specification parser} \label{fi:syn.old}
\end{figure}


%AlgebraName, ConstantName, VariableName and TypeName  are
%usual identifiers. An OperatorName consists of identifiers and the special 
%symbols listed below.
%\begin{quote}
%``\verb+=+'', ``\verb-+-'', ``\verb+*+'', ``\verb+/+'', ``\verb+$+'',
%``\verb+!+'', ``\verb+"+'', ``\verb+#+'', ``\verb+%+'', ``\verb+&+'',
%``\verb+'+'', ``\verb+;+'', ``\verb+<+'', ``\verb+>+'', ``\verb+?+'',
%``\verb+@+'', ``\verb+[+'', ``\verb+]+'', ``\verb+^+'', ``\verb+_+'',
%``\verb+`+'', ``\verb+{+'', ``\verb+|+'', ``\verb+}+'' and ``\verb+~+''. 
%\end{quote}
%A ConstantName can also be a nonnegative integer.
%A number is a positive integer to enumerate the rules. 

In the current implementation the term-syntax has the
following limitations that will be removed in a later revision.
Operator names may start with a special symbol and end with an
identifier or digit and on the other hand may start with an identifier
and end with an identifier or digit, therefore
it is impossible for the parser to uniquely interpret a term
\verb+(A*B)+ as  either \verb+(A * B)+ or \verb+(A *B ...)+, where
``{\tt *}'' is an infix operator. Therefore infix operators must be
separated by blanks from their arguments.
Roundfix Operators also have to be separated by blanks from their argument
list. Otherwise \verb+<A,B>+ might be interpreted as \verb+<A,+\ldots
instead of as \verb+< A,B >+, where ``{\tt <}'' and ``{\tt >}'' form a
roundfix operator. A later
revision might restrict operator names to consist either only of
special symbols or of identifiers and digits with a leading
identifier. 

The following 
semantic constraints are given: the argument list of an operator in a term must
be as long as the type list of its domain and the arguments must be of the
appropriate type. Also the name of an operator variable or constant
may not be the same as another operator's, variable's or constant's name.
Furthermore, no left-hand side of a Rule may consist only of a ConstantName
or a VariableName. If a constant is needed as a left-hand side of a
rule, a nullary operator can be used instead. 

After reaching the {\tt END} symbol, the programs stop reading the
algebraic specification. This feature can also be used to put 
comments at the end of algebraic specification files.

Algebraic specifications are called {\em data types} in the \redux\  system.
The data type describing lists over the elements 0 and 1 can be described as
shown in Figure~\ref{fi:exa.old}.

\begin{figure}[htbp]
\begin{center}
%\fbox{
\begin{minipage}{5.9in} %necessary if there is a footnote within the figure
\begin{quote}
\begin{verbatim}
# comment symbol
TYPE LIST. 
# STATUS completed
# AUTHOR Reinhard Buendgen
# DESCRIPTION List({0,1}): cons, app, rev
# ORIGIN  folklore
CONSTS 
   0,1-EL. 
   nil-LIST 
VARS 
   A,B,C-EL. 
   L,M,N-LIST
OPS 
   @(LIST,LIST)-LIST   FIX: INFIX.              # append 
   [(EL,LIST)-LIST  FIX: ROUNDFIX ROUND: ].     # cons
   rev(LIST)-LIST                               # reverse
AXIOMS
   1) (nil @ L) == L
   2) ([ A,L ] @ M) == [ A,(L @ M) ]
   3) rev(nil) == nil
   4) rev([ A,L ]) == (rev(L) @ [ A,nil ])
END
\end{verbatim}
\end{quote}
\end{minipage}
%} %end fbox
\end{center}
\caption{An example of a \redux\  data type specification} \label{fi:exa.old}
\end{figure}
The name of the data type is {\tt LIST}, the constants {\tt 0} and {\tt 1}
are of sort {\tt EL} and {\tt NIL} is of sort {\tt LIST}.
There are variables of sort {\tt EL} ({\tt A, B, C}) and of sort {\tt LIST}
({\tt L, M, N}) and operators {\tt APP, CONS} and {\tt REV} with range sorts 
{\tt LIST} $\times$ {\tt LIST}, {\tt EL} $\times$ {\tt LIST} and
{\tt LIST} respectively, and result sorts {\tt LIST}.

\subsection{Operator Properties}
\label{OperatorProperties}
The four property indicators have the following meaning:

\begin{description}
\item[USTAT:] This indicator is used in conjunction with the
values {\tt AC} and {\tt COM}. The property value {\tt AC} indicates
that the operator is associative and commutative. Therefore it can
only be used with binary operators whose arguments and range 
are of the same type.
The value {\tt COM} is used for commutative operators. It can only be
used with a binary operators whose arguments share the same type.

\item[FIX:]
The indicator {\tt FIX} is used to associate parsing properties
to operators, i.\,e.\ information if the operator ought to be parsed and
written as prefix, postfix, infix, roundfix operator, in LISP style
or as constant (i.\,e.\ without parentheses for the arguments).
See Appendix~\ref{OperatorNotation}.

\item[ROUND:] This indicator is used to indicate the second part of a
roundfix operator. It is used together with {\tt FIX: ROUNDFIX}.
See Appendix~\ref{OperatorNotation}.

\item[XINT:]
The {\tt XINT} indicator is used to adjoin an interpretation to an
operator. The corresponding value is a name for an \ALDES\ algorithm.
See Appendix~\ref{ExternalObjects}.
\end{description}

Each property indicator can be used at most once per operator.

\subsection{Operator Notation}
\label{OperatorNotation}
The standard parser supports operators with four different notations.
The following is a list of the notions and the respective declarations in
$\langle$prog$\rangle${\tt .in}:
\begin{itemize}
\item
Prefix notation: as in {\tt f(r,n)} \\
declaration of ``{\tt f}'' : {\tt f(REAL,NAT) - REAL FIX:PREFIX}\\
Prefix is the standard notation and will be supported if there is no 
``{\tt FIX}'' property added to the declaration of an operator.
\item 
Postfix notation as in {\tt (n)!} \\
declaration of ``{\tt !}'' : {\tt !(REAL) - REAL FIX:POSTFIX}
\item 
Infix notation as in {\tt (n + m)} \\
declaration of ``{\tt +}'' : {\tt +(NAT,NAT) - LIST FIX:INFIX} \\
The infix notation requires a domain type list of length two.
It also does not allow empty terms like {\tt ( + )} as the other notations
support. 
\item
Roundfix notation as in {\tt < v1,v2 >} e.\,g.\  for the scalar product\\
declaration of ``{\tt <\ldots >}'' : 
{\tt <(VEC,VEC) - NAT FIX:ROUNDFIX ROUND:>}
\item Lisp-fix notation as in {\tt (apply f a)} which is used in Lisp syntax\\
declaration of ``{\tt apply}'': 
{\tt apply(LISPOBJ,LISPOBJ) - LISPOBJ FIX: LISP} \\
If in case of an input error the parser requests to reenter the argument list
of an Lisp-fix operator the user is supposed to {\em first enter a blank
character} and then to enter the arguments separated by blanks and 
terminated by a closing parenthesis.

\item Constant notation as in {\tt CONS(A,nil)}\\
declaration of ``{\tt nil}'': {\tt nil() - LIST FIX: CONSTS}\\ 
This operator notation is restricted to nullary operators.
The fixing state {\tt FIX: CONSTS} can be omitted since it is the
default for nullary operators.
\end{itemize}
Associative commutative (AC) and commutative operators can be  combined
with any of the aforementioned notations. 
AC-operators can be parsed in flattened form, that is more then two
arguments are allowed in the argument list.
\begin{example}
If {\tt A}, {\tt B}, {\tt C} and {\tt D} are declared as variables (or
constants) of type {\tt NAT} and the following AC-operators are declared:
\begin{verbatim}
OPS
  *(NAT,NAT)-NAT USTAT: AC FIX: INFIX.
  f(NAT,NAT)-NAT USTAT: AC
\end{verbatim}
Then the term {\tt (f(A,A) * (B * f(C,C,C,C) * D))} is a correct term
and corresponds to the term {\tt (f(A,A) * (B * (f(C,f(C,f(C,C))),D)))}.
\hfill $\Box$
\end{example}
Note that only the AC-rewrite laboratory and the term ordering ({\tt
to}) program use the information that
operators with unification status AC are to be considered 
associative and commutative.
All other programs ignore the unification status except in the term
ordering algorithms.
In particular only these two programs can handle flat terms correctly.

\subsection{Constants vs. Nullary Operators}
Constants may also be defined as nullary operators in
the section `OperatorSpec' of  the grammar:
\begin{quote}
\begin{verbatim}
OPS
        ...
        nil() - LIST.
        ...
\end{verbatim}
\end{quote}
Even though they are declared with an empty argument list,
they are read and written {\em without} argument list inside a term
(e.\,g.\ {\tt nil} or {\tt cons(L,nil)} are correct terms).

If empty argument lists of nullary operators are desired inside a term,
the operator must be declared with an explicit notational property
(e.\,g.\ with property {\tt FIX: PREFIX}, {\tt nil()} is a correct term
and with {\tt FIX: LISP}, {\tt (nil)} is a correct term).

The main difference between constants and nullary operators is the
built-in assumption that constants are irreducible.
Therefore the left-hand side of an equation may not be a constant.
This restriction does not apply to nullary operators.
Thus in \redux\  constants can be handled more efficiently than
operators.
So far only operators may have properties (see
Appendices~\ref{OperatorProperties},
\ref{OperatorNotation} and \ref{ExternalObjects}).

\subsection{External Objects}
\label{ExternalObjects}
Some \redux\  programs allow to manipulate external objects (like
\ALDES\ integers, SAC-2 lists, SAC-2 (long) integers, etc.).
Each of these object classes must be defined in the `ExtSpec' part
of the data type definition by associating a \redux-type name with a set
of properties describing the interface to the external data type:

\begin{description}
\item[XREAD:] Procedure name of a nullary procedure to read an external object.

\item[XWRITE:] Procedure name of a nullary procedure to write
an external object.

\item[XEQ:] Procedure name of a binary boolean function to test the
equality of two external objects.

\item[XLT:] Procedure name of a binary boolean function to test if the
first of two external objects is less than the second one.
\[\vdots\]

\end{description}

\begin{example}
The following external specification allows the import of SAC-2 lists and
SAC-2 integers.
\begin{quote}
\begin{verbatim}
EXTERNAL
  SAC2LIST  XREAD: LREAD  XWRITE: LWRITE  XEQ: EQUAL.
  SAC2INT   XREAD: IREAD  XWRITE: IWRITE  XEQ: EQUAL
\end{verbatim}
\end{quote}
\end{example}

Operators may be assigned a property ({\tt XINT:} {\em AlgorithmName})
which describes the interpretation of the operator as function over the
external objects.

External objects may be introduced into terms via coercion operators.
A {\em coercion operator} is a unary operator whose argument type has been
defined in `ExtSpec' and whose result type is a normal term type (sort).
The argument of an coercion operator is read and written by the
procedures specified in {\tt XREAD} and {\tt XWRITE} properties
of the external type respectively.

Terms containing external objects are called
{\em external terms} or {\em mixed terms}.
A well-formed mixed term \(t=f(c_1(O_1),\ldots, c_n(O_n))\) 
may be computed in one step if \(f\) has been assigned an \(n\)-ary
interpretation algorithm \(A\)
under property {\tt XINT}, the \(c_i\) are coercion operators and the
\(O_i\) are external objects.
Then the computation of \(t\) evaluates to $c(O)$ where $c$ is the
coercion operator mapping the resulting external sort of $A$ to the
result sort of $f$ and $O$ is the result of 
\(A(O_1,\ldots,O_n)\).

Applying these one-step computations as often as possible to all
subterms is called {\em computing} that term.

External objects must not be used in {\tt ic}, {\tt to}, {\tt ts}
and {\tt uc}.
The {\tt ac} program can handle external terms, but in the term ordering
algorithms, external operators
are treated like constants, independent of their argument. Its
main menu contains an item to compute terms with external operators.
In {\tt trd}, external operators have a fixed argument that
must be supplied interactively by the user.
All programs using external objects read in the file {\tt algos}
that contains one entry for each algorithm that can be specified
after {\tt XREAD}, {\tt XWRITE}, {\tt XEQ}, {\tt XLT}, and {\tt XINT}.

