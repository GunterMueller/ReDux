\section{Introduction}
The \redux\  system is a work bench for term rewriting systems, based on 
the (low level modules of the) SAC-2 computer algebra system
(\cite{Collins:80}).
It consists of a large collection of algorithms written in \ALDES\
(\cite{Loos:76}, \cite{LoosCollins:92}) and a set of demo programs which
both show how to use
the \redux\  algorithms and solve some interesting problems in the area of
term rewriting.
This article describes the {\em use} of these programs, i.\,e.\ it
describes how to run the demo programs.
The implementation aspects of the \redux\  demo programs and 
a guide to programming
using the \redux\  algorithms can be found in \cite{RSD1.4}.

\redux\  demo programs manipulate a many sorted algebraic specification.
They allow for experiments with term orderings (Knuth-Bendix-,
polynomial interpretation- and path orderings).
In addition, there are several versions of Knuth-Bendix completion procedures
employing different confluence criteria.
In order to reason about the initial algebra of an algebraic specification
there are tools to analyze the set of ground normal forms and to prove
inductive theorems by inductive completion.
There is also a rewrite laboratory for specifications with commutative and
associative-commutative operators providing unification, matching, reduction 
and completion (\`{a} la Peterson and Stickel) procedures.
Some of the programs work with {\em external terms} which contain ordinary 
\ALDES/SAC-2 objects.
Table~\ref{ta:dp} gives an overview of the existing demo programs.
The UC-system is considered experimental and is not described here.
See for the documentation along with the code.
\begin{table}
 \centering
 \begin{tabular}{|l|l|l|l|l|}
  \hline
  program & sub-   & uses & program description & main \\ 
          & system &      &                     & authors \\
   \hline\hline
   TO & to &  tp & test environment for term orderings: & K\"{u}chlin \\
      &    &     & Knuth-Bendix ordering,               & Schw\"arzler \\
      &    &     & Recursive Path ordering,             & Joswig \\
      &    &     & Polynomial interpretation ordering   & B\"{u}ndgen \\
      &    &     & and lexicographic combinations       & Lauterbach \\
      &    &     & there of                             &  \\
  \hline
   TC & tc & tp, to & plain Knuth-Bendix completion procedure, & B\"{u}ndgen \\ 
      &    &        & Knuth-Bendix completion procedure        & K\"{u}chlin \\
      &    &        & applying subconnectedness criterion,     & \\ 
      &    &        & Knuth-Bendix completion procedure        & \\
      &    &        & applying transformation  criterion,      & \\ 
      &    &        & Knuth-Bendix completion procedure        & \\
      &    &        & applying both the transformation and the & \\
      &    &        & subconnectedness criteria                & \\ 
  \hline
   TS & ic & tc, tp & top set tree and ground normal   & B\"{u}ndgen \\
      &    &        & form grammar computation,        & Eckhardt \\ 
      &    &        & interleaved top set computation  & \\
      &    &        & based on top set trees,          & \\ 
      &    &        & pruned top set tree and ground   & \\
      &    &        & normal form grammar computation  & \\ 
  \hline
   IC  & ic &tc, to, tp & inductive completion laboratory   & B\"{u}ndgen \\
       &    &           & according to (\cite{Kuechlin:89}) & \\ 
  \hline
   AC   & ac &tc, to, tp & rewrite laboratory for term rewriting &B\"{u}ndgen \\
        &    &           & systems with associative and commutative& \\
        &    &           & operators & \\ 
  \hline
   EV   & acn &tc, to, tp & AC-system with support for &B\"{u}ndgen \\
        &    & ev        & partial evaluation domains  &Lauterbach  \\
  \hline
   TRD  & rd & tp, tc & random term generator & Walter \\
  \hline
   UC  & uc & tp, tc & unfailing completion & Hoss \\
       &    &       &                      & Mayer \\
       &    &       &                      & Sinz \\
       &    &       &                      & B\"{u}ndgen \\
  \hline
 \end{tabular}
 \caption{The \redux\  demo programs} \label{ta:dp}
\end{table}

The following description is based on the \redux\  installation at the
Wilhelm-Schickard-Institut at the Universit\"{a}t T\"{u}bingen.
This installation uses an \ALDES\ to C translator and runs on SPARCstation
under Solaris 2.x or SunOS 4.1.x.
Installations on Sun~3/SunOS, PC/Linux and
IBM RS6000/AIX have already been tested, too.
Because of the high portability of \ALDES\ all programs should behave much 
the same on other installations.
The only differences might be observed in the few I/O and operating system
dependent features (file names, shell scripts, etc.).
