\section{General Facts on the Demo Programs}
\subsection{Low Level Input and Output}
The character set available for input and output is the complete ASCII
character set. Characters in the range 0--31 are mapped to the
blank character.

The \redux\  demo programs use the \ALDES\ input units\footnote{The notion
``unit'' is taken from FORTRAN 77.
Originally, \ALDES\ was translated to FORTRAN.
To be compatible with the \ALDES\ to FORTRAN translators, the
I/O-primitives for the \ALDES\ to C translators simulate the
FORTRAN I/O.}
4 and 5. 
Input unit 4 is (by default) associated with the file 
{\it program}{\tt .in} in the current directory 
 where {\it program} is the name of
the \redux\  demo program.
Input unit 5 is (by dehault) the terminal input (in UNIX: stdin).
The \redux\  demo programs use \ALDES\ output units 6 and 0.
Output unit 6 is the terminal output (in UNIX: stdout),
output unit 0 is the error output stream (in UNIX: stderr).
The basic input and output routine {\tt IO} records
all terminal input and output to a log-file that is by default called
{\it program}{\tt .log}.

\subsection{Program Start}
\label{GeneralProgramStart}
The \redux\  demo programs {\tt ic}, {\tt to} and {\tt ts}
can be started by typing their name at the 
operating system prompt. 
In this case,  they read a {\em data type}\/\footnote{
A \redux\  data type is an equational specification, consisting of a
signature description and a set of equations.}
from the file {\it program}{\tt .in} in the working
directory and allow the user to work interactively with the 
algorithms of the \redux\  system. 
If the data type is contained in another file, the name of this file can be
given as command line option, i.\,e.\ the program {\tt ic} would
be invoked by typing {\tt ic spec/group.rdx} if the data type
file is {\tt spec/group.rdx}.

Some other demo programs (tc, ac, ev and trd) need a second file,
the so-called {\em algos-file\/}: It contains information concerning SAC--2
and \redux\  algorithms that are used by the \redux\  interpreter module.
The default name for this file is {\it program}{\tt .algos}.
If another algos-file is to be used, its name must be given as a
second command line argument, so that the program {\tt ac} with the
previous data type file and algos file {\tt
/usr/local/redux/src/it/algos} would be invoked with the command
\begin{quote}
{\tt ac spec/group.rdx /usr/local/redux/src/it/algos}.
\end{quote}
In either case, all terminal input and output is recorded in a log-file.
The name of the log-file is by default.
{\it program}{\tt .log} in the working directory.
The default name of the log-file may be overwritten by a second 
(if no algos file is need) or a third (if an algos file is needed) command
line argument.
For exact specification of the command line syntax of \redux\ programs
the the manual pages.

%For convenience, there are shell scripts (each shell script has the same
%name as the corresponding demonstration program but is spelled in upper
%case letters) that can be started by typing 
%\begin{quote}
% {\em shell-script-name}  
%    [{\em file-name} $|$ {\tt -D} [{\em directory-name} ]].
%\end{quote}

%\begin{itemize}
%\item
%If one of the shell scripts is called without arguments, it directly
%starts the appropriate demonstration program.
%If the data type file {\it program}{\tt .in} does not exist, the shell
%script issues an error message and aborts.

%\item
%If the argument {\em file-name} is supplied, the shell script searches
%for files named {\em file-name} in the global \redux\  specification 
%directory and its subdirectories.
%If the command line parameter {\em file-name} does not has the
%extension {\tt .rdx}, this suffix is appended before the search is done.
%For each of the found files, the user is asked if this data type
%shall be used. A symbolic link with name {\em program\tt .in}
%to the requested file is created and the demo program is started.

%\item
%When a script is called with the option {\tt -D}, the script searches
%in the directory named {\em directory} (default directory is the global
%\redux\  specification directory) for files with the file name
%extension {\tt .rdx}. A list of all data types found is displayed
%(piped through {\tt more}) and the user is prompted to enter one of
%these data type names. A symbolic link to the chosen file is created
%and the demo program is started.
%\end{itemize}
%If the demo program needs an algos file and there is a file called
%{\em algos} in the current directory, this file is used.
%Otherwise the global \redux\  algos file is used. 

\subsection{Interactive Input}
After reading a data type from the file {\it program}{\tt .in}, 
the programs may be controlled by interactive input.
During a dialogue, an asterisk ({\tt *}) in the
output serves as a prompt for user interaction.
Every interactive input must be followed by $\langle$return$\rangle$.
Terms and axioms must be terminated by a semicolon.

The program is either menu directed or a specification of the needed input 
data is given.
Input options are given in brackets [\,] separated by slashes.
%Letters in curly braces \{\,\} (separated by commas) indicate that a 
%string of the given letters must be entered. 
%This is done by concatenating some of the given letters.
Typing $\langle$return$\rangle$ without any of the given letters means that
a default value has been chosen.
The questions `{\tt READY *}', `{\tt continue *}'  or just `{\tt *}' must
be answered by hitting the $\langle$return$\rangle$ key.

\subsection{Algebraic Specifications}
This section gives a short summary of the \redux\  specification
parser.\footnote{A description of the old \redux\  specification parser
that was the only parser up to \redux\  version~1.4 can be found in
Appendix~\ref{OldParser}.}
A more thorough treatise on the \redux\  parser can be found in
{\em A New Parser for \redux\  --- User Guide}.

The input file  {\it program}{\tt .in} must contain a one- or 
many sorted algebraic specification.
This specification contains information about the data type
name, sorts, constants, variables, function symbols and the set of
axioms.
All function symbols (operators) must be 
listed together with their domains and ranges, all constants and variables 
together with their types. 
For the operators, information on the fixity, on precedence and
associativity and on unification states can be added.
In Figures~\ref{dtgram1}, \ref{dtgram2} and \ref{axgram},
we present the syntax of an algebraic specification in extended BNF.
\begin{figure}[htbp]
\begin{center}
\noindent
\rule{\textwidth}{0.6pt}

%\fbox{
\begin{minipage}{6in}
\begin{tabbing}
{\vs PrecedenceDecl} \= \gprod \= \kill
{\vs DataType} \> \gprod \ts{DATATYPE} {\vs DTName} \ts{;} \gstar{{\vs Declaration}} \ts{END} \\
\\
{\vs Declaration} \> \gprod {\vs SortDecl} \galt {\vs ExternalDecl}
  \galt {\vs ConstantDecl} \galt {\vs VariableDecl} \\
  \>\> \galt{\vs OperatorDecl} \galt {\vs CoercionDecl} \galt {\vs PrecedenceDecl} \galt
       {\vs AssocDecl} \\
  \>\> \galt {\vs NotationDecl} \galt {\vs TheoryDecl} \galt {\vs PropertyDecl}
       \galt {\vs AxiomDecl} \\
{\vs SortDecl} \> \gprod \ts{SORT} {\vs SortNames} \ts{;} \\
{\vs ExternalDecl} \> \gprod \ts{EXTERNAL} \gplus{{\vs External} \ts{;}} \\
{\vs ConstantDecl} \> \gprod \ts{CONST} \gplus{{\vs Constant} \ts{;}} \\
{\vs VariableDecl} \> \gprod \ts{VAR} \gplus{{\vs Variable} \ts{;}} \\
{\vs OperatorDecl} \> \gprod \ts{OPERATOR} \gplus{{\vs Operator} \ts{;}} \\
{\vs CoercionDecl} \> \gprod \ts{COERCION} \gplus{{\vs Coercion} \ts{;}} \\
{\vs PrecedenceDecl} \> \gprod \ts{PREC} \gplus{{\vs Precedence} \ts{;}} \\
{\vs AssocDecl} \> \gprod \ts{ASSOC} \gplus{{\vs Associativity} \ts{;}} \\
{\vs NotationDecl} \> \gprod \ts{NOTATION} \gplus{{\vs Notation} \ts{;}} \\
{\vs TheoryDecl} \> \gprod \ts{THEORY} \gplus{{\vs Theory} \ts{;}} \\
{\vs PropertyDecl} \> \gprod \ts{PROPERTY} \gplus{{\vs Property} \ts{;}} \\
{\vs AxiomDecl} \> \gprod \ts{AXIOM} \gplus{{\vs Axiom} \ts{;}} \\
\\
{\vs External} \> \gprod {\vs XSortNames} \ts{:} {\vs XInterface} \gstar{\ts{,} {\vs XInterface}} \\
{\vs XInterface} \> \gprod \group{\ts{XREAD} \galt \ts{XWRITE} \galt \ts{XEQ} \galt \ts{XLT}
  \galt \ts{XFG} \galt \ts{XTERM}} \ts{=} {\vs AlgoName} \\
{\vs Constant} \> \gprod {\vs ConstNames} \ts{:} {\vs SortName} \\
{\vs Variable} \> \gprod {\vs VarNames} \ts{:} {\vs SortName} \\
{\vs Operator} \> \gprod {\vs OpNames} \ts{:} {\vs DomainList} \ts{->} {\vs Range} \\
{\vs DomainList} \> \gprod {\vs Domain} \gstar {\ts{,} {\vs Domain}} \galt \gepsilon \\
{\vs Coercion} \> \gprod {\vs CoercionOpName} \ts{:} {\vs Domain} \ts{->} {\vs Range} \\
{\vs Precedence} \> \gprod {\vs OpNames} \gplus{\group{\ts{<} \galt \ts{=}} {\vs OpNames}} \\
{\vs Associativity} \> \gprod {\vs OpNames} \ts{:} 
  \group{\ts{LEFT} \galt \ts{RIGHT} \galt \ts{NONE}} \\
{\vs Notation} \> \gprod {\vs OpNames} \ts{:} {\vs Fixity} \\
{\vs Fixity} \> \gprod \ts{PREFIX} \galt \ts{POSTFIX} \galt \ts{INFIX} \galt \ts{CONSTANT} \\
  \>\> \galt \ts{FUNCTION} \galt \ts{LISP} \galt \ts{ROUNDFIX} {\vs ClosingRoundfixName} \\
{\vs Theory} \> \gprod {\vs OpNames} \ts{:} \group{\ts{AC} \galt \ts{COM}} \\
{\vs Property} \> \gprod {\vs ObjectNames} \ts{:} {\vs PropList} \\
{\vs PropList} \> \gprod {\vs Prop} \gstar{\ts{,} {\vs Prop}} \\
{\vs Prop} \> \gprod {\vs Indicator} \ts{=} {\vs Value} \\
{\vs Indicator} \> \gprod \ts{XINT} \galt \ts{KBINDEX} \galt \ts{KBWEIGHT} \galt \ts{COERC}
   \galt \ts{XFG} \galt \ts{HASH} \galt {\vs Ident}\\
\end{tabbing}
\end{minipage}

\noindent
\rule{\textwidth}{0.6pt}
%} % end fbox
\caption{The Syntax of \redux\  Data Types, Part 1}
\label{dtgram1}
\end{center}
\end{figure}

\begin{figure}[htbp]
\begin{center}
\noindent
\rule{\textwidth}{0.6pt}
 
%\fbox{
\begin{minipage}{6in}
\begin{tabbing}
{\vs PrecedenceDecl} \= \gprod \= \kill
{\vs Axiom} \> \gprod \ts{[} {\vs Number} \ts{]} {\vs Term} \ts{==} {\vs Term} \\
\\
{\vs Term} \> \gprod {\vs Var} \gopt{\ts{\_} {\vs Number}} \galt {\vs Const} 
       \galt \ts{(} {\vs Term} \ts{)} \galt {\vs PrefixTerm} \\
  \>\> \galt {\vs PostfixTerm} \galt {\vs InfixTerm} \galt {\vs FunctionTerm} \\
  \>\> \galt {\vs LispTerm} \galt {\vs RoundfixTerm} \galt {\vs CoercionTerm} \\
{\vs PrefixTerm} \> \gprod {\vs PrefixOp} {\vs Term} \\
{\vs PostfixTerm} \> \gprod {\vs Term} {\vs PostfixOp} \\
{\vs InfixTerm} \> \gprod {\vs Term} {\vs InfixOp} {\vs Term} \\
{\vs FunctionTerm} \> \gprod {\vs FuctionOp} \ts{(} \gopt{{\vs Term} 
  \gstar{\ts{,} {\vs Term}}} \ts{)} \\
{\vs RoundfixTerm} \> \gprod {\vs RoundfixOp} {\vs Term} \gstar{\ts{,} {\vs Term}}
   {\vs RoundfixOp} \\
{\vs LispTerm} \> \gprod \ts{(} {\vs LispOp} \gstar{{\vs Term}} \ts{)} \\
{\vs CoercionTerm} \> \gprod {\vs CoercionRoundfixOp} {\vs ExternalObject} {\vs CoercionRoundfixOp} \\
  \>\> \galt {\vs CoercionPrefixOp} {\vs ExternalObject} \\
  \>\> \galt {\vs CoercionFunctionOp} \ts{(} {\vs ExternalObject} \ts{)} \\
  \>\> \galt \ts{(} {\vs CoercionLispOp} {\vs ExternalObject} \ts{)}
\end{tabbing}
\end{minipage}
 
\noindent
\rule{\textwidth}{0.6pt}
%} % end fbox
\caption{The Syntax of ReDuX Data Types, Part 2}
\label{dtgram2}
\end{center}
\end{figure}
\begin{figure}[htbp]
\begin{center}
%\fbox{
\noindent
\rule{\textwidth}{0.6pt}

\begin{minipage}{6in}
\begin{tabbing}
{\vs ClosingRoundfixName} \gsep {\vs DTName} \gsep {\vs OpName} \gsep {\vs Value} \= \kill
{\vs ClosingRoundfixName} \gsep {\vs DTName} \gsep {\vs OpName} \gsep 
   {\vs Value} \> \gprod {\vs XIdent} \\
{\vs AlgoName} \gsep {\vs Domain} \gsep {\vs Range} \gsep {\vs SortName} \> \gprod {\vs Ident} \\
{\vs CoercionOpNames} \gsep {\vs ConstNames} \gsep \\
   {\vs ObjectNames} \gsep
   {\vs OpNames} \gsep {\vs VarNames} \> \gprod {\vs XIdents} \\
{\vs SortNames} \gsep {\vs XSortNames} \> \gprod {\vs Idents} \\
\\
{\vs InfixOp} \gsep {\vs LispOp} \gsep 
   {\vs PostfixOp} \gsep {\vs PrefixOp} \gsep {\vs RoundfixOp} \gsep
   {\vs Var} \= \kill
{\vs CoercionPrefixOp} \gsep {\vs CoercionRoundfixOp} \gsep {\vs Const} \gsep
   {\vs FunctionOp} \gsep \\
   {\vs InfixOp} \gsep {\vs LispOp} \gsep 
   {\vs PostfixOp} \gsep {\vs PrefixOp} \gsep {\vs RoundfixOp} \gsep
   {\vs Var} \> \gprod {\vs XIdent} \\
\\
{\vs XIdents} \= \gprod {\vs XIdent} \gstar{\ts{,} {\vs XIdent}} \\
{\vs Idents} \> \gprod {\vs Ident} \gstar{\ts{,} {\vs Ident}} \\
\\
{\vs Number} \> \gprod \gplus{{\vs Digit}}
\end{tabbing}
\end{minipage}
%} % end fbox

\noindent
\rule{\textwidth}{0.6pt}
\caption{Definition of Auxiliary Grammar Symbols}
\label{axgram}
\end{center}
\end{figure}

Identifiers ({\it Ident}) are standard identifiers starting with a letter
followed by an arbitrary string of letters or digits.
Extendent identifiers ({\it XIdent}) may be made up of letters, digits
and special characters in any order.
The special characters which can be used in all sections (without escape
character) are:
\begin{quote} 
{\tt\verb:! @ # $ * _ & + ^ | ~ [ ] { } ' " . \ / ? >:}
\end{quote}
\begin{verbatim} 
` < = : - % 
\end{verbatim}
Further special charecters can be allowed depending on the context.
\begin{verbatim} 
` < = : - % 
\end{verbatim}
In a context where they are not allowed these special characters must be 
preceded by an escape character.
The escape character is the backtick ({\tt `}). 
The backtick itself is made available by two subsequent backticks.
For more information on the exact syntax see \NPUG.

The double per cent sign (\verb/%%/) is used as comment sign.  After
reading the comment sign, all characters until the end of the line are
skipped.  This comment symbol may be used whenever terms or theorems
are read.

The following semantic constraints are given: the argument list of an
operator must be as long as the type list of its domain and the
arguments must be of the appropriate type. Also the name of an
operator, variable or constant may not be the same as another
operator's, variable's or constant's name.  Furthermore, no left-hand
side of an axiom may consist only of a constant or a variable because
there is the built-in assumption that constants and variables are
irreducible.
This restriction does not apply to nullary operators.  Thus
in \redux\  constants can be handled more efficiently than nullary operators.

Note that external objects must not be used in {\tt ic}, {\tt to}, {\tt ts}
and {\tt uc}.
The {\tt ac} program can handle external terms, but in the term ordering
algorithms, external operators
are treated like constants, independent of their argument. Its
main menu contains an item to compute terms with external operators.
In {\tt trd}, external operators have a fixed argument that
must be supplied interactively by the user.
The {\tt ev} program is prepared to use a special kind of external objects,
so called {\em partial evaluation domains}.
These objects a meant to support the normalization process and will neither
appear in input terms nor in output terms.
All programs using external objects read in the file {\tt algos}
(unless another file has been specified in the command line)
that contains one entry for each algorithm that can be specified
in the {\tt EXTERNAL} section of the used data type or as property under
the key {\tt XINT}.

After reaching the {\tt END} symbol, the programs stop reading the
algebraic specification. This feature can be used to put comments at
the end of algebraic specification files though we encourage to use the
\verb/%%/-sign for comments.

Algebraic specifications are called {\em data types} in the \redux\ 
system.  The data type describing lists over the elements 0 and 1 can
be described as shown in Figure~\ref{fi:exa}.

\begin{figure}[htbp]
\begin{center}
\noindent
\rule{\textwidth}{0.6pt}

%\fbox{
\begin{minipage}{5.9in} %necessary if there is a footnote within the figure
\begin{quote}
\begin{verbatim}
DATATYPE LIST;
%%  STATUS completed
%%  AUTHOR Reinhard Buendgen
%%  DESCRIPTION List({0,1}): cons, app, rev
%%  ORIGIN  folklore
SORT
   EL, LIST;
CONST
   0, 1: EL;
   nil: LIST;
VAR
   A, B, C: EL;
   L, M, N: LIST;
OPERATOR
   @: LIST, LIST -> LIST;   %%  append 
   [: EL, LIST -> LIST;     %%  cons
   rev: LIST -> LIST;       %%  reverse
NOTATION
   rev: FUNCTION;
   @: INFIX;
   [: ROUNDFIX ];
AXIOM
   [1] (nil @ L) == L;
   [2] ([ A,L ] @ M) == [ A,(L @ M) ];
   [3] rev(nil) == nil;
   [4] rev([ A,L ]) == (rev(L) @ [ A,nil ]);
END
\end{verbatim}
\end{quote}
\end{minipage}
 
\noindent
\rule{\textwidth}{0.6pt}
%} %end fbox
\end{center}
\caption{An example of a \redux\  data type specification} \label{fi:exa}
\end{figure}
The the data type is named {\tt LIST} and contains two sorts,
namely {\tt EL} and {\tt LIST}. The constants {\tt 0} and {\tt 1}
are of sort {\tt EL} and {\tt nil} is of sort {\tt LIST}.
There are variables of sort {\tt EL} ({\tt A, B, C}) and of sort {\tt LIST}
({\tt L, M, N}).
There are three operators in the data type:
The operator {\tt @} has the domain {\tt LIST} $\times$ {\tt LIST}
and is written as infix operator.
The operator {\tt [} has the domain {\tt EL} $\times$ {\tt LIST} and
is used as roundfix operator.
The last operator ({\tt rev}) has the domain sort {\tt LIST}
and is used as a function. 
The result sort of all three operators is {\tt LIST}.


\subsection{Traces}
The \redux\  demo programs provide several tracing facilities that can be
arbitrarily combined:
\begin{itemize}
\item The counter example trace
\item The critical pair trace
\item The formula reduction trace
\item The general ordering trace
\item The inessential critical pair trace
\item The node deletion trace
\item The path ordering trace
\item The polynomial ordering trace
\item The procedure name trace
\item The proved theorems trace
\item The substitution trace
\item The theorem derivation trace
\item The tree extension trace
\item The evaluation trace
\item The term generation trace
\end{itemize}

When traces can be selected, the programs ask for each of the available
traces whether it shall be switched on or off.
The user must type {\tt 1} if he wants to select the trace or {\tt 0} if
he does not.
A description of the traces is given in Appendix~\ref{Traces}.

\subsection{Timing}
The \redux\  demo programs provide timing facilities. Since the resolution
of the timing commands is rather slow ($1/60$~s or $1/100$~s depending on
the operating system), the displayed times
can be interpreted only if they are long relative to 
$1/60$~s or $1/100$~s respectively.

All traces need CPU time, therefore the traces should be switched off if
program timing is done.
