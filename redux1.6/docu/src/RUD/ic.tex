\section{The Inductive Completion Program}
\subsection{Introduction}
The {\tt ic} demo program allows the user to experiment with the
inductive completion algorithm given in \cite{Buendgen:87}.
For the {\tt ic} program the same restrictions as for {\tt ts}
apply. That is,
\begin{enumerate}
\item
for each sort at least one variable must be declared and
\item
non-left-linear term rewriting systems must be well-behaved
\cite{BuendgenEckhardt:92}.
\end{enumerate}
As all of the \redux\  demo programs, it reads an algebra specification.
After the data type is read, the user is asked to install and select
one or more term orderings.
(See Section~\ref{InitializationOfOrderings} for more information on the
initialization of term orderings.)

Then the program tries to determine if the axioms (oriented from 
left to right) are terminating and convergent. There are three cases:

\begin{itemize}
\item The axioms can be proved to be terminating. In this case,
the program tests the convergence of the axioms interpreted as term 
rewriting system.

\begin{itemize}
\item 
If the data type is convergent, the program prints this explicitly and
continues. 

\item If it is not convergent, the program 
prints a list of the divergent critical pairs.
Then the user is given the choice to run the Knuth-Bendix algorithm with
connectivity criterion (see Section~\ref{KBCompletion}).
Otherwise it is assumed that the data type is ground confluent.
\end{itemize}

\item The axioms cannot be proven to be terminating or not terminating. 
Then the  program assumes that they are terminating which is indicated
by a warning message. The program continues as in the above case.

\item The axioms can be proven to be non-terminating.
In this case the program prints a message and runs the Knuth-Bendix 
algorithm with connectivity criterion.
\end{itemize}

Then the user can decide if he wants to trace the program. If tracing has been
chosen,
the user can select among eleven traces, namely the tree extension trace, the
node deletion trace, the
theorem derivation trace, the proved theorems trace, the formula reduction 
trace, the inessential critical pair trace, the procedure name trace, 
the counter example trace, the substitution trace, the term ordering
trace and the path ordering trace.

Now the program decomposes the data type into constructors 
and defined function symbols and computes a ground reduction test set 
as is performed in the {\tt ts} program
(as described in Section~\ref{InterleavedTopSetTrees}).

\subsection{The Main Menu}
After these prerequisites the main menu is displayed:
\begin{quote}
\begin{verbatim}
You have the following choices: 
  i - description of irreducible ground terms,
  n - normalize a term,
  g - check ground reducibility,
  p - prove a theorem,
  b - display the basic axioms,
  c - display the current data type,
  s - show times and counters,
  r - reset times and counters,
  t - set/unset trace options
  O - de/install term orderings,
  o - select a term ordering/termination test,
  h - help, print the menu
  q - quit.
[i/n/g/p/b/c/r/s/t/o/O/h/q] *
\end{verbatim}
\end{quote}
The user must enter one of the menu items by selecting the corresponding
letter.

\begin{itemize}
\item {\bf Description of Irreducible Ground Terms (i)}\\
First, the decomposition into free constructors and defined function
symbols, the ground reduction test set and the size of the top
set is displayed. Then, the top set tree and the GNF grammar is
shown.

\item {\bf Normalize a Term (n)}\\
This menu item allows the user to enter a term consisting of
constants, variables and functions of the used data type.
The program tries to reduce the term by the axioms (interpreted as
rewrite rules) in the current data type. The normalized term
(which can be the same as the entered term) is printed.

\item {\bf Check Ground Reducibility (g)}\\
The program checks if the entered term is ground reducible.
If it is ground reducible, the program shows a list of all positions
at which the term is inductively complete and the corresponding
axioms. For each of those positions, some lines are displayed as in the 
following example:
\begin{quote}
\begin{verbatim}
+(A,0) is  ground reducible.
1. The set of positions: 
+(A,0) at pos. /\
is inductively complete with rules: 
 [4] +(S(B),A) == S(+(B,A));
 [3] +(0,A) == A;
 [5] +(P(B),A) == P(+(B,A));
\end{verbatim}
\end{quote}
%### Im Beispiel wurde integer.rdx verwendet

\item {\bf Prove Theorem (p)}\\
The user must enter an axiom (consisting of two terms over the used data
type that are separated by {\tt ==}). Then the program checks if
the axiom is an equational theorem or can be proved by inductive
completion. See Section~\ref{InductiveCompletion}.

\item {\bf Display the Basic Axioms (b)}\\
The basic axioms, i.\,e.\  the axioms without any {\em inductively} proven
theorems are displayed. If the Knuth-Bendix algorithm was used
to complete a non-terminating or divergent algebraic specification
during the program initialization, the completed data type is shown.

\item {\bf Display the Current Data Type (c)}\\
This item displays the current data type, i.\,e.\  the (possibly
completed) data type that has been extended with all 
inductively proven theorems.

\item {\bf Show Times and Counters (s)}\\
The counters (number of matches, created top set nodes, reductions
and unifications) and
the timers (cpu-time used so far, cpu-time spent for ground reduction
tests, matching, reduction and unification) are displayed.

\item {\bf Reset Times and Counters (r)}\\
The counters and timers are reset, either altogether or selectively.

\item {\bf Set/Unset Trace Options (t)}\\
Eleven traces can be selected, namely the tree extension trace, the
node deletion trace, the
theorem derivation trace, the proved theorems trace, the formula reduction 
trace, the inessential critical pair trace, the procedure name trace, 
the counter example trace, the substitution trace, the term ordering
trace and the path ordering trace.
A description of these traces is given in Appendix~\ref{Traces}.

\item {\bf De/Install Term Orderings (O)}\\
This item allows the user to install or de-install a term ordering. See
Section~\ref{InitializationOfOrderings} for more information
on term orderings.

\item {\bf Select a Term Ordering/Termination Test (o)}\\
The user can select a lexicographical combination of the initialized 
term orderings and the straight and reverse pseudo orderings.
To initialize a term ordering use the previous menu item.

Then the program performs the termination test for the new 
term ordering. The Knuth-Bendix procedure cannot be run in this
item, i.\,e.\  if the rules are not terminating with the selected
ordering, a new ordering has to be selected.

\item {\bf Help, Print the Menu (h)}\\
This item prints the main menu. It can be useful if the program
displays the short form  of the main menu.

\item {\bf Quit (q)}\\
This item allows the user to quit the program.
\end{itemize}

\subsection{Inductive Completion}
\label{InductiveCompletion}
If the menu item ``prove a theorem'' has been selected,
the user must enter an axiom (consisting of two terms over the used data
type that are separated by {\tt ==}).

The program checks first if the entered theorem is an equational 
consequence of the algebra specification (i.\,e.\  the terms on both sides of
the entered theorem can be reduced to the same term by application of the 
axioms in the algebra). In this case, a message like
\begin{quote}
\begin{verbatim}
The theorem +(0,A) == A
is valid in all models of the data type INT.
Both terms can be reduced to A.
\end{verbatim}
\end{quote}
is displayed.

If the theorem cannot be proved by application of the axioms in the data type,
the program prints a message like:
\begin{quote}
\begin{verbatim}
The theorem  0) +(A,0) == A
is not an equational consequence, 
it may however be an inductive consequence of
the INT data type.
\end{verbatim}
\end{quote}
Then the program tries to prove the theorem by applying the extended 
Knuth-Bendix algorithm for inductive completion given in \cite{Kuechlin:89}.
The entered 
theorem is oriented and added to the set of axioms. If the new set of
axioms  can 
be completed by the algorithm, the entered theorem is a inductive consequence
of the initial set of axioms, otherwise it cannot be proved by inductive
completion.

To start the completion, the user must decide if trace information shall
be displayed.
There are six traces which can be selected.
Then the program asks if the inductive completion shall be done in step 
mode or in automatic mode. If step mode has been chosen, the user must 
orient the computed critical pairs manually.

In the completion process terms may be proved to be inductively 
reducible. In general, a term can be inductively reducible at several
positions. 
These inductively reducible terms are displayed, together with the
positions at which they are reducible and the axioms for which they
are inductively complete.
Positions are printed as a sequence of positive numbers
that are separated by points. The empty position $\lambda$ is displayed 
as \verb+/\+. The user must choose one of the given positions.
In the following example the term is reducible only
at one position, therefore the user has no real choice:

\begin{quote}
\begin{verbatim} 
+(+(A,B),C) is inductively reducible.
1. The set of positions: 
+(A,B) at pos. 1
is inductively complete with rules: 
 4) +(S(B),A) == S(+(B,A))
 3) +(0,A) == A
 5) +(P(B),A) == P(+(B,A))
 Choose position set [1 - 1]. *
\end{verbatim}
\end{quote}

Then the number of critical pairs computed with the new axioms are 
printed. The programs distinguishes between essential and inessential
critical pairs. The inessential critical pairs are not 
necessary for correctness reasons, but may be helpful to ensure the 
termination of the completion process.
If an essential critical pair can be reduced by an inessential critical
pair, the program asks if the inessential critical pair shall be treated next:

\begin{quote}
\begin{verbatim}
The essential critical pair
( 5 and 13) ( 5)  P(+(B,S(M(0,B)))) == 0
can be reduced by the inessential critical pair
( 8 and 13) ( 5)  P(+(B,S(M(0,B)))) == 0
using a straight ordering.
Do you want to treat the inessential critical pair next [y/n]? *
\end{verbatim}
\end{quote}
If the user enters {\tt Y}, the
inessential critical is the next critical pair that is oriented,
otherwise the essential critical pair is oriented next.

If the completion succeeds, the program prints the proven theorem
together with a message that it is an inductive theorem of the data 
type. Then the program displays the basic axiom set and a list of
the new theorem and the lemmas that have
been proved while the completion procedure. Now the program prints 
some statistics and a menu that allows the user to choose the data 
type for the following computations:

\begin{quote}
\begin{verbatim}
You have the following choices: 
 B - continue using the basic axiom set,
 L - continue using the last axiom set,
 N - continue using the new axiom set,
 E - exit program.
Enter your choice [B/L/N/E]! *
\end{verbatim}
\end{quote}

\begin{itemize}
\item Continue using basic axiom set (B)\\
The basic axiom set (the possibly completed algebra specification fed
in at the program start) is used for further computations.
\item Continue using last axiom set (L)\\
The current data type before the last inductive completion process is used.
\item Continue using new axiom set (N)\\
The new data type (including the new theorem and lemmas) is used. Therefore the 
program must compute a new ground reduction test set.
\item Exit program (E)\\
This menu item allows to leave the program after some statistical 
information concerning the garbage collector has been displayed.
\end{itemize}
 
If the completion does not succeed, this is told to the user.
Then he has the choice to use the basic axioms or the last axioms for the
next computation or to leave the program.
