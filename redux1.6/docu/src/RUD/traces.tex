\section{Traces}
\label{Traces}
\subsection{The Counter Example Trace}
The counter example trace traces the cover algorithm, the core of the
ground reducibility test.
If the considered term can not be unified with a left-hand side of an
axiom, this trace prints just the considered term.
Otherwise, it shows a subterm together with the sets of
rmgus (restricted most general unifiers, i.\,e.\  the substitutions are
restricted to the principal subterms) and the ground term sequence
which cannot be matched by these rmgus.

\noindent
Example:
\begin{quote}   
\begin{verbatim}
+(A,A)
None of the following term lists
0   A   
1   0   
s-matches
1   1   
\end{verbatim}
\end{quote}
Here $(0,A)$ and $(1,0)$ are the terms of two rmgus and $(1,1)$ is the
term list which cannot be covered by the rmgus.

\subsection{The Critical Pair Trace}
The critical pair trace can be activated in the {\tt ac} program.
It can be selected in two versions:
If it is selected by typing {\tt 2}, it prints a line like 
\begin{quote}
\begin{verbatim}
Computing critical pairs between axioms no. 2 and 1 ...
\end{verbatim}
\end{quote}
if the program starts to compute the critical pairs between two axioms.
While the program is creating the critical pairs, it prints messages like 
\begin{quote}
\begin{verbatim}
Proposing new critical pair ...
  using unifying substitution:
{ X <- +(I(0),x),  x_1 <- 0,  x_2 <- x,  X_1 <- 0,  x_3 <- I(0) }
  Pair:       x    +(I(0),x)
  from axioms +(x,+(I(X),X)) == x
  and         +(0,X) == X
  at position +(x,+(I(X),X)).
Unresolved critical pair: 
( 2 and  1) ( 3)  x == +(I(0),x)
\end{verbatim}
\end{quote}
if a critical pair has been computed. 
The variables \verb/x/, \verb/x_1/, \verb/x_2/ and \verb/x_3/ are
auxiliary variables. 
If the considered critical pair
can be resolved, the last two lines are omitted.

If the critical pair trace is called with {\tt 1}, the trace output is
restricted to unresolved critical pairs. They are printed together with
their origins and weights.

\subsection{The Formula Reduction Trace}
This trace can be used in the completion procedures.
If a new axiom is put into the set of new axioms, the theorems in the
theorem queue and the axioms in the set of new axioms are reduced if
possible. For each reduction of an axiom, the formula reduction trace 
prints information lines like the following:
\begin{quote}
\begin{verbatim}
 Old axiom no 5,            
+(I(0),A) == A,
 considered.
AXIOM 9 YIELDS
0
AXIOM 1 YIELDS
A
and proved.
\end{verbatim}
\end{quote}
(Only the lines {\tt AXIOM \ldots\ YIELDS} and the lines that follow the
{\tt YIELDS} are printed by the formula reduction trace. The other
lines are normal program output.)

The trace output in this example means that two rules (9 and 1) could be
applied to terms or subterms in the given axiom and the results of the
applications are {\tt 0} and {\tt A}.

However, it is not specified which subterm of the considered axiom is
reduced. In step mode, it is possible to look at the set of new axioms to
see the axioms whose numbers are given by the formula reduction trace.
With that information it should be possible to conclude what subterm has
been reduced. Different instances of a variable have no different
representations in the output of the formula reduction trace.

If {\em theorems} are reduced, they are not printed
by the formula reduction trace. (Only information like ``{\tt AXIOM 9
YIELDS \ldots}'' is printed.) 
Therefore it is recommended to use this 
trace together with the theorem derivation trace and the proved
theorems trace.

If the formula reduction trace and the theorem derivation trace are
combined, the theorem derivation trace prints all derived theorems, 
not only the theorems that cannot be proved as it does when it is used
without the formula reduction trace:
\begin{quote}
\begin{verbatim}
 Deriving the set of new theorems  -

 Theorem
+(0,+(0,C)) == C
 (from 9 and 4) derived
AXIOM 1 YIELDS
C
AXIOM 1 YIELDS
C
 and proved.
\end{verbatim}
\end{quote}

\subsection{The General Ordering Trace}
The general ordering trace can be used in the completion procedure.
It prints for each critical pair (theorem) which
must be ordered a message as in the next example:
\begin{quote}
\begin{verbatim}
TPPO: 'P' ordered reverse.
\end{verbatim}
\end{quote}
The string after {\tt TPPO:} (here: {\tt 'P'} for path ordering) is the 
lexicographical combination of orderings chosen in the initialization step.

If the automatic ordering fails, the general ordering trace displays:
\begin{quote}
\begin{verbatim}
TPPO: 'Y' failed.
\end{verbatim}
\end{quote}
(Here the polynomial ordering has been initialized.)

\subsection{The Inessential Critical Pair Trace}
The inessential critical pair trace can be used in the inductive completion
procedures.
It prints lists of the inessential critical pairs, together with
the axioms they are derived from and their weights.

\noindent
Example:
\begin{quote}
\begin{verbatim}
Inessential critical pairs:

  origin   weight Inessetial critical pairs: 
 
  origin   weight    theorem
 
 
(12 and 26) ( 0)  +(0,*(0,y)) == *(0,+(*(y,y_1),y))
(12 and 26) ( 0)  +(*(x,0),*(x,0)) == *(x,+(*(0,x_1),0))
\end{verbatim}
\end{quote}

\subsection{The Node Deletion Trace}
This trace can be used during the top set tree computations.
It prints lines like the following:
\begin{quote}
\begin{verbatim}
red. OPP(OPP(A))
\end{verbatim}
\end{quote}
This means that a term could be reduced by the axioms and need not
to be entered into the top set tree. The remaining output of the node
deletion trace is of the following form:
\begin{quote}   
\begin{verbatim}
deleted: Label: SUCC(OPP(A))
is not ground. It's level is 1
Father's label: SUCC(A)
d-level list (labels): 
OPP(A)  
labels of successors: 
Unifying rules: AXIOMS
 3) SUCC(OPP(SUCC(A))) == OPP(A)
\end{verbatim}
\end{quote}
First the label of the deleted node is printed. The next lines give
information about that node. First it is printed if the node is ground 
(contains no variables) and its level (the root has level $-1$). Then
the label of the father's node is printed. The next lines show the
d-level-list (i.\,e.\  the list of principal subterms of the node's label).
In the following line the labels of the successors of the
considered node are printed. (Here this list is empty). 
The last lines show the rules whose left-hand sides unify with the node
label.

\subsection{The Path Ordering Trace}
When the recursive path ordering is involved, trace information of the
form
\begin{quote}
\begin{verbatim}
TERMS: I(+(A,B)) +(I(B),I(A))
PSETS: ((B.+(A,B).I(+(A,B))), (A.+(A,B).I(+(A,B))))
       ((A.I(A).+(I(B),I(A))), (B.I(B).+(I(B),I(A))))
PATHS: (+(A,B).I(+(A,B))) (I(A).+(I(B),I(A)))
\end{verbatim} 
\end{quote}
is displayed. The line starting with {\tt TERMS:} indicates the
comparison of two terms, the line starting with {\tt PSETS:} the
comparison of two path-sets and the last line the comparison of two
paths. (The terms in a path are separated by dots, the paths must be
read from right to left.)

\subsection{The Polynomial Ordering Trace}
This trace prints messages like

\begin{quote}
\begin{verbatim}
CPPI Polynomial interpretation LHS:
x3^2x1^2+2x3^2x2x1+x3^2x2^2
CPPI Polynomial interpretation RHS:
x3^2x1^2+x3^2x2^2
CPPI LHS-RHS:
2x3^2x2x1
\end{verbatim}
\end{quote}

if the polynomial ordering is involved and two terms are
compared.

The polynomial interpretations of the left-hand side and right-hand side of an
axiom is printed together with the difference of left-hand side minus
right-hand side.

\subsection{The Procedure Name Trace}
The procedure name trace prints the names of some procedures in the
program. Therefore it is only useful together with the source code.

\subsection{The Proved Theorems Trace}
If a new axiom is put into the set of new axioms, possibly some of the
theorems in the theorem queue can be proved by the new axiom. In this
case, the proved theorems trace displays all of the theorems in the
theorem queue that can be proved and are therefore deleted from the
theorem queue. 
\begin{quote}
\begin{verbatim}
 Reducing the set of old theorems -

 Theorem
I(0) == 0
 (from 1 and 8) reduced and proved.
\end{verbatim}
\end{quote}
Theorems that have not been entered into the theorem queue (because they 
could be proved by the new axioms that were in the set of new axioms
while the theorems are derived) are not printed by the proved theorem
trace.

\subsection{The Substitution Trace}
The substitution trace can be used in ground reducibility tests.
It traces the rmgus computed during the test.
It displays a subterm of a term to be tested for inductive
reducibility. If this subterm unifies with the left-hand side of a axiom
then this axiom, the rmgu (the most general unifier restricted to the
variables of the subterm) of the 
subterm and this rule are displayed.

\noindent
{\bf Example:}
\begin{quote}   
\begin{verbatim}
rmgus at +(A,A_1)
A_1-->B
A-->SUCC(A_2)
 5) +(SUCC(A),B) == SUCC(+(A,B))
A_1-->A_3
A-->0
 4) +(0,A) == A
\end{verbatim}
\end{quote}   
This means that $rmgu(+(A,A_1),+(succ(A_2),B))=\{A_1 \mapsto B, A \mapsto
succ(A_2)\}$ and $rmgu(+(A,A_1),\\+(0,A_3))=\{A_1 \mapsto A_3, A \mapsto
0\}$. Note that the variables of the rules are renamed before the
unification.

%\subsection{The Term Ordering Trace}

\subsection{The Theorem Derivation Trace}
After adding an axiom to the set of new axioms, a set of new theorems
(critical pairs)
is derived. Some theorems can be proved by the axioms in the set of new
axioms. The other theorems are added to the queue of theorems. If such
theorems are derived by the new axiom, the theorem derivation trace will
print a message like in the following example:
\begin{quote}
\begin{verbatim}
 Deriving the set of new theorems which cannot be proved -
    
 
 Theorem
+(I(I(0)),A) == A
 (from 5 and 4) derived and retained.
\end{verbatim}
\end{quote}
Each unproved theorem is printed together with the number of the new 
axioms it was derived from.

\subsection{The Tree Extension Trace}
The tree extension trace prints messages like the following:
\begin{quote}
\begin{verbatim}
Label: +(A,A_1)
is not ground. It's level is 0
Father's label: A
d-level list (labels): 
A  A  
labels of successors: +(+(A,A_1),0)  +(+(A,A_1),OPP(A_2))  
  +(+(A,A_1),SUCC(A_2))  +(+(A,A_1),+(A_2,A_3))  
Unifying rules: AXIOMS
 4) +(0,A) == A
 5) +(SUCC(A),B) == SUCC(+(A,B))
 6) +(OPP(A),B) == OPP(+(A,OPP(B)))
\end{verbatim}
\end{quote}
The meaning of this output is the same as in the node deletion trace.
Some nodes may be deleted in the
further process, therefore it is possible that the list of successors
contains some nodes that are not in the resulting tree.
The last lines show the rules whose left-hand sides unify with the node
label.

\subsection{The Evaluation Trace}

For each pass of the evaluation and de-evaluation of a term the evaluation 
traces shows how the dictionary relating non-evaluable terms to generators
of the evaluation domain is built up.

A typical trace is the following
\begin{verbatim}
Input term:  *
1+(h(i) + b + h(i));
The normal form of 1 + h(i) + b + h(i) is 
TNMEVI (1): [(1 + h(i) + b + h(i)/0/SAC2IP)]
EVCLGD (2): 1 + h(i) + b + h(i)  [1:<b||() >,2:<h(i)||() >]
EVCLGD (3): --- start sorted dictionary list ---
SAC2IP:  [1:<b||() >,2:<h(i)||() >]
---- end sorted dictionary list ----
TNMEVI (3): --- start sorted dictionary list ---
SAC2IP:  [1:<b||(0 (1 1)) >,2:<h(i)||(1 (0 1)) >]
---- end sorted dictionary list ----
EVCLEV (2) evaluating 1 + h(i) + b + h(i)
TEVDIC (1):         1 + h(i) + b + h(i)
    dictionary:     [1:<b||(0 (1 1)) >,2:<h(i)||(1 (0 1)) >]
    external sort:  SAC2IP
    interpretation: IPRSUM 
TEVDIC (1):         1
    dictionary:     [1:<b||(0 (1 1)) >,2:<h(i)||(1 (0 1)) >]
    external sort:  SAC2IP
    interpretation: IPRONE 
TEVDIC (1):         h(i)
    dictionary:     [1:<b||(0 (1 1)) >,2:<h(i)||(1 (0 1)) >]
    external sort:  SAC2IP
    interpretation: () 
TEVDIC (1):         b
    dictionary:     [1:<b||(0 (1 1)) >,2:<h(i)||(1 (0 1)) >]
    external sort:  SAC2IP
    interpretation: () 
TEVDIC (1):         h(i)
    dictionary:     [1:<b||(0 (1 1)) >,2:<h(i)||(1 (0 1)) >]
    external sort:  SAC2IP
    interpretation: () 
                      (1 (0 2) 0 (1 1 0 1)) 
dict after eval:      [1:<b||(0 (1 1)) >,2:<h(i)||(1 (0 1)) >]
             yielding 1 + b + h(i) + h(i)
dict after de-eval:   [1:<b||(0 (1 1)) >,2:<h(i)||(1 (0 1)) >]
1 + b + h(i) + h(i)
Go on [e|m|u|i|s|o|O|N|n|p|c|C|w|W|r|d|t|h|q]? * 
\end{verbatim}
The labels \texttt{TNMEVI (1):}, \texttt{EVCLGD (2):}, \texttt{EVCLGD (3):},
\texttt{TNMEVI (3):}, \texttt{EVCLEV (2)}, \texttt{TEVDIC (1):} denote
the place in the source code (procedure name and step number)
from where the trace output was written.

The information following \texttt{TNMEVI (1):} describes a list
of evaluation candidates (see \RUD).

A dictionary is a list in square brackets (\texttt{[}, \texttt{]})
of the form $n$\texttt{:<}$t$\texttt{||}$x$\texttt{>}
where $n$ is a natural number, $t$ is a term and $x$ is the  
representation of the $n$-th generator of the evaluation domain.
This means that the unevaluable term $t$ is to be mapped to the generator $x$.
If there is more than one evaluation domain one dictionary for each of the
evaluation domains may be computed.

The information following \texttt{TNMEVI (1):} displays the data
available when interpreting the term just behind the colon:
the dictionary, the external sort representing the evaluation domain
\wrt\ which the term is to be evaluated and the interpretation of the
top symbol of the term for that evaluation domain

The lines
\begin{verbatim}
                       (1 (0 2) 0 (1 1 0 1)) 
dict after eval:      [1:<b||(0 (1 1)) >,2:<h(i)||(1 (0 1)) >]
             yielding 1 + b + h(i) + h(i)
dict after de-eval:   [1:<b||(0 (1 1)) >,2:<h(i)||(1 (0 1)) >]
1 + b + h(i) + h(i)
\end{verbatim}
describe the  result of  evaluating the term which is printed after
the previous occurrence of \texttt{EVCLEV (2):}
\texttt{(1 (0 2) 0 (1 1 0 1))} and the result of its de-evaluation
\texttt{1 + b + h(i) + h(i)} \wrt\ the dictionaries shown.

\subsection{The Term Generation Trace}
\label{TermGenerationTrace}
The term generation trace displays information while the terms are
constructed.
Some knowledge of the involved algorithms is required to
understand its messages.

To explain the trace information, we show a complete trace of the
generation process (where the first mapping algorithm is used) of a
small term over the signature given in {\tt typ/stack.typ}:

\begin{quote}
\begin{verbatim}
NT2TE: n=105 no. of positions=5
NT2T1E: n=21 sort=STACK
\end{verbatim}
\end{quote}
The first line states that the topmost generation algorithm {\tt
NT2TE} has been called with the argument 105. This argument is the
number of the term which is to be generated. (Note that the numbers
which represent terms start with 0.) The first result of the computation
is that the term will have 5 positions.

The second line means that the algorithm {\tt NT2T1E} has been called
with the argument 21. This number is the number of the desired term in
the set of terms with 5 positions. The next result of  the computation
is  the information that the term will be of return type (sort)
{\tt STACK}.

\begin{quote}
\begin{verbatim}
NT2T2E{ p=5 s=STACK n=21
   top-level operator=PUSH n'=21
   distribution of pos. over arguments (1, 3) n''=21
\end{verbatim}
\end{quote}
The first of the next three lines reveal that the function {\tt NT2T2E}
is called to generate a term with 5 positions and return type {\tt
STACK} whose number is 21.

The next line show that this term will have the top-level operator {\tt
PUSH} and that it will be the 21-st ({\tt n'=21}) of all terms with 5
positions, return type {\tt STACK} and top-level operator {\tt PUSH}.

The third line indicates that the first argument of {\tt PUSH} will get
one of the four positions which remain to be
distributed over the arguments of {\tt PUSH}, the second argument will
have 3 positions. The desired term is term number 21 in the set of terms
with 5 positions, return type {\tt STACK}, top-level operator {\tt PUSH}
and the mentioned distribution of positions over the arguments.

After those computations, {\tt NT2T2E} is called recursively: 
\begin{quote}
\begin{verbatim}
NT2T2E{ p=1 s=EL n=1
NT2T2E} A
NT2T2E{ p=3 s=STACK n=5
   top-level operator=PUSH n'=5
   distribution of pos. over arguments (1, 1) n''=5
NT2T2E{ p=1 s=EL n=1
NT2T2E} A
NT2T2E{ p=1 s=STACK n=1
NT2T2E} ERROR
NT2T2E} PUSH(A,ERROR)
NT2T2E} PUSH(A,PUSH(A,ERROR))
\end{verbatim}
\end{quote}
The first two lines show that {\tt NT2T2E} is called to generate a term
with 1 position, return type {\tt EL} whose number is one. Since the
program maintains an array with all atomic terms, it can look up the
desired term which turns out to be the variable {\tt A}.

The next lines reveal that {\tt NT2T2E} is called again to compute the
second argument of the top-level operator {\tt PUSH}. This term is term
number five in the set of term with three positions and return type
{\tt STACK} and will again have the top-level operator {\tt PUSH}.
Each of its arguments will have one position.  Of course there is no
real choice in distributing the remaining two positions since each term
has at least one position.

The next four line show again the computation of two atomic terms, and
the last two lines tell how the already computed sub-results are composed
to the desired term which is {\tt PUSH(A,PUSH(A,ERROR))}.

Note that the nesting level of the calls to {\tt NT2T2E} is not
visualized by different indentation. This is because of the fact that the
nesting can become rather deep.  To compensate for the missing
indentation, all calls to {\tt NT2T2E} are marked with ``{\tt \{}'', all
returns from {\tt NT2T2E} are marked with ``{\tt \}}''.  This can be used to
navigate in the trace output in the log-file {\tt trd.log} if a
text editor is used which can move the cursor from one bracket symbol to
the corresponding one. (For the {\tt vi}--editor, the percent sign {\tt
\%} is used.)

If the second mapping algorithm is used, the trace output looks similar:
The algorithm names of the first mapping function are replaced by
those of the second one (whose last letters are ``D'').
Furthermore, all occurrences of ``number of positions'' are replaced by
``depth''.


\section{Changing the Output Mode}
\label{ap:output}

In some programs the ouput mode may be changed within the section to 
change the trace options.
\begin{verbatim}
Change output mode? [y|n] *
\end{verbatim}
if this question is answered positively the following menue appears
\begin{verbatim}
Current write mode:
term output: parentheses...
   1) around subterms with higher precedence = off
   2) around subterms with same precedence   = off
   3) to indicate associativity              = on
   4) around lisp-operator's infix arguments = on
   5) around prefix operator's argument      = off
   6) around postfix operator's argument     = off
   7) around infix operator and arguments    = off
term output:
   8) debugging          = off
data type output:
   9) rescannable output = off
global output:
  10) full debugging     = off
Toggle which flag?    [1-10, 0 for none] *
\end{verbatim}
This describes the current settings which may be modified  (toggled) until a 
\texttt{0} is entered.

The different output modes are explained in the following in full detail:
\begin{description}
\item[Parentheses around subterms with same precedence:]
  In the case of equal precedence of a term's top operator and one
  of its operators at a direct subposition, it is not always necessary
  to write parentheses around the subterm. 
  %According to the rule given in  section \ref{precrule} 
  This may happen e.g. if an infix operator occurs at top
  position and a postfix operator at subterm-position.
  With this option switched on, parentheses are always printed when
  the precedence of both aforementioned operators is the same.
  Turning off this option results in omitting brackets whenever it is possible
  due to equal precedence.
\item[Parentheses around subterms with higher precedence:]
  Normally, no parentheses are necessary around a subterm whose operator
  has a higher precedence than the term's top operator. Switching on this
  option produces parentheses in this case, though.
\item[Parentheses to indicate associativity:] When associativity is specified
  for an infix operator, brackets can be left away in sequences of such
  operators. 
  With this switch set to ``{\bf on}'' parentheses are written, however.
\item[Parentheses around lisp-operator's infix arguments:] Writing a lisp
  operator's infix arguments without parentheses may look a bit confusing, 
  though  it is correct. 
  To avoid this, set this option to write parentheses for a more
  user-friendly output.
\item[Parentheses around prefix operator's argument:] Prefix operators (they
  are always unary) are normally written without parentheses around the 
  argument.
  This can be avoided either using function notation or setting this option.
  Note, however, that these brackets do not change precedence.
  %, as was shown in Section \ref{strangeprec}.
\item[Parentheses around postfix operator's argument:] As in the case of prefix
  operators, this switch set to ``{\bf on}'' forces brackets around postfix 
  operator's
  arguments. In contrast to prefix operators, this is the only possibility to
  achieve this.
\item[Parentheses around infix operator and arguments:] This option produces
  brackets around an infix operator and its arguments as a whole. This switch
  was implemented mainly for compatibility reasons (old parser).
\item[Debugging for term output:] This option not merely writes the terms, 
  but also
  the address of each term position in the \ALDES\ {\tt SPACE}-array. 
  This can be
  useful to examine the correct handling of variables.
\item[Rescanable input for data types:] In some cases data type output cannot
  be reused as input for the parser. 
  The reason for this is that escape characters
  are omitted in the signature part of the data type. 
  To force these escape characters
  to be written on data type output and thus allowing rescanning of
  printed data types  this switch can be used.
\item[Full debugging for global output:]
  Switching on full debugging, makes \redux\  use not the normal output
  routines, but the \ALDES-function {\tt UWRITE} instead.
  So the internal structure of data types and other objects is made 
  accessible to  the user (programmer).
\end{description}
