\subsection{The Term Generation Trace}
\label{TermGenerationTrace}
The term generation trace displays information while the terms are
constructed.
Some knowledge of the involved algorithms is required to
understand its messages.

To explain the trace information, we show a complete trace of the
generation process (where the first mapping algorithm is used) of a
small term over the signature given in {\tt typ/stack.typ}:

\begin{quote}
\begin{verbatim}
NT2TE: n=105 no. of positions=5
NT2T1E: n=21 sort=STACK
\end{verbatim}
\end{quote}
The first line states that the topmost generation algorithm {\tt
NT2TE} has been called with the argument 105. This argument is the
number of the term which is to be generated. (Note that the numbers
which represent terms start with 0.) The first result of the computation
is that the term will have 5 positions.

The second line means that the algorithm {\tt NT2T1E} has been called
with the argument 21. This number is the number of the desired term in
the set of terms with 5 positions. The next result of  the computation
is  the information that the term will be of return type (sort)
{\tt STACK}.

\begin{quote}
\begin{verbatim}
NT2T2E{ p=5 s=STACK n=21
   top-level operator=PUSH n'=21
   distribution of pos. over arguments (1, 3) n''=21
\end{verbatim}
\end{quote}
The first of the next three lines reveal that the function {\tt NT2T2E}
is called to generate a term with 5 positions and return type {\tt
STACK} whose number is 21.

The next line show that this term will have the top-level operator {\tt
PUSH} and that it will be the 21-st ({\tt n'=21}) of all terms with 5
positions, return type {\tt STACK} and top-level operator {\tt PUSH}.

The third line indicates that the first argument of {\tt PUSH} will get
one of the four positions which remain to be
distributed over the arguments of {\tt PUSH}, the second argument will
have 3 positions. The desired term is term number 21 in the set of terms
with 5 positions, return type {\tt STACK}, top-level operator {\tt PUSH}
and the mentioned distribution of positions over the arguments.

After those computations, {\tt NT2T2E} is called recursively: 
\begin{quote}
\begin{verbatim}
NT2T2E{ p=1 s=EL n=1
NT2T2E} A
NT2T2E{ p=3 s=STACK n=5
   top-level operator=PUSH n'=5
   distribution of pos. over arguments (1, 1) n''=5
NT2T2E{ p=1 s=EL n=1
NT2T2E} A
NT2T2E{ p=1 s=STACK n=1
NT2T2E} ERROR
NT2T2E} PUSH(A,ERROR)
NT2T2E} PUSH(A,PUSH(A,ERROR))
\end{verbatim}
\end{quote}
The first two lines show that {\tt NT2T2E} is called to generate a term
with 1 position, return type {\tt EL} whose number is one. Since the
program maintains an array with all atomic terms, it can look up the
desired term which turns out to be the variable {\tt A}.

The next lines reveal that {\tt NT2T2E} is called again to compute the
second argument of the top-level operator {\tt PUSH}. This term is term
number five in the set of term with three positions and return type
{\tt STACK} and will again have the top-level operator {\tt PUSH}.
Each of its arguments will have one position.  Of course there is no
real choice in distributing the remaining two positions since each term
has at least one position.

The next four line show again the computation of two atomic terms, and
the last two lines tell how the already computed sub-results are composed
to the desired term which is {\tt PUSH(A,PUSH(A,ERROR))}.

Note that the nesting level of the calls to {\tt NT2T2E} is not
visualized by different indentation. This is because of the fact that the
nesting can become rather deep.  To compensate for the missing
indentation, all calls to {\tt NT2T2E} are marked with ``{\tt \{}'', all
returns from {\tt NT2T2E} are marked with ``{\tt \}}''.  This can be used to
navigate in the trace output in the log-file {\tt trd.log} if a
text editor is used which can move the cursor from one bracket symbol to
the corresponding one. (For the {\tt vi}--editor, the percent sign {\tt
\%} is used.)

If the second mapping algorithm is used, the trace output looks similar:
The algorithm names of the first mapping function are replaced by
those of the second one (whose last letters are ``D'').
Furthermore, all occurrences of ``number of positions'' are replaced by
``depth''.
