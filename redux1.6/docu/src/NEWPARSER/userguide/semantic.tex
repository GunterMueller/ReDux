\section{Semantical Aspects}

\subsection{The Semantics of Data Types}

The semantics of the language defined above shall be described in the next
paragraphs.
This is done section by section, according to figure~\ref{dtgram1}.

\begin{description}
\item[Sort declarations:]
By means of sort declaration, types are made available for the
variable, constant, operator, etc. sections. It is necessary to declare
all sorts, which are to appear later, before their first use.
All sort names must be distinct, but sort names are not inserted into the
data type's symbol list, so they are allowed as variable, constant, operator,
etc. names as well.

\item[External declarations:]
This kind of declaration adds new external sorts to the data type, which are
necessary to parse external objects. These objects comprise any \ALDES\ or 
SAC-2 objects, such as lists, $\beta$-integers, long integers, polynomials, etc.
Each external sort can be associated with a set of properties describing
the interface to the external data type. The values these properties can be set to
are algorithm names. What sort of algorithm has to follow a specific indicator
and what it is used for is shown in the following:
\begin{itemize}
\item{{\bf XREAD:}} Name of a nullary function (or nullary procedure with
three output parameters in case
of polynomial input) to read an external object.
\item{{\bf XWRITE:}} Name of a unary (or 3-ary in case of polynomials)
procedure to write an external object.
\item{{\bf XEQ:}} Name of a binary boolean function to test the equality
of two external objects.
\item{{\bf XLT:}} Name of a binary boolean function to test if the first
of two external objects is less than the second one.
\item{{\bf XFG:}} Name of a generator function -- for future use with
                  built-in evaluation domains.
\item{{\bf XTERM:}} Name of a decomposition function for external objects.
                 -- for future use with built-in evaluation domains.
\end{itemize}

It is not checked whether or not the specified algorithm names are valid,
i.e. whether the algorithms really exist. 
It is within the responsibility of the user to ensure this.
Wrongly used names may crash the demo programs!

Each property should be specified at most once. Multiple definition of a property
results in a warning and sets the indicated property to the value specified last.

\begin{example}
{\rm The following specifications allow the importation of SAC-2 lists,
integers and integral polynomials, respectively.
\begin{verbatim}
EXTERNAL
   SacList: XREAD=LREAD,  XWRITE=LWRITE, XEQ=EQUAL;
   SacInt:  XREAD=IREAD,  XWRITE=IWRITE, XEQ=EQUAL;
   SacPoly: XREAD=IPREAD, XWRITE=IPWRIT, XEQ=EQUAL;
\end{verbatim}
} % end roman
\end{example}

For external sort names the same restrictions as for normal sort names
apply, namely that they must be distinct, but are allowed as e.g. 
operator names. Though not recommended, external sort names may be
chosen the same as sort names.

\item[Constant declarations:]
Constants are declared within the constant section of a data type
specification by associating a sort (type) to each constant name.
All constant names must be distinct and different from variable,
operator and coercion operator names. The constants' sorts must not
be external sorts.

\item[Variable declarations:]
Variable declarations are similar to constant declarations in the way
that the same restrictions on variable names and sorts apply as for constant names.
It should be avoided, though, to use underscores within variable names,
because the underscore is used in the {\tt AXIOM}-section
to indicate variable indices.\footnote{Variable indices are explained
later in the paragraph about term semantic.}

\item[Operator declarations:]
Operator declarations consist of operator names together with the definition
of arity, domain types and range type. The arity is specified implicitly
by the number of domain types given. Domain types, as well as the range
type, must not be external sorts. For operator names the same restrictions
are valid as for constant or variable names. This includes that any
operator name is not allowed as constant, variable or coercion operator
name as well.

\item[Coercion operator declarations:]
Coercion operators allow the introduction of external objects into terms.
A coercion operator is a unary operator whose argument (domain) type (external sort)
has been specified in the {\tt EXTERNAL}-section and whose result (range) type
is a normal type. The argument of a coercion operator is read and
written using the procedures or functions specified via {\tt XREAD}
and {\tt XWRITE} in the external sort declaration section.

\item[Precedence declarations:]
Precedence declarations offer a natural way to avoid parentheses in complex
terms. This is achieved by assigning operators with higher precedence
a greater binding power, thus resulting in unambiguous terms, even
if parentheses are omitted.
Precedence can be assigned to every infix, prefix or postfix operator
with the exception of external prefix operators, i.e. prefix coercion operators.
Coercion operators always obtain a higher precedence than all other
infix, prefix or postfix operators.
When no precedence is specified for an operator in a situation where
term evaluation without precedence information is ambiguous, the following
rule applies:
\begin{quote}
{\em Postfix operators bind stronger than prefix operators and
prefix operators bind stronger than infix operators.}
\label{precrule}
\end{quote}

Comma-separated operators within the precedence declaration section,
as well as operators separated by {\tt =} are assigned the same precedence,
whereas the left operator gets lower precedence than the right operator
when the operators are combined using {\tt <}.
Each precedence declaration list must contain at least one {\tt <}
sign.
It is not allowed to use an operator more than once for a precedence
declaration.

\begin{example}
{\rm
The operators {\tt f} and {\tt g} get lower precedence than {\tt h};
{\tt +} and {\tt -}, which share the same precedence, get a lower
precedence than {\tt *} and {\tt /} by using the following declaration:
\begin{verbatim}
PREC
   f,g < h;
   + = - < * = /;
\end{verbatim}
} % end roman
\end{example}

Internally, a precedence declaration assigns each operator in the specified 
list a non-negative integer value. 
Higher precedence values indicate higher binding power. 
So a given precedence declaration can induce further unspecified precedence
relations. 
One way to avoid this, is to use only one list, thus constructing a total 
order.


\item[Associativity declarations:]
Associativity rules are applied when a series of infix
operators with same precedence is written without parentheses.
In the associativity declaration section it is possible to assign to
an infix operator left or right associativity or no associativity status.
This is done by the commands {\tt LEFT}, {\tt RIGHT} and {\tt NONE},
respectively. Left (right) associativity results in a leftmost (rightmost) 
setting of parentheses, whereas no associativity status leads to a warning
when parentheses are omitted in an ambiguous term.

\def\op{\mathop{\rm op}\nolimits}

Table~\ref{assoc} shows how a term $t$ of the form 
$$ t = x \op_1 y \op_2 z $$
is interpreted, when $\op_1$ and $\op_2$ are infix operators with the same
precedence.
$A(\op)$ denotes the associativity status of operator $\op$.

\begin{table}[htbp]
\begin{center}
\begin{tabular}{|l|l|l|}
\hline
$A(\op_1)$ & $A(\op_2)$ & $t$ is interpreted as \\
\hline \hline
left  & left  & $(x \op_1 y) \op_2 z$ \\ \hline
left  & right & warning, resolved by $(x \op_1 y) \op_2 z$ \\ \hline
left  & none  & $(x \op_1 y) \op_2 z$ \\ \hline
right & left  & warning, resolved by $(x \op_1 y) \op_2 z$ \\ \hline
right & right & $x \op_1 (y \op_2 z)$ \\ \hline
right & none  & $x \op_1 (y \op_2 z)$ \\ \hline
none  & left  & $(x \op_1 y) \op_2 z$ \\ \hline
none  & right & $x \op_1 (y \op_2 z)$ \\ \hline
none  & none  & warning, resolved by $(x \op_1 y) \op_2 z$ \\ \hline
\end{tabular}
\end{center}
\caption{Associativity Rules}
\label{assoc}
\end{table} 

\item[Notation declarations:]
In the notation declaration section the syntactic notation (``fixity'')
of an operator can be set.
Possible notations are prefix, postfix, infix, roundfix as well as function, 
lisp and constant notation. 
Every operator can be assigned at most one fixity.

Coercion operators may only be used in prefix, roundfix, function or lisp notation.

\begin{itemize}
\item{{\bf Prefix:}} Prefix notation is allowed for any operator, no matter
which arity it possesses.\footnote{Prefix operators which are not unary are called
{\em FunctionOps} in the grammar of figure~\ref{dtgram2}.}
Unary prefix operators can be written without parentheses
in the {\tt AXIOM}-section, all other prefix operators are automatically converted
to function notation and therefore must be written with
parentheses around the (comma-separated) list of arguments. This includes
nullary prefix operators, which have to be written with a following empty
argument list: \verb|()|.

\item{{\bf Postfix:}} Postfix notation can be assigned merely to unary operators.
Parentheses around the argument are optional.

\item{{\bf Infix:}} The infix notation requires an operator with a domain list of length
two. Parentheses are optional.

\item{{\bf Roundfix:}} Roundfix can be used for any operator with
arity greater than zero. A roundfix notation for nullary operators
is to be replaced by a constant notation. The symbol specified in the
{\tt OPERATOR}-section is used as opening roundfix symbol, the
symbol used for closing a roundfix term must be specified after the {\tt ROUNDFIX}
keyword. Reference to roundfix operators in other sections uses
only the first, opening roundfix symbol.

\item{{\bf Function notation:}} Function notation is a valid specification for
operators of all arities. It is similar to prefix notation, but requires
parentheses around the argument list. Nullary operators in function notation
must be succeeded by a pair of (opening and closing) brackets: \verb|()|.

\item{{\bf Lisp notation:}} Lisp notation is characterized by the operator
written within the argument list's parentheses, and the arguments,
as well as the operator, separated by blanks. Lispfix notation is valid
for all operators.

\item{{\bf Constant notation:}} Constant notation can only be applied to nullary
operators and allows writing the operator without parentheses, i.e.
the same way as constants are written.
\end{itemize}

\begin{example} $ $
\begin{verbatim}
VAR
   x,y: N;
OPERATOR
   !,|,f: N -> N;
   +,<  : N, N -> N;
   add  : N, N -> N;
   c    : -> N;
NOTATION
   f: PREFIX;
   !: POSTFIX;
   +: INFIX;
   |: ROUNDFIX |;
   <: ROUNDFIX >;
   add: LISP;
   c: CONSTANT;
\end{verbatim}
{\rm After the declarations above the terms in the following list
are all correct:}
\begin{verbatim}
fx      f(x)    x!      (x)!    x+y
|x|     <x,y>   c       fx+y!   (add x y)
\end{verbatim}
\end{example}

\item[Theory declaration:]
A theory (unification status) declaration allows specification of additional operator properties.
It can be set for binary operators either to the value {\tt AC} or {\tt COM}.
The value {\tt AC} indicates that the operator is associative and commutative.
Therefore it can only be used with operators whose domain and range types are
the same. The value {\tt COM} is used for commutative operators. It can
only be used with an operator whose domain types are the same.

\item[Property declaration:]
In this section different properties can be set. In the current version
there are five different ones:
\begin{itemize}
\item{{\bf XINT:}} Under property {\tt XINT} an operator may be assigned 
an algorithm name which describes the interpretation of the operator as function
over external objects.

If $f$ has been assigned an $n$-ary interpretation algorithm ${\cal A}$ under property
{\tt XINT}, then the term\footnote{Terms containing external objects
are called {\em external terms} or {\em mixed terms}.}
$$ t = f(c_1(X_1),\dots,c_n(X_n)), $$
where the $c_i$ are coercion operators and the $X_i$ are appropriate
external objects,
can be computed in one step.\footnote{Application of these one-step computations
as often as possible to all subterms is called {\em computing} that term.}
The computation of $t$ evaluates to the result
$$ c({\cal A}(X_1,\dots,X_n)), $$
where $c$ is a coercion operator. The operator $c$ is chosen automatically if it
is not specified explicitly using the {\tt COERC}-property explained later.
If $f$ is a function of type
$$S_1 \times \dots \times S_n \rightarrow S_r$$
then $c$ is the
first coercion operator of type $S_x \rightarrow S_r$ for some $S_x$ which was declared
before in the coercion operator section.

It is neither checked, whether the specified interpretation algorithm really exists,
nor if it has the appropriate arity (as well as argument and return types).

\item{{\bf KBINDEX, KBWEIGHT:}} These two properties are used in conjunction
with the Knuth-Bendix ordering to set the weights and indices of operators and
constants.
See the ReDuX User Guide for more information.

\item{{\bf COERC:}} This property is closely related to the {\tt XINT} property.
It allows to override the default selection of the coercion operator $c$ in
the computation of an external term, as explained before in the {\tt XINT}-section.
The specification of a {\tt COERC}-property makes sense only in conjunction
with an {\tt XINT}-property for an operator $f$ of type $S_1 \times \dots \times S_n
\rightarrow S_r$ and if the value of the {\tt COERC}-property is the name of a coercion
operator $c$ with type $S_x \rightarrow S_r$ for some type $S_x$.
This operator is then used during computation of external terms, whenever operator $f$
has to be evaluated, to convert the result of the computation back to type $S_x$.
\end{itemize}

Note, that there is no consistency check implemented so far, i.e. it is not checked
whether a specified property is sensible for a given object.

\item[Axiom declaration:]
The {\tt AXIOM}-section consists of equations that make up the theory that
is to be examined. There are two constraints which apply for axioms: no
left-hand side of an equation may consist only of a constant or a variable
and the sorts of the terms on both sides must be the same.

\end{description}


\subsection{The Semantics of Terms}
Terms are built up from constants, variables and operators, which all have to be
declared in the respective sections before they are used. A correct term
must fulfill the following restrictions:
\begin{itemize}
\item
The argument list of all operators must be as long as the type list of its
domains.
\item
The arguments must be of the appropriate type.
\end{itemize}
Thus, if $t_i$ are terms of sort $S_i$ and $f$ is an $n$-ary function
$(1 \leq i \leq n, \; n \geq 0)$
with domain types $S_i$ and range type $S_r$, i.e. $f: S_1 \times \dots \times
S_n \rightarrow S_r$, then $f(t_1,\dots,t_n)$
is a correct term of sort $S_r$. Variables and constants of any type are correct
terms.

If a variable with name $n$ is declared in the {\tt VAR}-section,
indexed variables $n_i$ where $i$ is a {\vs Number} are declared
implicitly. These indexed variables, which are accessible via
{\tt n\_i}, can be used like normal variables
in a term, but they are all distinct.

External objects can be integrated into mixed terms via coercion functions.
Coercion functions with range type $S_r$ convert suitable external objects
into terms of type $S_r$. To read and write external objects the algorithms
as declared in the {\tt EXTERNAL}-section are used.
\begin{quote}
{\em
{\bf WARNING:} It is within the responsibility of the user to check that
the external parsing functions specified under property {\tt XREAD} are
compatible with the term parser. The parser can easily be confused
by malformed external parsers!
} % end emph.
\end{quote}

\subsection{Default Settings}
There are several default settings for operator notation, precedence and associativity
if these properties are not explicitly specified in the corresponding
sections. These default settings are:

\begin{description}
\item[Notation:]
Constant notation for nullary operators, prefix for unary operators,
infix for binary operators and function notation for all other operators.
Coercion operators get prefix notation as default.
\item[Precedence:]
If no precedence is assigned to an operator the rule of the last section
is applied, i.e prefix operators without explicitly declared precedence
bind stronger than all infix operators, whether they are assigned a
precedence or not. Postfix operators without a specified precedence have
higher priority than all prefix and infix operators. Coercion operators
get the highest binding power, no matter if specified or not.
\item[Associativity:]
The default selection is {\tt NONE}. It is merely set for infix operators.
\end{description}

\subsection{Strange Precedence Declarations with Prefix and Postfix Operators}
\label{strangeprec}
Prefix and postfix operators can be written without parentheses around
the argument. If parentheses are used though, it may seem that this implies
a binding of the argument to that operator. This is not the case, however.

\begin{example} $ $
\begin{verbatim}
VAR      x,y: N;
OPERATOR s: N -> N;
         +: N, N -> N;
PREC     s < +;
\end{verbatim}
{\rm After these declarations the term {\tt s(x)+y} is interpreted as
{\tt s(x+y)}.}
\end{example}

This different interpretation with respect to the old parser occurs only
if pre- or postfix operators are assigned lower precedence than
infix operators. It can be avoided by using function notation instead of prefix
in case of prefix operators.

\subsection{Constants vs. Nullary Operators}
The main difference between constants and nullary operators is the built-in
assumption that constants are irreducible. Therefore the left-hand side
of an equation may not be a constant. This restriction does not apply to
nullary operators. Thus in \redux\ constants can be handled more efficiently
than operators.
Moreover, not all properties which can be used for nullary operators can also
be applied to constants.

\subsection{AC-Terms}
AC-operators have an additional feature when they are used in terms:
they can be parsed in flattened form, that is more than two arguments
are allowed in the argument list.

\begin{example}$ $
\begin{verbatim}
VAR
   w,x,y,z: N;
OPERATOR
   f,[: N, N -> N;
NOTATION
   f: PREFIX;
   [: ROUNDFIX ];
THEORY
   f,[: AC;
\end{verbatim}
{\rm
When using these declarations, the term {\tt [f(x,x),y,f(z,z,z,z),w]}
is correct and corresponds to the term {\tt [[[f(x,x),y],f(f(f(z,z),z),z)],w]}.
}
\end{example}

Infix AC-operators are automatically converted to flat representation.
