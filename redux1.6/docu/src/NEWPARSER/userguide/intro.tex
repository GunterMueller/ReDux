\section{Introduction}

The \redux\ system consists of a large set of algorithms as well as a collection of demo
programs covering the realm of term rewriting. For almost every practical purpose
(e.g. proving equational consequences, performing completion procedures, etc.)
it is necessary to
have an algebraic specification consisting of a signature description
and a set of equations. The language and the corresponding parser presented here
offer a notation for many-sorted algebraic specifications, which allows the
definition and usage in an easy and intuitive manner.
Throughout \redux, this language is used to perform input and output of
terms, axioms and complete specifications.

Algebraic specifications, called ``data types'', consist of constants, variables
and operators with associated types, which constitute the signature of the
data type. The equations usually following this specification make use of the
preceding definitions in order to build up complex terms and axioms.

One of the goals in the design process of the parser was to give the user a
simple and intuitive specification language, similar to normal mathematical notation.
This includes different common notations, such as prefix, postfix, infix, roundfix
(e.g. for absolute value operator) and lisp (e.g. for program verification) as well
as the usual function notation.
Further on, parentheses and blanks between symbols can be omitted whenever possible;
any prefix, postfix or infix operator can obtain a precedence in relation to another,
in order to allow the operators' usual binding powers; associativity can be specified
for infix operators.
Another objective was to put as little restrictions as possible on variable,
constant and function names.

A special feature of the parser is its ability to work with external objects.
External objects comprise, for example, \ALDES\ integers, 
lists or SAC-2 long integers.
Each of these object classes can be used almost like standard sorts, but
their objects additionally allow direct manipulation.

As a first example showing the outlines of a data type we present  
a specification of an additive group:

\begin{example}
$ $ \input{group.rdx}
\end{example}

In due course, lexical, syntactical and semantic aspects of the data type 
language will be described in full detail.
