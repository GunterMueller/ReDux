\section{Syntactical Structure}

The parser presented in this paper offers facilities for data type, operator,
../dtgram1.tex
constant, variable, axiom and term parsing. The different parsers are described
in the following sections.

There is no parser to read just coercion operators or external sorts.

\subsection{Data Type Parser}
The grammar of a \redux\ data type is shown in figure~\ref{dtgram1} and
figure~\ref{dtgram2}.
../dtgram2.tex
The symbols {\vs Ident}, {\vs XIdent} and {\vs Digit} 
have already been explained in the last section.
All other symbols that are
not defined within this diagram stand for identifiers, extended
identifiers or comma-separated lists of both kinds of identifiers.
They are written in a more descriptive way in the grammar
in order to make the grammar more intelligible. Which of the symbols
stands for a normal and which for an extended identifier (or list) can be seen
in figure~\ref{axgram}. The start symbol for data type parsing is
{\vs DataType}.
../axgram.tex

All of the following parsers (with the exception of the term parser)
allow input of parts of a data type. This includes constants, variables,
operators and axioms. It is necessary to have a valid data type structure
in order to use any of the aforementioned parsers, because at least
the sort information of a data type is needed to achieve this.
(In many cases, such as
axiom parsing, this is also the case for the complete signature.)


Additional objects parsed with one of the following parsers are not immediately
added to the data type used for these procedures, but this can be done
afterwards by using the appropriate \redux\ functions.

\subsection{Axiom Parser}
The axiom parser uses the following grammar.
It is used to parse a single axiom.
The number of the axiom is set to zero, but it can be set to any number
explicitly after parsing.
\begin{center}
\fbox{
\begin{minipage}{6in}
\begin{tabbing}
{\vs AxiomParse} \gprod {\vs Term} \ts{==} {\vs Term} \ts{;}
\end{tabbing}
\end{minipage}
} % end fbox
\end{center}

\subsection{Axiom Set Parser}
The axiom set parser is used to parse a non-empty set of axioms.
The grammar is almost the same
as in the equation part of the data type parser.
\begin{center}
\fbox{
\begin{minipage}{6in}
\begin{tabbing}
{\vs AxiomSetParse} \gprod \gplus{{\vs Axiom} \ts{;}} \ts{END}
\end{tabbing}
\end{minipage}
} % end fbox
\end{center}

\subsection{Constant Set Parser}
The constant set parser allows parsing of a non-empty set of constants.
The obligate sort information is taken from a formerly parsed data type.
\begin{center}
\fbox{
\begin{minipage}{6in}
\begin{tabbing}
{\vs ConstantSetParse} \gprod \gplus{{\vs Constant} \ts{;}} \ts{END}
\end{tabbing}
\end{minipage}
} % end fbox
\end{center}

\subsection{Operator Parser}
The operator parser is, by contrast to the operator set parser, an interactive
parser. I.e. additional information for an operator, such as notation
(fixity), associativity or precedence is interactively asked of the user.
The operator parser allows parsing of only one operator at once, as can be
seen from the grammar.
\begin{center}
\fbox{
\begin{minipage}{6in}
\begin{tabbing}
{\vs OperatorParse} \gprod {\vs OpName} \ts{:} {\vs DomainList} \ts{->}
   {\vs Range} \ts{;}
\end{tabbing}
\end{minipage}
} % end fbox
\end{center}

\subsection{Operator Set Parser}
The operator set parser allows parsing of operator sets. Additional operator
information, such as notation, precedence etc., can be specified after operator
declaration, as can be seen from the following grammar.
\begin{center}
\fbox{
\begin{minipage}{6in}
\begin{tabbing}
{\vs OperatorSetParse} \= \kill
{\vs OperatorSetParse} \> \gprod \gplus{{\vs Operator} \ts{;}} 
   \gstar{{\vs OpInfo}} \ts{END} \\
{\vs OpInfo} \> \gprod {\vs AssocDecl} \galt {\vs NotationDecl} \galt {\vs PrecedenceDecl}
   \galt {\vs PropertyDecl} \galt {\vs TheoryDecl}
\end{tabbing}
\end{minipage}
} % end fbox
\end{center}

\subsection{Term Parser}
In order to parser terms, the following parser can be used. There are
different versions of the term parser in order to parse ``related'' terms,
i.e. terms, which share common variables. Details on how variables
are handled during term parsing can be found in the Programmer's Guide.
\begin{center}
\fbox{
\begin{minipage}{6in}
\begin{tabbing}
{\vs TermParse} \gprod {\vs Term} \ts{;}
\end{tabbing}
\end{minipage}
} % end fbox
\end{center}

\subsection{Variable Set Parser}
The variable set parser makes parsing of non-empty variable sets possible.
Sort information is taken from a data type. Input must be according to the
following grammar.
\begin{center}
\fbox{
\begin{minipage}{6in}
\begin{tabbing}
{\vs VariableSetParse} \gprod \gplus{{\vs Variable} \ts{;}} \ts{END}
\end{tabbing}
\end{minipage}
} % end fbox
\end{center}

\vspace{2em}

There is another parser for ``operator-relation-operator lists''. This parser is
used by some demo programs to read in a term ordering (path ordering). For the sake of
completeness, the grammar of this parser is included in this manual.

\subsection{Operator - Relation - Operator List Parser}
\begin{center}
\fbox{
\begin{minipage}{6in}
\begin{tabbing}
{\vs OpRelOpListParse} \gprod \gopt{{\vs OpName} \group{ \ts{<=} \galt \ts{>=} \galt \ts{~}
  \galt \ts{\#} \galt \ts{<} \galt \ts{>}} {\vs OpNames}} \ts{;}
\end{tabbing}
\end{minipage}
} % end fbox
\end{center}
\vspace{1em}