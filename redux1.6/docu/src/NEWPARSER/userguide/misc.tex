\section{Miscellaneous}

\subsection{Error Detection and Recovery}
When a syntactical error occurs an error message and the faulty
input line are written to the output stream immediately.
On semantic errors it is possible that the error is not reported
before parsing of the complete line (or the term) has finished.
All input following an error up-to a semicolon or {\tt END}-keyword,
depending on the kind of parser that is used\footnote{Semicolon is
used only in the axiom, operator, term and operator-relation-operatorlist parser.},
is discarded.
The erroneous object must then be re-typed.

\subsection{Ambiguities During Term Parsing and Term Output}
As there are only few limitations on which identifier names
(i.e. names for variables, constants or operators) are allowed,
it may be the case that a written term is ambiguous.
\begin{example} $ $
\begin{verbatim}
VAR        f: N;
OPERATOR  ff: N -> N;
         fff: N -> N;
NOTATION  ff: PREFIX;
         fff: POSTFIX;
\end{verbatim}
{\rm
The term {\tt ff((f)fff)} may be written as {\tt ffffff}, but can not
be re-parsed afterwards (principle of longest match, see below).
}
\end{example}
During parsing there is one rule to disambiguate term input: Always
match the longest possible identifier (principle of longest match).

Thus, to enter the term of the example above without parentheses, one
could write: {\tt ff f fff}.

On term output similar problems arise. They are coped with by
writing (depending on term output mode) parentheses around subterms
or spaces between identifiers, when necessary.

