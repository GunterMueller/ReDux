\section{Lexical Structure}

The lexical structure of \redux\ data types describes the low-level input 
and output on character level.
Lexemes, the smallest entities with a definite meaning,
are built up from letters (upper and lower case), digits
and ``special'' characters. These special characters are:

{\tt
\begin{verbatim}
! @ $ % * _ & + ^ | ~ ( ) [ ] { } ` ' " . \ / ? < = > , ; : # -
\end{verbatim}}

White spaces (blanks, newlines and tabulators) are treated as delimeters, 
and for this reason they are not allowed within lexemes.

The logically splitted structure of a data type (signature vs. equations) is
reflected by a similar devision on the lexical level. Thus, the lexical
structure for specifying signatures is different from that of equations.

\subsection{Signature Definition}

Within the signature part of a data type, which consists of all but the
{\tt AXIOM}-section\footnote{Sections are described later, cf.\ 
figure~\ref{dtgram1}},
lexemes can be devided into several groups: Keywords, reserved lexemes,
(normal) identifiers and extended identifiers.

\begin{description}
\item[Keywords:]
Keywords consist of letters and have a predefined meaning
in the signature part of a data type specification. Therefore, they can not
be used otherwise, e.g. as identifiers. 
The keywords of the data type definition language are:
\begin{flushleft}
{\tt AC, ASSOC, AXIOM, COERC, COERCION, COM, CONST,
CONSTANT, DATATYPE, END, EXTERNAL, FUNCTION, INFIX,
KBINDEX, KBWEIGHT, LEFT, LISP, NONE, NOTATION, OPERATOR,
POSTFIX, PREC, PREFIX, PROPERTY, RIGHT, ROUNDFIX,
SORT, THEORY, VAR, XEQ, XFG, XINT, XLT, XREAD, XTERM, XWRITE}
\end{flushleft}

\item[Reserved lexemes:]
Reserved lexemes are of similar nature as keywords, in the way that they
have a definite, unchangeable meaning. Reseved lexemes consist of special characters,
as they are defined above. Like keywords, they are not allowed as identifiers.
These reserved lexemes are:

{\tt\verb|( ) , == ; %%|}

The first four of them ({\tt\verb|( ) , ==|}) have a special meaning in the
equational part of the data type definition.
The semicolon indicates the end of a (definition) command, axiom or term.
(It is not used as a command separator!)
Two percent signs ({\tt\verb|%%|}) make up the line-comment symbol.
All characters after the line-comment symbol upto a newline\footnote{Newlines
are also inserted by the ALDES I/O-system after a specific number
of characters. This property may be ignored, as there is another limitation
on line-length in the I/O-system. See the ALDES manuals for details.}
are skipped by the parser.

Moreover, identifiers can not have any of the special characters
{\tt\verb|( ) , ;|} as components. Because the lexemes {\tt =} and {\tt \%}
are not reserved, they are valid identifiers. Identifiers starting with
{\tt \%\%} or {\tt ==} are prohibited due to their special meaning as 
line-comment
symbol and equivalence indicator in the {\tt AXIOM}-section respectively, but
they may be suffixes of valid identifiers.

\begin{example}
{\rm {\tt <==}, {\tt\verb|%|} and {\tt\verb|x%%|} are valid identifiers,
{\tt ==>} and {\tt\verb|%%a|} are prohibited.}
\end{example}

\item[Identifiers:]
Normal identifiers consist of sequences of letters (upper and lower case)
and digits, with
the first character being letter. In a more formal way, we can write:

{\vs Ident} \gprod {\vs Letter} \gstar{{\vs Letter} \galt {\vs Digit}},

where {\vs Letter} is any character of {\tt a,b,c,\dots,x,y,z,
A,B,C,\dots,X,Y,Z} and
{\vs Digit} is one of {\tt 0,1,2,3,4,5,6,7,8,9}.

\item[Extended identifiers:]
Extended identifiers (by contrast to normal identifiers) may use special characters and
need not start with a letter.
Formally written:

{\vs XIdent} \gprod \gplus{{\vs Letter} \galt {\vs Digit} \galt {\vs SpecChar}},

where {\vs Letter} and {\vs Digit} are defined as above, and {\vs SpecChar}
indicates a special character.

Which special characters are allowed and which not, is context-dependent and
refers to the reserved characters in the corresponding section.

Those special characters which are not allowed in a given context can be used, if
they are preceded by an escape character. The escape character is the backtick
({\tt `}). The backtick itself is made available by two subsequent backticks.

The special characters which can be used in all sections (without escape
character) are:

{\tt\verb:! @ # $ * _ & + ^ | ~ [ ] { } ' " . \ / ? >:}

The context-dependent special characters as shown in table~\ref{contdep}
are not allowed without escape character in the respective sections.
All other special characters can be used without escape character,
as long as they are not reserved lexemes or one of the special characters
{\tt\verb!% ` =!}.  Reserved lexemes can never be used
as identifiers.

\begin{table}[htbp]
\begin{center}
\begin{tabular}{|l|l|l|}
\hline
Context/Section & Reserved characters\\ \hline
\hline
Data type name & none \\ \hline
External sort declaration & {\tt\verb!= :!} \\ \hline
Constant declaration & {\tt\verb!:!} \\ \hline
Variable declaration & {\tt\verb!:!} \\ \hline
Operator declaration & {\tt\verb!:!} \\ \hline
Coercion function declaration & {\tt\verb!:!} \\ \hline
Precedence declaration & {\tt\verb!< =!} \\ \hline
Associativity declaration & {\tt\verb!:!} \\ \hline
Notation declaration & {\tt\verb!:!} \\ \hline
Theory declaration & {\tt\verb!:!} \\ \hline
Property declaration & {\tt\verb!= :!} \\ \hline
\end{tabular}
\end{center}
\caption{Context-Dependent Reserved Special Characters}
\label{contdep}
\end{table}

Though not recommended, it may be useful to be able to use keywords as identifiers.
This can be achieved by appending an escape character in front of the keyword.
\begin{example}
{\rm {\tt`END} makes the identifier {\tt END} available in the equational part.}
\end{example}

Unnecessary escape characters are ignored.

\end{description}

\subsection{Equational Definition}
The equational definition part of a data type consists of the {\tt AXIOM}-section.
In contrast to the signature part there is no escape character
in this section. Identifiers which needed an escape character on definition
must be written without here. There are only five lexemes with a special
meaning in this section:

\begin{itemize}
\item
The opening and closing parentheses ({\tt (} and {\tt )}) are used around function
arguments and to determine evaluation order.
\item
The semicolon ({\tt ;}) is used to indicate the end of an equation (axiom)
or term.
\item
The comma ({\tt ,}) serves as argument separator.
\item
The equivalence sign ({\tt ==}) is used to express equality within
axioms.
\item
The line-comment symbol ({\tt\verb|%%|}) has the same semantic as in the
signature part.
\end{itemize}

Parsing of terms or axioms may be ambiguous due to the parser's ability to read
sequences of identifiers which are not separated by white spaces.
These ambiguities are
resolved (as usual) by a principle of longest match (in conjunction with white
spaces as delimiters).

It should be mentioned, that writing a data type to an output stream normally
uses a format simplifying the application of escape characters. Those users, who 
are only
interested in using the term and axiom parser, but do not want to write their own 
data types, thus need not know anything about the rules for escape characters 
given above: On printing the data type, all (extended) identifiers
are written exactly the way they can be re-typed for term or axiom input. I.e. escape
characters are omitted, though they would be necessary in a signature specification
for parsing.\footnote{There is an output option, which disables this feature, thus
permitting rescanning of printed data types.}

\begin{note}
The \ALDES\ I/O-system restricts the length of input lines to a fixed 
number of characters.
All additional characters occurring in an input line which is longer 
than this fixed number are ignored. 
See the \ALDES\ manuals for details.
\end{note}
