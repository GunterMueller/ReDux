\section{Demo Program: {\tt parser}}

There is a demo program ({\tt parser}) to test all the parsing and
output functions. When the program is started, the user has the following
choice:

\begin{verbatim}
/--------\
| PARSER |
\--------/
Loading your data type.
 
DATATYPE LIST;
SORT    INT, LIST;
CONST   0: INT;
        nil: LIST;
VAR     x, y, z: INT;
        L, M, N: LIST;
OPERATOR
        s: INT -> INT;
        p: INT -> INT;
        +: INT, INT -> INT;
        @: LIST, LIST -> LIST;
        [: INT, LIST -> LIST;
        |: LIST -> INT;
NOTATION
        s, p: PREFIX;
        +, @: INFIX;
        [: ROUNDFIX ];
        |: ROUNDFIX |;
AXIOM
 [1] nil @ L == L;
 [2] [x,L] @ M == [x,L @ M];
 [3] spx == x;
 [4] psx == x;
 [5] 0 + x == x;
 [6] sx + y == s(x + y);
 [7] px + y == p(x + y);
 [8] |nil| == 0;
 [9] |[x,L]| == s|L|;
END
 
Do you want to parse a (t)erm, an (a)xiom, an (o)perator,
term(s), a(x)ioms, (v)ariables, (c)onstants, o(p)erators,
a (r)elation list, write the current (d)ata type,
change term (w)rite mode or (q)uit ? *
\end{verbatim}


All but the last three selections start the corresponding parsers.
When the user types ``w'' he or she is able to change the term output mode:
\begin{verbatim}
Current write mode:
term output: parentheses...
   1) around subterms with higher precedence = off
   2) around subterms with same precedence   = on
   3) to indicate associativity              = on
   4) around lisp-operator's infix arguments = on
   5) around prefix operator's argument      = off
   6) around postfix operator's argument     = off
   7) around infix operator and arguments    = off
term output:
   8) debugging          = off
data type output:
   9) rescanable output = off
global output:
  10) full debugging     = off
Toggle which flag?    [1-10, 0 for none] *
\end{verbatim}

The first seven switches determine when parentheses are printed. With
all options set to ``{\tt off}'', all superfluous brackets are omitted.
Toggles eight to ten control other, different aspects of term and data type
output.

The different output modes are explained in the following in full detail:
\begin{description}
\item[Parentheses around subterms with same precedence:]
  In the case of equal precedence of a term's top operator and one
  of its operators at a direct subposition, it is not always necessary
  to write parentheses around the subterm. According to the rule given in
  section \ref{precrule} this happens e.g. if an infix operator occurs at top
  position and a postfix operator at subterm-position.
  With this option switched on, parentheses are always printed when
  the precedence of both aforementioned operators is the same.
  Turning off this option results in omitting brackets whenever it is possible
  due to equal precedence.
\item[Parentheses around subterms with higher precedence:]
  Normally, no parentheses are necessary around a subterm whose operator
  has a higher precedence than the term's top operator. Switching on this
  option produces parentheses in this case, though.
\item[Parentheses to indicate associativity:] When associativity is specified
  for an infix operator, brackets can be left away in sequences of such
  operators. With this switch set to ``{\bf on}'' parentheses are written, however.
\item[Parentheses around lisp-operator's infix arguments:] Writing a lisp
  operator's infix arguments without parentheses may look a bit confusing, though
  it is correct. To avoid this, set this option to write parentheses for a more
  user-friendly output.
\item[Parentheses around prefix operator's argument:] Prefix operators (they
  are always unary) are normally written without parentheses around the argument.
  This can be avoided either using function notation or setting this option.
  Note, however, that these brackets do not change precedence, as was shown in
  section \ref{strangeprec}.
\item[Parentheses around postfix operator's argument:] As in the case of prefix
  operators, this switch set to ``{\bf on}'' forces brackets around postfix operator's
  arguments. In contrast to prefix operators, this is the only possibility to
  achieve this.
\item[Parentheses around infix operator and arguments:] This option produces
  brackets around an infix operator and its arguments as a whole. This switch
  was implemented mainly for compatibility reasons (old parser).
\item[Debugging for term output:] This option not merely writes the terms, but also
  the address of each term position in the \ALDES\ {\tt SPACE}-array. This can be
  useful to examine the correct handling of variables.
\item[Rescanable input for data types:] In some cases data type output cannot
  be reused as input for the parser. 
  The reason for this is that escape characters
  are omitted in the signature part of the data type. 
  To force these escape characters
  to be written on data type output and thus allowing rescanning of
  printed data types  this switch can be used.
\item[Full debugging for global output:]
  Switching on full debugging, makes \redux\  use not the normal output
  routines, but the \ALDES-function {\tt UWRITE} instead.
  So the internal structure of data types and other objects is made 
  accessible to  the user (programmer).
\end{description}
