\subsection{The Parse Process}

The subsequent steps of a typical parse run are preferably shown for
parsing a data type. 
This process includes parsing signature definitions
as well as terms (within axioms) and therefore covers most parts of
the algorithms provided by the parser modules.

Parsing normally starts with a call to one of the interface functions
of module {\tt objparse.ald} (here {\tt DTPRS}). The interface routine
directly invokes the main parser ({\tt YYPARS} of file {\tt parse.y}).
The parser scans the input file and builds up the appropriate parts
of the data type step by step. Therefore it uses auxiliary routines
from {\tt parsax.ald}. As soon as certain parts of the data type are parsed
(e.g. constants) they are checked for semantical correctness ({\tt check.ald})
and inserted into an initially empty data type ({\tt insert.ald}).
In case of syntactical errors, these are handled directly by
the main parser.
Semantical errors which are found either by {\tt check.ald},
{\tt insert.ald} or {\tt parsax.ald} are reported by {\tt semerr.ald}.

This first part of the parse process makes use of the symbol table
only to insert symbols and to check whether objects, which are to be assigned
special properties (e.g. notation, precedence, associativity),
have been declared properly before.
All other lexical analysis work is done directly by the LEX-scanner in file
{\tt scan.l}. Especially, lexemes with definite predefined meaning
(keywords, special lexemes) make use of the statical lexical analyzer.

The second step of the process consists of the parsing of the
{\tt AXIOM}-section.
This step heavily depends upon symbol table access.
The parser successively uses the symbol table (via the scanner) to look
for a suitable symbol. It is in this step where the principle of longest
match has its main importance.
After successfully parsing a term, it has to be transformed according to the
precedence and associativity declarations of the
signature part of the declaration. 
This is done by {\tt fxpras.ald}.

These two steps almost complete the parsing process. Post processing
is still necessary, though. This includes setting of default values,
ambiguity checks and external (coercion) operator post processing.
