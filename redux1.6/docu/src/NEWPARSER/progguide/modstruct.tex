\section{How the Parser Works}

\subsection{Module Structure}

The new \redux\ parser is split up into several modules, as can be seen
in Figure~\ref{modules}. 
In the sequel, the different modules and
their purposes will be explained in full detail.

\begin{figure}[htbp]
\begin{center}
\vspace{1cm}
\epsfig{file={modules.eps},angle=270}
\caption{The Module Structure of the Parser}
\label{modules}
\end{center}
\end{figure}

\begin{description}
\item[objparse.ald:] This \ALDES-module contains interface routines to the
parser. It provides the programmer with functions to parse terms, axioms,
data types etc.
\item[parse.y:] This file contains the YACC parser. It consists mainly of
the grammar (and semantic actions associated with each production rule)
of data types and other structures intended to be parsed.
Error handling for syntactical mistakes is also done here.
\item[scan.l:] The scanner (or lexical analyzer) makes up the biggest part
of this file. It is held in LEX-syntax with some extensions\footnote{These
extensions comprise renaming of the scanner main function and changing
of the input file access method.} that make it
necessary to use GNU FLEX instead of standard LEX to translate it.
Symbol table handling is not done within this module.
\item[symlist.ald:] This file is responsible for symbol table management.
During parsing of signature descriptions, symbols are entered into the
data type symbol list; thereafter these symbols can be used to
build terms and axioms. During term parsing almost all of the lexical
analysis is done in this module.
\item[parsax.ald:] The file {\tt parsax.ald} contains auxiliary parsing
routines. It is called from the main parser of file {\tt parse.y} to
do most of the semantical actions. The procedures and functions would
have been included into the parser, if C were the only implementation
language used. To keep the main parser from coping with \ALDES\ memory
management problems they were excluded;
thus merely \ALDES-function calls occur within the
parser. Most of the work is done in {\tt parsax.ald} which is to a
great part responsible for construction of the appropriate data structures.
\item[check.ald:] To check the semantical correctness of parser input
procedures of this file are called.
\item[semerr.ald:] Whenever a semantic error was found by the parser or
by auxiliary routines this module is used to print out error messages.
Warnings are also reported by a function of this module.
\item[insert.ald:] Insertion of parts of a data type, e.g. variables,
constants, operators, etc. into a data type is maintained within this file. 
Moreover, it provides the user with functions to add new objects into
the symbol table of a prior parsed data type.
\item[ambig.ald:] As the parser puts only few restrictions on which
object names may be used, and blanks can always be  omitted,
problems arise during term output about when it is unavoidable to
write blanks between objects names to keep output unambiguous and
to enable rescanning of terms and equations. Therefore every signature
object is extended by an {\em ambiguity list}\footnote{See the section
on data structures for a more thorough description of ambiguity lists.},
where possible follow-up object names are stored. Whenever those
follow-up object names occur during term output blanks or
parentheses, depending on term output mode, have to be written
additionally.
This file contains the routines necessary to handle ambiguities.
\item[fxpras.ald:] The structure of terms is, according to the grammar,
not always uniquely determined. Thus, semantical information, such as
precedence and associativity, is needed to transform the textual description
of a term into its internal, unambiguous representation.
The functions in this module allows to restructure terms
(fxpras: {\it f}i{\it x} term, according to  {\it pr}ecedence and 
{\it as}sociativity declarations) 
in order to fulfill the data type declarations.
\item[extern.ald:] Input and output functions for external objects (i.e.
ALDES or SAC2 objects) are gathered in this file.
\end{description}

The complete parser consists of some additional files not mentioned here,
but their function is either related to output of \redux\ objects, 
or they provide auxiliary routines (e.\,g., initialization) which are not 
vital to understand the parse process. 
Both kinds of files will be explained later.
