\section{Output Modes}

The write routines of the new \redux\ parser allow different modes
of term and data type output. 
Switching between different modes is possible at run-time.

All output modes are registered in one global variable ({\tt OTMODE};
the lowest ten bits of this integer-valued variable control ten
different output mode flags, which will be explained later).

The different output modes can be subdivided into alternatives for
term output and selection of global output appearance.

\subsection{Term Output Modes}
Term output modes provide a detailed control on usage of parentheses
and a simple debugging facility. The purpose of the different term
mode flags is explained in Table~\ref{tomodes}.

\begin{table}[htbp]
\begin{center}
\begin{tabular}{|p{0.8in}|l|p{2.5in}|}
\hline
Bit Position in {\tt OTMODE} & Output Mode Flag & Meaning \\
\hline \hline
0 & TOPLOWPREC &  parentheses around subterms when main term has lower
     precedence \\ \hline
1 & TOPEQPREC & parentheses around subterms with same precedence
     as main term \\ \hline
2 & TOPASSOC & parentheses to indicate associativity \\ \hline
3 & TOPLISPARG & parentheses around lisp operator's infix arguments \\ \hline
4 & TOPPREARG & parentheses around prefix operator's argument \\ \hline
5 & TOPPOSTARG & parentheses around postfix operator's argument \\ \hline
6 & TOPINFIX & parentheses around infix operator and arguments \\ \hline
7 & TODEBUG & debugging of term output \\ \hline
\end{tabular}
\end{center}
\caption{Term Output Mode Flags}
\label{tomodes}
\end{table}

\subsection{Global Output Modes}
By means of the global output mode flags rescannable data type output
and another debugging mode can be selected. 
These flags are explained in Figure~\ref{gomodes}.

\begin{table}[htbp]
\begin{center}
\begin{tabular}{|p{0.8in}|l|p{2.5in}|}
\hline
Bit Position in {\tt OTMODE} & Output Mode Flag & Meaning \\
\hline \hline
8 & DTORSCAN & rescannable output of data types \\ \hline
9 & OFULLDEBUG & full debugging of global output \\ \hline
\end{tabular}
\end{center}
\caption{Global Output Mode Flags}
\label{gomodes}
\end{table}

For a more detailed description of the different flags see \NPUG.
All combinations of these flags result in valid term output modes.

Note, that turning on full debugging mode results in the output module
ignoring the state of all other output mode flags.

\subsection{Standard output modes}

To simplify output mode handling there are three default modes
which implement suitable settings for most purposes. Which flags
are set when using one of these standard modes can be seen from
Table~\ref{standardom}

\begin{table}[htbp]
\begin{center}
\begin{tabular}{|p{1.5in}|p{1.5in}|p{2in}|}
\hline
Standard Output Mode Name & Description & Flags Set \\
\hline \hline
OMSPARSE & sparse use of parentheses & none \\ \hline
OMNORMAL & normal use of parentheses & TOPEQPREC, TOPASSOC, TOPLISPARG \\
   \hline
OMCOMPLETE & maximal sensible number of parentheses &
   TOPLOWPREC, \newline TOPEQPREC, TOPASSOC, TOPLISPARG, TOPPREARG, TOPPOSTARG,
   TOPINFIX \\ \hline
\end{tabular}
\end{center}
\caption{Standard Output Modes}
\label{standardom}
\end{table}

Setting one of these standard output modes can be done by explicitly
setting the global variable {\tt OTMODE} to one of the three aforementioned
standard output mode names.

Toggling of individual flags is performed by functions explained
in the next section.

\subsection{Controlling Output Modes}

The parser allows setting of the output mode by three functions provided
by file {\tt write.ald}.

\begin{description}
\item[t $\leftarrow$ OMODGT(F):] This functions can be used to get the
state of an individual output mode flag {\tt F}. The returned value
{\tt t} is either {\tt TRUE} or {\tt FALSE} depending on whether flag
{\tt F} is set or reset. The output mode is not modified by this function.
\item[OMODST(F,t):] To make changes of the output mode, this function can
be used. {\tt F} is any output mode flag and {\tt t} is either {\tt TRUE}
or {\tt FALSE}. 
Then flag {\tt F} is set or reset, depending on {\tt t}.
\item[QOPMOD():] This function (query output mode) allows interactive
setting of a new output mode. All flags may be changed by this function,
which has no return value, but modifies the global variable {\tt OTMODE}.
\end{description}
