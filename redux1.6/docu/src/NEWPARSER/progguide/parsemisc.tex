\subsection{Miscellaneous}

Some special programming techniques which may not be obvious at a
first glance will be explained in the following.

\subsubsection{Different \redux\ Parsers Using One Common YACC Parser}
The new \redux\ parser provides the user with a rich collection of
parsers for many purposes. It seemed more practical, however, to
integrate these parser into one YACC parser in order to keep executable
program files at a reasonable size.

This was achieved by including an additional grammar production rule at top
of the grammatical description:
\begin{verbatim}
start: STERM term
     | SAXIOM axiom
     | SDATATYPE datatype
     | SOPERATOR operator
     | SOPSET opset
     | SAXSET axset
     | SCONSTSET constset
     | SVARSET varset
     | SOPRELOPLIST opreloplist
     ;
\end{verbatim}
In this production, {\tt start} is the starting nonterminal for all
parser activities. {\tt term}, {\tt axiom}, etc. are nonterminals
indicating entry points to the different parsers. The pseudo-terminals
{\tt STERM}, {\tt SAXIOM}, etc. are provided by the scanner when it
is called the first time. It then returns a suitable token without
reading any input. Which token it returns depends on which parser
interface routine is used to start the YACC parser. Passing of
start pseudo-tokens from ReDuX interface routines of file {\tt objparse.ald}
to the LEX-scanner is done by means of the global variable
{\tt STSYMB}.

After having produced the pseudo-token, the lexical analyzer returns
to normal operating mode and scans the input file before providing
the parser with any further tokens.

Additional information is available in the comments of the files {\tt scan.l}
and {\tt parse.y}.

\subsubsection{ALDES Memory Management within YACC Parser}

Though most of the \ALDES\ source code of the parser is excluded
from the C YACC parser, a small fraction of \ALDES\ code remains part
of the C parser file. To prevent the \ALDES\ memory management from
corrupting the data structures under construction, it is
unavoidable to indicate to the  \ALDES\ system which parts of memory are
used for which structures.

As the garbage collector scans the SPACE array for those cells
which are reachable from the current \ALDES\ stack, it is necessary to
push all unsafe variables onto this stack.

This is done by the C function {\tt newo}:
\begin{verbatim}
AOBJ newo(AOBJ o, int f)
{
  if(f==0) IUP(1); else IDN(f-1);        /* INDEX=INDEX-f+1; */
  STACK[INDEX] = o;

  return o;
}
\end{verbatim}
This function pushes the object {\tt o} onto the \ALDES\ stack, posterior to
having freed {\tt f} cells on top of the stack.

Adapting the parser to one's needs has to respect these facts.
Roughly speaking, it is correct to free $n$ cells for a production
rule with $n$ nonterminals on its right hand side\footnote{Note, that
only those nonterminals, whose semantical values represent lists
have to be considered. Identifiers and extended identifiers
do not have to be saved on the stack.}.

User extensions which lead to stack errors are usually reported by
the parser during run-time.

\subsubsection{Extended Identifiers and LEX Environments}
\label{environments}

Which characters are allowed for extended identifiers depends
on the section\footnote{See \NPUG\ for a description of sections.}
they are used in.
All characters that are part of a reserved lexeme in a section should
not be accessible without escape character. This can be controlled
by setting up environments in the LEX scanner.

The character classes for extended identifiers and different environments
are declared in {\tt scan.l}. These character classes have the following
meaning:
\begin{description}
\item[EXTENSION:] Those special characters that are allowed in all sections
 without preceding escape character.
\item[ALL:] This is the complete set of special characters.
\item[SPECIAL:] In this character class all characters of EXTENSION are
 contained as well as all characters in combination with a preceding
 escape character.
\end{description}

User extensions of the scanner which change any of these character classes
should be performed with greatest care. It is at least necessary to adapt
{\tt type.ald}.


\subsubsection{External Object Parsers in Combination with LEX Scanner}
\label{extparse}
In order to parse external objects, the scanner has to switch between
normal operation and external object parsing accordingly.
As LEX scanners maintain an internal buffer of input characters and
perform look-ahead, whereas external parsers directly access ALDES I/O-routines,
the situation frequently occurs that the LEX scanner has alread read some
characters into its internal buffer, which the external parser
assumes to see next when calling low-level I/O-routines.
Thus these look-ahead characters have to be pushed back to the low-level
input stream before calling an external parser.

Unfortunately, LEX does not provide a facility to let the user know the number of
look-ahead characters currently in use. Therefore a look-ahead counter
is maintained explicitly within the LEX scanner.\footnote{The variable {\tt lac} in
file {\tt scan.l} is the look-ahead counter.}

The look-ahead counter is incremented whenever a new character is read from
the ALDES input stream (in function {\tt YY\_INPUT})
and decreased appropriately whenever a correct lexeme was recognised.

To simplify handling of the look-ahead counter and to avoid faulty
settings of this counter, the macro {\tt RETALAC}
(return and adapt look-ahead counter) should be used instead of an ordinary
return instruction in the scanner. How this macro is employed will be explained
now.

When writing LEX scanners, it is usually sufficient to simply return
the token of the scanned lexeme (after putting the semantic value for this
token into the variable {\tt yylval}).
Here, such a practice would invariably lead to a wrong setting of the
look-ahead counter, thus making external object parsing impossible.
Therefore, instead of {\tt return(tok)} the programmer should use
{\tt RETALAC(tok)} whenever the length of the lexeme is known to the
LEX scanner. This holds as long as no calls to {\tt input} or {\tt YYINPT}
are performed within the C-code associated with a lexeme (action part of a
rule).
Whenever additional calls to {\tt input} or {\tt YYINPT} occur in the
action part (e.g. symbol table look-up), adaption of the look-ahead counter
has to be made manually. This adaption can easily be achieved be inserting
an instruction like {\tt lac -= lexeme\_length} before returning the
corresponding token, where {\tt lexeme\_length} is the length of the complete
lexeme just scanned.
