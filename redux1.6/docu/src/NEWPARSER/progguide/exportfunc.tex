\section{Exported Functions}

Most  functions of the parser modules are not intended to be called
from user source code.
The interface routines to the parser, write routines and some other
functions are  exported and expected to be called from user programs,
however.
These functions will be shown in the next sections.

\subsection{Parser Interface}

The parser interface incorporates functions to parse data types,
parts of data types and provides the user with different term
parsers. They are contained in file {\tt objparse.ald}.

\begin{description}
\item[AS $\leftarrow$ ASPRS(DT):] This parser can be used to
parse non-empty
axiom sets. {\tt DT} is a data type which is usually parsed before the
axiom set parser is called,
{\tt AS} is a list of axioms.
Signature information is taken from data type {\tt DT}. The parsed
axioms {\tt AS} are not inserted into the data type.
\item[A $\leftarrow$ AXPRS(DT):] This parser is intended to parse one
axiom {\tt A}
(equation) according to the signature information provided by
data type {\tt DT}. The axiom number of the parsed axiom is set to zero,
but can be changed afterwards if desired. Axiom {\tt A} is not added
to the data type.
\item[CS $\leftarrow$ CSPRS(DT):] This is the constant set parser.
Similar to the parsers before, sort information is taken from
data type {\tt DT} to parse {\tt CS}, a list of constants. Parsed
constants are not implicitly added to the data type.
\item[DT $\leftarrow$ DTPRS():] To parse a complete data type,
this function has to be called. Note, that the data type parser
aborts program executions if an error was found in the data type
specification.
\item[O $\leftarrow$ OPRS(DT):] This is an interactive operator parser.
{\tt DT} is a data type which provides the operator parser with
the necessary signature declarations. The operator {\tt O} is
read using the YACC parser, before additional properties of operator
{\tt O}, such as notation, associativity or precedence, can be
interactively set. This parser is not recommended for batch file usage.
Use {\tt OSPRS} instead. {\tt O} is not inserted into {\tt DT}.
\item[OS $\leftarrow$ OSPRS(DT):] Operator sets can be read with this parser.
Specification is similar to {\tt OPRS}, but additional properties
can be added using the same notation as in data type descriptions.
Signature information is taken, as usual, from data type {\tt DT}. The
list of operators {\tt OS} is not inserted into {\tt DT}.
\item[OROLRD(; O,R,OS,ne):] This parser (operator - relation - operator list)
is needed by some demo programs to read in information for a term ordering
(path ordering). 
{\tt O} is a symbol (denoting a constant or operator name), {\tt R} is a
string 
(in $\{$ {\tt >=}, {\tt <=}, {\verb+~+}, {\tt \#}, {\tt <}, {\tt >}$\}$)
denoting a relation, {\tt OS} is a list of symbols 
(each a constant or operator name) and {\tt ne} signals whether a valid input
was read in.
\item[T $\leftarrow$ TEPRS(DT):] This is the main term parser. Using
signature information from data type {\tt DT}, a term is read.
\item[TPRSC(DT,V; T,V'):] This term parser (parse term in context)
should be used to parse related terms, i.e. terms, which share
common variables, e.\,g., the left and right hand sides of an axiom.
{\tt DT} is a data type, which provides a signature description,
V is a (possibly empty) parser variable dictionary.
To parse related terms, a first call to {\tt TPRSC} should be
performed with {\tt V=()}. The parser then returns the parsed term
{\tt T} in conjunction with a new, updated parse variable dictionary
that can be used for further term parsing. Entries in the variable dictionary
provide the parser with information about which variables (and which indices)
are already known from preceding calls to {\tt TPRSC}.
\item[VS $\leftarrow$ VSPRS(DT):] This parser is responsible for
parsing of variable sets. Specification is almost like that for {\tt CSPRS},
with the only difference that {\tt VS} is a list of variables.
\end{description}

All these parsers are based on the assumption that the global
parser variables
were properly initialized (by {\tt ITGLPR}) before their first usage.

\subsection{ReDuX Object Write Functions}

The new parser demands a change of the output functions to be
adapted to the new data type style. All functions provided by
the old write module are still provided (in file {\tt write.ald}):

{\tt ASWRT}, {\tt AXWRT}, {\tt CONWRT}, {\tt CSWRT}, {\tt DTWRT},
{\tt OPWRT}, {\tt OSWRT}, {\tt RRWRT}, {\tt SBDIW}, {\tt SBWRT},
{\tt SLWRT}, {\tt SWRT}, {\tt TLWRT}, {\tt TDICT}, {\tt TDIWRT},
{\tt TWRT}, {\tt VAWRT}, {\tt VSWRT}, {\tt VDWTS}, {\tt YXSWRT},
{\tt YXWRT}

Their specifications were left unchanged.
Besides these functions, the new parser defines the following output
functions:

\begin{description}
\item[OSWRTN(OS):] Write set of ``normal'' operators. {\tt OS}
is a set of operators, probably containing also some coercion operators.
The set {\tt OS} is printed ignoring all coercion operators.
\item[CESWRT(OS):] Write set of coercion operators. {\tt OS} is a
set of operators, probably a mixture of normal and coercion operators.
All coercion operators of {\tt OS} are printed, normal operators
are ignored.
\item[FXWRT(OS):] Write fixity for operator set. {\tt OS} is a set
of operators (maybe also coercion operators). The notation information
for all operators in the set {\tt OS} is printed.
\item[ASCWRT(OS):] Write associativity for set of operators. {\tt OS}
is a set of operators. The associativity status for all operators
which have set the {\tt ASSOC} property is printed.
\item[PRWRIT(OS):] Write precedence for set of operators. {\tt OS}
is a set of operators. A precedence list containing all operators of
{\tt OS} which have set the {\tt PREC} property is printed.
\item[USWRT(OS):] Write unification status. Unification status for
all operators of list {\tt OS} which belong either to theory {\tt AC} or
{\tt COM} is printed.
\item[PPWRT(OS):] Property write for operator set {\tt OS}. This includes
properties {\tt COERC}, {\tt KBINDEX}, {\tt KBWEIGHT} and
{\tt XINT}.
\item[SRTWRT(S):] Write sorts. {\tt S} is a list of sorts that is printed.
\item[SOWRIT(SO):] Write symbol of signature object. {\tt SO} is a signature
object (or term). The symbol {\tt TNAME(SO)} is written with
the state of the output mode flags being considered.
\end{description}

\subsection{Data Type Manipulation Tools}

The parser modules provide a variety of functions for data type manipulation.
Most of them cover insertion of additional objects like constants, operators,
etc. into a data type.
Partly, the functions are contained in file {\tt insert.ald}:

\begin{description}
\item[INSAXS(DT,AS):] Insert axiom set into the data type. {\tt DT} is
the data type where the axioms of {\tt AS} are to be added.
No errors can occur here, but the user should make sure that no
axiom numbers are used multiply.
\item[INSCNS(DT,CS; IS,ES):] Insert constants into the data type. {\tt DT}
is a data type into which the constant set {\tt CS} is to be inserted.
Then {\tt IS} is the set of constants that were successfully added,
{\tt ES} is the set of constants that could not be added to the data
type because of name conflicts. {\tt IS} concatenated with {\tt ES}
is the complete constant set {\tt CS}.
\item[INSOPS(DT,OS; IS,ES):] Insert operators into the data type. {\tt OS}
is a set of operators, all other parameters have the same meaning as
in {\tt INSCNS}.
\item[INSVRS(DT,VS; IS,ES):] Insert variables into the data type. {\tt VS}
is a variable list, all other parameters have the same meaning as in
{\tt INSCNS} or {\tt INSOPS}.
\item[t $\leftarrow$ INSOBJ(DT,OL):] Insert objects into the data type
symbol table. 
{\tt DT} is a data type, {\tt OL} a list of constants,
variables and/or operators. These objects are inserted into {\tt DT}'s
symbol list. If an error occurs due to name conflicts {\tt t=FALSE}.
In this case it is not assured that all objects really have been inserted
into the symbol table. Otherwise {\tt t=TRUE}.
\item[t $\leftarrow$ INSOB(DT,O):] Insert object into data type's
symbol table. Specification is similar to {\tt INSOBJ}, but only
one object can be inserted using this function.
\end{description}

The file {\tt dtax.ald} offers some more operations on data types:

\begin{description}
\item[P $\leftarrow$ DTGET(DT,I):] Get data type property. {\tt DT}
is a data type, {\tt I} an indicator symbol. Then {\tt P} is the value
(property) stored under indicator {\tt I} in the property list of
data type {\tt DT}.
\item[DTPUT(DT,I,P):] Put data type property. {\tt DT} is a data type,
{\tt I} is an indicator under which value {\tt P} is stored in {\tt DT}'s
property list.
\item[STADD(DT,SS):] Add new sort to data type. The symbol {\tt SS}
is interpreted as a sort name and stored in the sort list of data type
{\tt DT}. It is not checked, whether the sort is already in the sort list
or not.
\item[XSTADD(DT,XS):] Add new external sort to data type. The symbol
{\tt XS} is interpreted as an external sort name and stored in the
external sort list of data type {\tt DT}. The symbol {\tt XS} is
assumed to have suitable properties (e.g. {\tt XREAD},{\tt XWRITE})
added. There is no occurrence check for symbol {\tt XS} in the external
sort list.
\end{description}

\subsection{Miscellaneous}
The function {\tt ITGLPR} (initialize global variables for parser) of file
{\tt pinit.ald} should be called before any other parser action
is started.
It should be called only once\footnote{Only in case the program uses
more than one global symbol table multiple calls to ITGLPR make sense.}.

There are further exported functions that control output modes. These functions
will be described in the next section.
