\section{User Extensions}

This section describes how to extend the parser to a programmer's
or user's special need. This is done exemplary for the case of an
additional property for operators. 
Other extensions require similar adaptations.

\begin{enumerate}
\item{\bf Adding new lexemes, tokens and environments to the scanner:}
 At first, a new environment should be made up, if the scanner demands
 this\footnote{See also section~\ref{environments} for more information.}.
 This should be necessary only in case of additional sections.
 To add new operator properties this is not necessary, as there is already
 a property section which presumably uses the same environment as this
 extension. Thus, it is sufficient to add new scanner rules and tokens.
 New rules have to be inserted in file {\tt scan.l}, whereas new tokens
 go to the respective section of {\tt parse.y}.

 When adding new rules to the scanner, care should be taken in
 adapting the look-ahead counter. See section \ref{extparse} for details.
\item{\bf Adapting the parser:} To parse further properties, new production
 rules have to be joined to the already declared ones. This is done
 in the property section of the YACC-parser grammar.
\item{\bf Semantic actions} have to be adapted accordingly. For operator
 properties, not even a change in function {\tt FXPROP} of file
 {\tt parsax.ald} is necessary, as this function already works with arbitrary
 properties (there is no consistency check!). Other extensions may
 require insertion of new functions in file {\tt check.ald}. It is recommended
 to take a look at other parser functions in order to understand
 what should be done in the files {\tt check.ald} and {\tt parsax.ald}.
\item{\bf Error checking:} Whenever additional lexemes were added in
 step one, appropriate syntactical error messages should be set up
 in file {\tt parse.y}. Semantic error messages must be inserted
 into file {\tt semerr.ald} and should be called from routines
 of {\tt check.ald} or {\tt parsax.ald}.
\end{enumerate}

Though this section tried to give a little help on user extensions,
it can obviously not be sufficient for all future implementations.
Examination of source code is still a good (but hard) way to get the
necessary information, however.
