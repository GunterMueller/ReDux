\section{Data Structures Used by Parser}

The new parser forced a change of fundamental \redux\ data structures
as well as the introduction of new ones. 
These structures are explained in the following.

\subsection{New Global Variables}
In order to handle new term properties like precedence or associativity,
additional notations and output modes, new global variables had to be
inserted.
See Table~\ref{globsym} for a list of new global symbols, and 
Appendix~\ref{globals} for a complete list of additional variables.

\begin{table}[htbp]
\begin{center}
\begin{tabular}{|l|l|l|l|}
\hline
Symbol Name & Print Name & Kind & Meaning \\
\hline \hline
DAM & AMBIG & indicator & ambiguity list \\ \hline
DBN & BRACKET & grammar symbol & dummy symbol: bracket object \\ \hline
DDF & DEFFIX & indicator & default fixity (notation) \\ \hline
DFF & FUNC & property & function notation \\ \hline
DLA & ALEFT & property & left associativity \\ \hline
DNA & ANONE & property & no associativity status \\ \hline
DOA & ASSOC & indicator & infix operator associativity \\ \hline
DOP & PREC & indicator & operator precedence \\ \hline
DRA & ARIGHT & property & right associativity \\ \hline
DXEF & XPREF & grammar symbol & external prefix operator \\ \hline
DXRF & XRNDF & grammar symbol & external roundfix operator \\ \hline
\end{tabular}
\end{center}
\caption{Global Symbols Used by Parser}
\label{globsym}
\end{table}

All new symbols are initialized by function {\tt ITGLPR} of file
{\tt pinit.ald}.

For compatibility reasons, symbols used by the old parser are still
maintained.

\subsection{New Term Properties and Usage of Old Properties}

All term properties which were available in the old version of the
parser keep their validity in the new version, too. Under indicator
{\tt DFS} (operator fixing) the new property {\tt DFF} can be set.
Additional term properties (respectively indicators) include {\tt DAM},
{\tt DDF}, {\tt DOA} and {\tt DOP}. All of these have been explained formerly.

\subsection{New Data Type Structure}
The parser now stores sort and external sort information explicitly
within the data type and makes use of a symbol list.
Therefore a change of the data structure for the data types was necessary. 
The new structure of data types can be seen in Table~\ref{datatype}.

\begin{table}[htbp]
\begin{center}
\begin{tabular}{|l|l|p{2in}|p{2in}|}
\hline
Field No. & Mnemonic & Name & Possible Entries\\
\hline \hline
1 & DTAX & data type axiom set & list of axioms \\ \hline
2 & DTCONST & data type constant set & list of constants \\ \hline
3 & DTVARS & data type variable set & list of variables \\ \hline
4 & DTOPERS & data type operator and coercion operator set 
  & list of (coercion) operators \\ \hline
5 & DTNAME & data type name & symbol denoting data type name \\ \hline
6 & DTSORTS & data type sorts & list of symbols denoting sorts \\ \hline
7 & DTXSORTS & data type external sorts
  & list of symbols denoting external sorts \\ \hline
8 & DTSYMLST & data type symbol list & list of signature objects \\ \hline
$>$8 & DTPROP & data type properties & property list \\ \hline
\end{tabular}
\end{center}
\caption{The New Data Type Structure}
\label{datatype}
\end{table}

\subsection{Parser Variable Dictionary Structure}
On declaring a variable with name $n$ in the {\tt VAR}-section of
a data type, indexed variables $n_i$ are declared implicitly, where
$n_0$ is identified with $n$.
This feature allows re-parsing terms containing auxiliary variables printed by
\redux.

Variables with same indices are represented uniquely, whereas variable
instances (variables with same name, but different index) share only a
common signature information ({\tt TSIGN} field). 
Their binding field ({\tt VARBIND} field) may be different.

To handle these variable instances the parser has to keep track of
which indices are used. This is done by a data structure called
{\em parser variable dictionary}.

A parser variable dictionary is a (possibly empty) list of period two
$$ PD = (n_1,vl_1,\dots,n_k,vl_k), $$
where at the odd positions
variable names (symbols $n_j$) are stored, and at the even positions
corresponding variable lists ($vl_j$). These variable lists
$$ vl_j = (i_{j1},v_{j1},\dots,i_{jl},v_{jl}) $$
contain variable indices ($i_{jm}$) and occurrences of variable signature
objects (variable terms) assigned to the variable with name $n_j$ 
and index $i_{jm}$.

It is not recommended to access parser variable dictionaries directly,
use the macros of {\tt parse.h} instead. The definition of these
macros is:
$$\begin{array}{rcl}
  \mbox{PDVARCONS}(n,vl,PD) & = & \begin{array}{lcl}
     (n,vl,n_1,vl_1,\dots,n_k,vl_k) & if & 
     PD = (n_1,vl_1,\dots,n_k,vl_k) \end{array} \\
  \mbox{PDVARNAME}(PD) & = & \left\{ \begin{array}{lcl}
     undefined & if & PD = () \\
     n_1 & if & PD = (n_1,vl_1,\dots,n_k,vl_k) \end{array}
   \right. \\
   \mbox{PDVARLIST}(PD) & = & \left\{ \begin{array}{lcl}
     undefined & if & PD = () \\
     vl_1 & if & PD = (n_1,vl_1,\dots,n_k,vl_k) \end{array}
   \right. \\
   \mbox{PDVARNEXT}(PD) & = & \left\{ \begin{array}{lcl}
     undefined & if & PD = () \\
     (n_2,vl_2,\dots,n_k,vl_k) & if & PD = (n_1,vl_1,\dots,n_k,vl_k)
   \end{array} \right. \\
   \\
  \mbox{PDVLCONS}(i,v,VL) & = & \begin{array}{lcl}
     (i,v,i_1,v_1,\dots,i_k,v_k) & if & 
     VL = (i_1,v_1,\dots,i_k,v_k) \end{array} \\
  \mbox{PDVLINDEX}(VL) & = & \left\{ \begin{array}{lcl}
     undefined & if & VL = () \\
     i_1 & if & VL = (i_1,v_1,\dots,i_k,v_k) \end{array}
   \right. \\
   \mbox{PDVLVAR}(VL) & = & \left\{ \begin{array}{lcl}
     undefined & if & VL = () \\
     v_1 & if & VL = (i_1,v_1,\dots,i_k,v_k) \end{array}
   \right. \\
   \mbox{PDVLNEXT}(VL) & = & \left\{ \begin{array}{lcl}
     undefined & if & VL = () \\
     (i_2,v_2,\dots,i_k,v_k) & if & VL = (i_1,v_1,\dots,i_k,v_k)
   \end{array} \right. \\   
\end{array}$$

It should be mentioned, that variable indices may differ between input
and output of terms. So a variable with name {\tt v\_100} during term parsing
may be written, for example, as {\tt v\_2}.

\subsection{Ambiguity List Structure}

To avoid ambiguous term output, which can be caused by writing
two identifiers (variable, constant, operator names,\dots)
directly one after the other without separating blanks or parentheses,
ambiguity lists are stored within the term structure (more exactly: within
signature objects), if necessary.
These ambiguity lists are added as a term property under indicator {\tt DAM}.

The ambiguity list of a symbol $s$ represents a set of symbol lists
such that $(s_1,\ldots,s_n)$ being in the ambiguity list of $s$ means that
if $s$ is to be printed followed by the symbols $s_1, \ldots, s_n$ then
a space charater must be written between $s$ and $s_1$.

Ambiguity lists are represented in a trie structure, thus
avoiding to dulpicate common prefixes of several different members.
An ambiguity list is a (possibly empty) list of period two
$$ AL = (n_1,al_1,\dots,n_k,al_k), $$
where the $n_i$ are object names (symbols) and the $al_i$ are further
ambiguity lists, called continuation lists.

Like parser variable dictionaries, ambiguity lists should not be accessed
directly. The parser provides the following macros to manipulate
them:

$$\begin{array}{rcl}
  \mbox{AMBCONS}(n,c,AL) & = & \begin{array}{lcl}
     (n,c,n_1,al_1,\dots,n_k,al_k) & if &
     AL = (n_1,al_1,\dots,n_k,al_k) \end{array} \\
  \mbox{AMBNAME}(AL) & = & \left\{ \begin{array}{lcl}
     undefined & if & AL = () \\
     n_1 & if & AL = (n_1,al_1,\dots,n_k,al_k) \end{array}
   \right. \\ 
  \mbox{AMBCONT}(AL) & = & \left\{ \begin{array}{lcl}
     undefined & if & AL = () \\
     al_1 & if & AL = (n_1,al_1,\dots,n_k,al_k) \end{array}
   \right. \\ 
  \mbox{AMBNEXT}(AL) & = & \left\{ \begin{array}{lcl}
     undefined & if & AL = () \\
     (n_2,al_2,\dots,n_k,al_k) & if & AL = (n_1,al_1,\dots,n_k,al_k)
  \end{array} \right. \\
\end{array}$$

The semantics of such a list can be defined as follows:
If $s$ is a signature object with name $n$ and ambiguity list 
$al=(n_1,al_1,\ldots,n_k,al_k)$,
then a separator (space or parentheses) has to be written
after $n$, if the sequence $(s_1,\dots,s_n)$ of symbols to be written
directly following $n$ on term output is in
% a maximal sequence of the set
\[
  \mbox{set}(al) = \{ (s_1,\dots,s_j)  \mid  \;
         \exists i: s_1 = n_i \land (s_2,\ldots,s_j) \in \mbox{set}(al_i) \}
\]
where $\mbox{set}(()) = \{ () \}$.

The current implementation may add additional alternatives to ambiguity
lists. As this case occurs rarely, it is an acceptable compromise
between simple algorithms and minimal ambiguity lists.